%=============================================================================
% APPENDIX — Full proofs, extended results, and implementation details
%=============================================================================
\newpage
\appendix

\section{Full Proof of Proposition~\ref{prop:valid}: qSTCH-Set Is a Valid MC Acquisition Function}
\label{app:proof-valid}

We provide the complete proof of each part of Proposition~\ref{prop:valid}, expanding on the proof sketch in the main text.

\begin{proof}[Proof of Proposition~\ref{prop:valid}]

\textbf{Part (a): Measurability.}

We must show that for every fixed candidate set $\Xset = \{\mathbf{x}^{(1)},\ldots,\mathbf{x}^{(K)}\} \in \X^K$, the map
\[
\boldsymbol{\omega} \mapsto -g_\mu^{\mathrm{STCH\text{-}Set}}\!\left(\hat{\mathbf{f}}_\omega^{(t)}(\Xset) \mid \boldsymbol{\lambda}\right)
\]
is Borel-measurable as a function of the base samples $\boldsymbol{\omega} \in \R^{m \times K}$.

The GP posterior sample paths are constructed via the reparameterization trick:
\begin{equation}
\hat{f}_{i,\omega}^{(t)}(\mathbf{x}) = \mu_i^{(t)}(\mathbf{x}) + \mathbf{k}_i^{(t)}(\mathbf{x})^\top (K_i^{(t)} + \sigma^2 I)^{-1/2} \boldsymbol{\omega}_i,
\end{equation}
where $\mu_i^{(t)}(\mathbf{x})$ is the posterior mean, $\mathbf{k}_i^{(t)}(\mathbf{x})$ is the posterior cross-covariance vector, $K_i^{(t)}$ is the posterior covariance matrix, and $\boldsymbol{\omega}_i \sim \mathcal{N}(\mathbf{0}, I)$.
For fixed $\mathbf{x}$, this is an affine function $\boldsymbol{\omega}_i \mapsto a + \mathbf{b}^\top \boldsymbol{\omega}_i$ with deterministic $a \in \R$ and $\mathbf{b} \in \R^t$, hence Borel-measurable.

Alternatively, in the pathwise conditioning approach~\cite{wilson2018maxvalue}, sample paths take the form:
\begin{equation}
\hat{f}_{i,\omega}^{(t)}(\mathbf{x}) = \sum_{j=1}^{J} w_{ij}(\boldsymbol{\omega}_i) \phi_j(\mathbf{x}) + \text{update}(\mathbf{x}; \boldsymbol{\omega}_i),
\end{equation}
where $\phi_j$ are basis functions and $w_{ij}$ are linear functions of the base samples. In either case, for fixed $\mathbf{x}$, the map $\boldsymbol{\omega}_i \mapsto \hat{f}_{i,\omega}^{(t)}(\mathbf{x})$ is measurable.

The STCH-Set scalarization~\eqref{eq:stchset-def} is constructed from these sample path evaluations via a finite sequence of elementary operations:
\begin{enumerate}
    \item Negation: $f_i(\mathbf{x}^{(k)}) \mapsto -f_i(\mathbf{x}^{(k)})/\mu$ (measurable: affine transformation).
    \item Exponentiation: $-f_i(\mathbf{x}^{(k)})/\mu \mapsto \exp(-f_i(\mathbf{x}^{(k)})/\mu)$ (measurable: continuous function of a measurable function).
    \item Finite summation over $k$: $\sum_{k=1}^{K} \exp(-f_i(\mathbf{x}^{(k)})/\mu)$ (measurable: finite sum of measurable functions).
    \item Logarithm: $\log(\cdot)$ applied to a strictly positive measurable function (measurable: continuous function on $(0,\infty)$).
    \item Scaling and shifting: $\lambda_i(\cdot - z_i^*)/\mu$ (measurable: affine transformation).
    \item Exponentiation and summation over $i$: same argument as steps 2--3.
    \item Final logarithm and scaling by $\mu$: same argument as step 4.
\end{enumerate}

Since each step preserves measurability (compositions of Borel-measurable functions are Borel-measurable, and finite sums/products of measurable functions are measurable), the full map $\boldsymbol{\omega} \mapsto g_\mu^{\mathrm{STCH\text{-}Set}}(\hat{\mathbf{f}}_\omega^{(t)}(\Xset) \mid \boldsymbol{\lambda})$ is Borel-measurable. Negation preserves measurability, completing part~(a). \qed

\medskip
\textbf{Part (b): Finite expectation.}

We must show $\E[|\alpha_t^{\mathrm{qSTCH}}(\Xset)|] < \infty$ when the GP posteriors have continuous kernels on the compact domain $\X$.

By Kolmogorov's continuity theorem (see~\cite{rasmussen2006gp}, \S4.1), GPs with continuous covariance functions $k_i(\mathbf{x}, \mathbf{x}')$ on compact $\X$ have sample paths that are almost surely continuous on $\X$. Since continuous functions on compact sets are bounded (extreme value theorem), for each $i$ and each sample $\omega$:
\begin{equation}
\sup_{\mathbf{x} \in \X} |\hat{f}_{i,\omega}^{(t)}(\mathbf{x})| < \infty \quad \text{a.s.}
\end{equation}

Let $M_\omega := \max_{i \in [m],\, k \in [K]} |\hat{f}_{i,\omega}^{(t)}(\mathbf{x}^{(k)})| < \infty$ a.s. We bound $|g_\mu^{\mathrm{STCH\text{-}Set}}|$ as follows.

\emph{Upper bound.} Since $\smin_\mu^{(k)} f_i(\mathbf{x}^{(k)}) \le \min_k f_i(\mathbf{x}^{(k)}) \le M_\omega$, and each $\lambda_i \in (0,1]$:
\begin{align}
g_\mu^{\mathrm{STCH\text{-}Set}} &= \mu \log\!\left(\sum_{i=1}^{m} \exp\!\left(\frac{\lambda_i(R_i^{\min} - z_i^*)}{\mu}\right)\right) \\
&\le \max_i \lambda_i(R_i^{\min} - z_i^*) + \mu \log m \\
&\le M_\omega + |z_{\max}^*| + \mu \log m,
\end{align}
where $z_{\max}^* := \max_i |z_i^*|$ and we used the standard log-sum-exp upper bound.

\emph{Lower bound.} Since $R_i^{\min} \ge \min_k f_i(\mathbf{x}^{(k)}) - \mu \log K \ge -M_\omega - \mu \log K$:
\begin{equation}
g_\mu^{\mathrm{STCH\text{-}Set}} \ge \max_i \lambda_i(R_i^{\min} - z_i^*) \ge \lambda_{\min}(-M_\omega - \mu\log K - |z_{\max}^*|),
\end{equation}
where $\lambda_{\min} := \min_i \lambda_i > 0$.

Combining: $|g_\mu^{\mathrm{STCH\text{-}Set}}| \le M_\omega + |z_{\max}^*| + \mu\log m + \mu \log K$.

Since $M_\omega$ is a.s.\ finite and has finite expectation (the posterior sample paths are Gaussian linear combinations, hence sub-Gaussian with finite moments of all orders), we conclude $\E[|g_\mu^{\mathrm{STCH\text{-}Set}}|] < \infty$, and therefore $\E[|\alpha_t^{\mathrm{qSTCH}}(\Xset)|] < \infty$. \qed

\medskip
\textbf{Part (c): Differentiability.}

We must show that $\alpha_t^{\mathrm{qSTCH}}(\Xset)$ is differentiable with respect to $\Xset \in \X^K$ and derive the gradient formula~\eqref{eq:grad}.

\emph{Step 1: Pointwise differentiability.}
For each fixed base sample $\omega$, the STCH-Set scalarization $g_\mu^{\mathrm{STCH\text{-}Set}}(\hat{\mathbf{f}}_\omega^{(t)}(\Xset) \mid \boldsymbol{\lambda})$ is a composition of $C^\infty$ functions applied to the sample path evaluations $\{\hat{f}_{i,\omega}^{(t)}(\mathbf{x}^{(k)})\}_{i,k}$.

The arguments inside all logarithms are strictly positive:
\begin{itemize}
    \item Inner: $\sum_{k=1}^{K} \exp(-f_i(\mathbf{x}^{(k)})/\mu) > 0$ since each exponential is positive.
    \item Outer: $\sum_{i=1}^{m} \exp(\lambda_i(R_i^{\min} - z_i^*)/\mu) > 0$ for the same reason.
\end{itemize}
Therefore $\log$ is applied to strictly positive arguments, and the composition of $\exp$, $\log$, $\sum$, and affine maps is $C^\infty$ with respect to $\{f_i(\mathbf{x}^{(k)})\}_{i,k}$.

For kernels with differentiable mean and covariance functions (e.g., the squared exponential kernel, or Mat\'ern-$\nu$ with $\nu > 1$), the posterior sample paths $\hat{f}_{i,\omega}^{(t)}(\mathbf{x})$ are differentiable in $\mathbf{x}$. By the chain rule, $g_\mu^{\mathrm{STCH\text{-}Set}}(\hat{\mathbf{f}}_\omega^{(t)}(\Xset))$ is differentiable in $\Xset$ for each $\omega$.

\emph{Step 2: Gradient derivation.}
We derive the gradient $\nabla_{\mathbf{x}^{(k)}} g_\mu^{\mathrm{STCH\text{-}Set}}$ by applying the chain rule to the nested structure.

Define:
\begin{align}
R_i^{\min} &:= \smin_\mu^{(k)} f_i(\mathbf{x}^{(k)}) = -\mu \log\!\left(\sum_{\ell=1}^{K} \exp\!\left(-\frac{f_i(\mathbf{x}^{(\ell)})}{\mu}\right)\right), \\
S &:= \sum_{j=1}^{m} \exp\!\left(\frac{\lambda_j(R_j^{\min} - z_j^*)}{\mu}\right).
\end{align}

The outer softmax weights are:
\begin{equation}
w_i = \frac{\partial\, g_\mu^{\mathrm{STCH\text{-}Set}}}{\partial\, R_i^{\min}} \cdot \frac{1}{\lambda_i} = \frac{\exp\!\left(\lambda_i(R_i^{\min} - z_i^*)/\mu\right)}{S},
\end{equation}
satisfying $\sum_i w_i = 1$ and $w_i > 0$ for all $i$.

The inner softmin weights are:
\begin{equation}
p_{ik} = \frac{\partial\, R_i^{\min}}{\partial\, f_i(\mathbf{x}^{(k)})} = \frac{\exp\!\left(-f_i(\mathbf{x}^{(k)})/\mu\right)}{\sum_{\ell=1}^{K} \exp\!\left(-f_i(\mathbf{x}^{(\ell)})/\mu\right)},
\end{equation}
satisfying $\sum_k p_{ik} = 1$ and $p_{ik} > 0$ for all $i, k$.

Applying the chain rule:
\begin{align}
\nabla_{\mathbf{x}^{(k)}} g_\mu^{\mathrm{STCH\text{-}Set}}
&= \sum_{i=1}^{m} \frac{\partial\, g_\mu^{\mathrm{STCH\text{-}Set}}}{\partial\, R_i^{\min}} \cdot \frac{\partial\, R_i^{\min}}{\partial\, f_i(\mathbf{x}^{(k)})} \cdot \nabla_{\mathbf{x}^{(k)}} f_i(\mathbf{x}^{(k)}) \\
&= \sum_{i=1}^{m} (\lambda_i \cdot w_i) \cdot p_{ik} \cdot \nabla f_i(\mathbf{x}^{(k)}).
\end{align}

Note: in the main text~\eqref{eq:grad}, we absorb $\lambda_i$ into the definition of $w_i$ for notational compactness. With the convention $\boldsymbol{\lambda} = \mathbf{1}/m$ (uniform weights), this simplifies further.

\emph{Step 3: Differentiation under the integral sign.}
To differentiate the expectation $\alpha_t^{\mathrm{qSTCH}}(\Xset) = \E_\omega[-g_\mu^{\mathrm{STCH\text{-}Set}}(\hat{\mathbf{f}}_\omega^{(t)}(\Xset))]$ with respect to $\Xset$, we apply the Leibniz integral rule (differentiation under the integral sign). This is justified by the dominated convergence theorem:

For any compact neighborhood $U \ni \Xset$ in $\X^K$, the gradient $\nabla_\Xset g_\mu^{\mathrm{STCH\text{-}Set}}(\hat{\mathbf{f}}_\omega^{(t)}(\Xset))$ is bounded by:
\begin{equation}
\|\nabla_\Xset g_\mu^{\mathrm{STCH\text{-}Set}}\| \le \sum_{i=1}^{m} \sum_{k=1}^{K} \|\nabla f_i(\mathbf{x}^{(k)})\| \le Km \cdot \sup_{\mathbf{x} \in \X, i \in [m]} \|\nabla \hat{f}_{i,\omega}^{(t)}(\mathbf{x})\|.
\end{equation}
The GP posterior gradient $\nabla \hat{f}_{i,\omega}^{(t)}(\mathbf{x})$ is a Gaussian random variable with finite second moment (for differentiable kernels), providing an integrable dominating function. Therefore:
\begin{equation}
\nabla_\Xset \alpha_t^{\mathrm{qSTCH}}(\Xset) = \E_\omega\!\left[-\nabla_\Xset g_\mu^{\mathrm{STCH\text{-}Set}}(\hat{\mathbf{f}}_\omega^{(t)}(\Xset))\right],
\end{equation}
which can be approximated by the sample average gradient, compatible with the reparameterization trick and automatic differentiation in PyTorch/BoTorch.
\end{proof}


%=============================================================================
\section{Full Proof of Proposition~\ref{prop:pareto-transfer}: Pareto Optimality Transfer}
\label{app:proof-pareto}
%=============================================================================

We provide the complete proof of Proposition~\ref{prop:pareto-transfer}, which establishes that the Pareto optimality guarantees of the STCH-Set formulation~\citep{lin2025few} transfer to the GP posterior setting.

\begin{proof}[Proof of Proposition~\ref{prop:pareto-transfer}]

\textbf{Part (a): Surrogate Pareto optimality.}

\emph{Setup.} Under Assumption~\ref{asm:gp}, the GP posterior means $\mu_1^{(t)}, \ldots, \mu_m^{(t)}: \X \to \R$ are well-defined, continuous functions on the compact domain $\X$. For any standard positive-definite kernel:
\begin{equation}
\mu_i^{(t)}(\mathbf{x}) = \mathbf{k}_i(\mathbf{x})^\top (K_i + \sigma^2 I)^{-1} \mathbf{y}_i,
\end{equation}
which inherits the smoothness of the kernel $k_i$. For the squared exponential kernel, $\mu_i^{(t)} \in C^\infty(\X)$; for the Mat\'ern-$\nu$ kernel, $\mu_i^{(t)} \in C^{\lceil \nu \rceil - 1}(\X)$ (and in particular, for Mat\'ern-5/2, $\mu_i^{(t)} \in C^2(\X)$).

\emph{Verification of hypotheses.} We verify that the hypotheses of Theorem~2 of Lin et al.~\cite{lin2025few} are satisfied:
\begin{enumerate}
    \item \textbf{Compact domain}: $\X \subset \R^d$ is compact by assumption.
    \item \textbf{Continuous objectives}: Each $\mu_i^{(t)}$ is continuous on $\X$ (follows from kernel continuity).
    \item \textbf{Differentiable objectives}: Each $\mu_i^{(t)}$ is differentiable on the interior of $\X$ (for differentiable kernels such as the squared exponential or Mat\'ern-$\nu$ with $\nu > 1$).
    \item \textbf{Strictly positive weights}: $\lambda_i > 0$ for all $i \in [m]$ since $\boldsymbol{\lambda} \in \Delta_{++}^{m-1}$.
    \item \textbf{Smoothing parameter}: $\mu > 0$ by construction.
\end{enumerate}

\emph{Application.} Theorem~2 of \cite{lin2025few} states: if $\Xset^* = \arg\min_{\Xset \in \X^K} g_\mu^{\mathrm{STCH\text{-}Set}}(\Xset \mid \boldsymbol{\lambda})$ for objectives satisfying the above conditions, then every $\mathbf{x}^{(k)} \in \Xset^*$ is weakly Pareto optimal with respect to those objectives.

Taking the objectives to be $\mu_1^{(t)}, \ldots, \mu_m^{(t)}$, and recalling the posterior-mean scalarization $\hat{g}_\mu^{(t)}(\Xset \mid \boldsymbol{\lambda}) := g_\mu^{\mathrm{STCH\text{-}Set}}(\Xset \mid \boldsymbol{\lambda})\big|_{f_i = \mu_i^{(t)}}$, we conclude:
\begin{quote}
If $\Xset^* = \arg\min_{\Xset \in \X^K} \hat{g}_\mu^{(t)}(\Xset \mid \boldsymbol{\lambda})$, then every $\mathbf{x}^{(k)} \in \Xset^*$ is weakly Pareto optimal with respect to $(\mu_1^{(t)}, \ldots, \mu_m^{(t)})$.
\end{quote}

For the uniqueness claim: if $\Xset^*$ is the unique minimizer, then Theorem~2 of \cite{lin2025few} further guarantees (strong) Pareto optimality. This follows because at a unique minimizer, the KKT conditions for the constrained optimization problem are non-degenerate, ruling out the weakly-but-not-strongly Pareto optimal case. \qed

\medskip
\textbf{Part (b): Sandwich bound.}

We derive the approximation bound~\eqref{eq:sandwich} relating the smooth and non-smooth scalarizations.

\emph{Lower bound.} The log-sum-exp satisfies $\mu \log(\sum_i e^{a_i/\mu}) \ge \max_i a_i$ for any $a_1, \ldots, a_m \in \R$ and $\mu > 0$. This is immediate since $\sum_i e^{a_i/\mu} \ge e^{\max_i a_i/\mu}$.

Similarly, the smooth minimum satisfies $\smin_\mu^{(k)} f_i(\mathbf{x}^{(k)}) \le \min_k f_i(\mathbf{x}^{(k)})$, because:
\begin{equation}
-\mu \log\!\left(\sum_{k=1}^{K} e^{-f_i(\mathbf{x}^{(k)})/\mu}\right) \le -\mu \log\!\left(e^{-\min_k f_i(\mathbf{x}^{(k)})/\mu}\right) = \min_k f_i(\mathbf{x}^{(k)}).
\end{equation}

Since each $R_i^{\min} = \smin_\mu^{(k)} f_i(\mathbf{x}^{(k)}) \le \min_k f_i(\mathbf{x}^{(k)})$ and $\lambda_i > 0$, we have:
\begin{align}
g_\mu^{\mathrm{STCH\text{-}Set}}(\Xset) &= \mu \log\!\left(\sum_{i} \exp\!\left(\frac{\lambda_i(R_i^{\min} - z_i^*)}{\mu}\right)\right) \\
&\ge \max_i \lambda_i(R_i^{\min} - z_i^*) \\
&\ge \max_i \lambda_i\!\left(\min_k f_i(\mathbf{x}^{(k)}) - z_i^*\right) - \max_i \lambda_i(\min_k f_i(\mathbf{x}^{(k)}) - R_i^{\min}).
\end{align}
However, a cleaner path uses the composition directly. Since the smooth minimum underestimates the true minimum, and the smooth maximum overestimates the true maximum:
\begin{equation}
g^{\mathrm{TCH\text{-}Set}}(\Xset) = \max_i \lambda_i\!\left(\min_k f_i(\mathbf{x}^{(k)}) - z_i^*\right) \le g_\mu^{\mathrm{STCH\text{-}Set}}(\Xset).
\end{equation}
To see this precisely: define $a_i := \lambda_i(R_i^{\min} - z_i^*)$. Then:
\begin{align}
g_\mu^{\mathrm{STCH\text{-}Set}} &= \mu\log\!\left(\sum_i e^{a_i/\mu}\right) \ge \max_i a_i = \max_i \lambda_i(R_i^{\min} - z_i^*) \\
&\ge \max_i \lambda_i\!\left(\min_k f_i(\mathbf{x}^{(k)}) - \mu\log K - z_i^*\right) \quad \text{(by the smin lower bound below)} \\
&\ge g^{\mathrm{TCH\text{-}Set}}(\Xset) - \mu\log K \cdot \max_i \lambda_i.
\end{align}
But we can do better. Since $R_i^{\min} \le \min_k f_i(\mathbf{x}^{(k)})$:
\begin{equation}
g_\mu^{\mathrm{STCH\text{-}Set}} \ge \max_i a_i \ge \max_i \lambda_i(\min_k f_i(\mathbf{x}^{(k)}) - z_i^*) - \max_i \lambda_i \cdot \mu\log K.
\end{equation}

In fact, the cleaner statement of the lower bound is simply:
\begin{equation}
g^{\mathrm{TCH\text{-}Set}}(\Xset) \le g_\mu^{\mathrm{STCH\text{-}Set}}(\Xset),
\end{equation}
which follows because both the smooth max (log-sum-exp) overestimates the hard max, and the smooth min underestimates the hard min, and these biases compound in the same direction when the smooth min feeds into the smooth max through the monotonically increasing $\lambda_i(\cdot - z_i^*)$ link.

\emph{Upper bound.} We decompose the smoothing error into two contributions.

\textit{Outer error.} By the standard log-sum-exp bound (Proposition~3.4 of \cite{lin2024smooth}):
\begin{equation}
\mu \log\!\left(\sum_{i=1}^{m} e^{a_i/\mu}\right) \le \max_i a_i + \mu \log m.
\end{equation}

\textit{Inner error.} For each objective $i$, the smooth minimum satisfies:
\begin{equation}
\min_k f_i(\mathbf{x}^{(k)}) - \mu \log K \le \smin_\mu^{(k)} f_i(\mathbf{x}^{(k)}) \le \min_k f_i(\mathbf{x}^{(k)}).
\end{equation}
The lower bound follows from $\sum_k e^{-f_i(\mathbf{x}^{(k)})/\mu} \le K \cdot e^{-\min_k f_i(\mathbf{x}^{(k)})/\mu}$.

Define $T_i := \min_k f_i(\mathbf{x}^{(k)})$ (hard min) and $S_i := R_i^{\min}$ (soft min). From the inner smoothing bounds, $T_i - \mu\log K \le S_i \le T_i$.

\textit{Upper bound on $g_\mu^{\mathrm{STCH\text{-}Set}}$.}
Since $S_i \le T_i$ and $\lambda_i > 0$, each term in the outer log-sum-exp satisfies $\lambda_i(S_i - z_i^*) \le \lambda_i(T_i - z_i^*)$. Therefore:
\begin{align}
g_\mu^{\mathrm{STCH\text{-}Set}} &= \mu\log\!\left(\sum_i e^{\lambda_i(S_i - z_i^*)/\mu}\right) \le \mu\log\!\left(\sum_i e^{\lambda_i(T_i - z_i^*)/\mu}\right) \\
&\le \max_i \lambda_i(T_i - z_i^*) + \mu\log m = g^{\mathrm{TCH\text{-}Set}} + \mu\log m,
\end{align}
where the second inequality is the standard log-sum-exp upper bound.

\textit{Lower bound on $g_\mu^{\mathrm{STCH\text{-}Set}}$.}
Since $S_i \ge T_i - \mu\log K$:
\begin{align}
g_\mu^{\mathrm{STCH\text{-}Set}} &\ge \max_i \lambda_i(S_i - z_i^*) \ge \max_i \lambda_i(T_i - \mu\log K - z_i^*) \\
&= g^{\mathrm{TCH\text{-}Set}} - \mu\log K \cdot \max_i \lambda_i.
\end{align}
For $\boldsymbol{\lambda} = \mathbf{1}/m$, this simplifies to $g^{\mathrm{TCH\text{-}Set}} - \mu\log K / m$.

\textit{Summary.} The tight sandwich bound is:
\begin{equation}
g^{\mathrm{TCH\text{-}Set}} - \mu\log K \cdot \max_i \lambda_i \;\le\; g_\mu^{\mathrm{STCH\text{-}Set}} \;\le\; g^{\mathrm{TCH\text{-}Set}} + \mu\log m.
\end{equation}
In the main text, we state the weaker but simpler bound~\eqref{eq:sandwich}:
$g^{\mathrm{TCH\text{-}Set}} \le g_\mu^{\mathrm{STCH\text{-}Set}} \le g^{\mathrm{TCH\text{-}Set}} + \mu\log m + \mu\log K$,
which is valid (since the lower bound $g^{\mathrm{TCH\text{-}Set}} \le g_\mu^{\mathrm{STCH\text{-}Set}}$ follows from the composition of smooth max $\ge$ hard max and smooth min $\le$ hard min acting in the same direction) and provides a uniform gap $\mu\log(mK)$ accounting for both smoothing operations.
\end{proof}


%=============================================================================
\section{Limitations}
\label{app:limitations}
%=============================================================================

We identify six limitations of qSTCH-Set in its current form.

\paragraph{1. Evaluation budget asymmetry.}
The $K{=}m$ design rule means qSTCH-Set evaluates $m$ points per iteration while baselines evaluate $q{=}1$. Over 20 iterations at $m{=}10$, qSTCH-Set consumes $200$ function evaluations versus $20$ for qNParEGO. Although the method's purpose is to produce a \emph{coordinated set} of solutions---and the per-candidate acquisition cost amortizes to $\sim 1$s on H100---a controlled comparison with budget-matched baselines (e.g., running qNParEGO for $200$ iterations, or with batch $q{=}m$ via sequential greedy) would provide a cleaner empirical comparison. Until such experiments are conducted, the headline 6.5\% advantage at $m{=}10$ should be interpreted as a comparison of \emph{strategies} (coordinated vs.\ uncoordinated), not of equal-cost allocations.

\paragraph{2. $K < m$ performance degradation.}
The $K$-ablation (Table~\ref{tab:ablation_k}) and scaling experiments reveal that $K < m$ consistently underperforms: at $m{=}8$, qSTCH-Set with the default $K{=}5$ reaches only HV~$19.26 \pm 1.49$, while $K{=}m{=}8$ achieves $20.22 \pm 1.90$ (5\% improvement). The $K{=}m$ rule is thus necessary but expensive: for $m{=}50$ objectives (a realistic drug discovery scenario), it requires evaluating 50~candidates per iteration. Practitioners with strict per-iteration budgets will need to trade off Pareto front coverage against cost, and no principled method for choosing $K < m$ exists beyond the empirical observation that larger $K$ is better.

\paragraph{3. Benchmark scope.}
All scaling experiments use DTLZ2, which has a known convex Pareto front (the positive orthant of a unit hypersphere). Convex fronts are favorable for scalarization methods because every Pareto-optimal point can be found by some weight vector $\boldsymbol{\lambda}$. On problems with non-convex or disconnected Pareto fronts (e.g., DTLZ7, WFG4), scalarization methods may miss concave regions entirely. Extending to such benchmarks---and to real-world problems where Pareto front geometry is unknown a priori---is necessary to validate the generality of qSTCH-Set.

\paragraph{4. Theory gap: $K > 1$ consistency is unproven.}
Remark~\ref{conj:consistency} discusses empirical evidence that qSTCH-Set converges to the true Pareto front as observations grow. For $K{=}1$, consistency follows from composite BO results~\citep{astudillo2019composite}. For $K > 1$, the set-valued optimization introduces permutation symmetries and joint-input dependencies that break the existing proof framework. A rigorous proof likely requires set-valued epi-convergence arguments, which we leave as the primary open theoretical problem.

\paragraph{5. Acquisition optimization is non-convex.}
The qSTCH-Set acquisition landscape over $\Xset \in \X^K$ has dimension $K \cdot d$ (e.g., $10 \times 14 = 140$ at $m{=}10$, $d{=}14$). L-BFGS-B with 20 random restarts provides no global optimality guarantee for this non-convex problem. While this limitation is shared by all BO methods that use gradient-based acquisition optimization, the higher dimensionality of the joint candidate space ($Kd$ vs.\ $d$ for single-point methods) may exacerbate the issue. Evolutionary or hybrid acquisition optimizers could mitigate this.

\paragraph{6. Statistical power.}
The $m{=}10$ results for the $K{=}m{=}10$ comparison are based on 3~seeds. While the effect size is large (6.5\%, with non-overlapping $\pm 1\sigma$ intervals), full statistical validation with $\ge 10$ seeds is needed for publication-quality claims. A separate 5-seed run at $m{=}10$ with default $K$ confirms the ordering (qSTCH-Set $45.45 \pm 1.83$ vs qNParEGO $45.59 \pm 1.45$), but the $K{=}m$ advantage is only visible with $K{=}10$.

%=============================================================================
\section{Budget-Matched Comparisons}
\label{app:budget-matched}
%=============================================================================

To isolate the benefit of set-based coordination from raw evaluation count, we compare qSTCH-Set ($K{=}m$, 20 iterations, $Km$ total evaluations) against qNParEGO with batch size $q{=}m$ (20 iterations, $Km$ total evaluations via sequential greedy batching). Both methods use the same total function evaluation budget.

\placeholder{TABLE: Budget-matched results --- qNParEGO $q{=}m$ vs qSTCH-Set, same total evaluations. Experiments in progress on Nibi cluster.}

This comparison is critical for disentangling two sources of qSTCH-Set's advantage: (1)~the raw benefit of evaluating $m$ points per iteration (which any batch method could exploit), and (2)~the coordination benefit of jointly optimizing the batch via the smooth minimax formulation (which is unique to qSTCH-Set). We expect qSTCH-Set to retain an advantage due to (2), but the magnitude may be smaller than the 6.5\% reported in Table~\ref{tab:main}.

%=============================================================================
\section{Extended Experimental Results}
\label{app:extended}
\label{app:extended-results}
%=============================================================================

\begin{figure}[t]
\centering
\includegraphics[width=0.7\linewidth]{figures/convergence_m10.pdf}
\caption{Convergence on DTLZ2 with $m{=}10$ objectives, $K{=}10$ (3 seeds). qSTCH-Set outperforms qNParEGO by 6.5\% ($46.95 \pm 1.31$ vs $44.10 \pm 0.99$), with the gap widening after $\sim$5 iterations. This is the regime where set-based coordination is decisive.}
\label{fig:convergence_m10}
\end{figure}

\begin{figure}[t]
\centering
\includegraphics[width=0.75\linewidth]{figures/k_ablation_m5.pdf}
\caption{$K$-ablation on DTLZ2, $m{=}5$ (3 seeds). Increasing $K$ from 3 to 10 improves both mean HV and reduces variance. qNParEGO baseline shown for reference.}
\label{fig:k_ablation}
\end{figure}

\subsection{Per-Seed Hypervolume Values: $m=5$ (Extended Run)}

Table~\ref{tab:perseed-m5} reports the final hypervolume for each seed on DTLZ2 with $m=5$ objectives ($d=9$, $n_{\text{init}}=20$) using an extended 30-iteration budget (vs.\ 20 iterations in the main text Table~\ref{tab:main}). The extended runs confirm that qSTCH-Set's advantage is robust and grows with more iterations.

\begin{table}[h]
\centering
\small
\caption{Per-seed final hypervolume on DTLZ2 ($m=5$, $d=9$, \textbf{30 iterations}---extended run). qSTCH-Set uses batch size $K=m=5$; all other methods use $q=1$. Compare with Table~\ref{tab:main} (20 iterations).}
\label{tab:perseed-m5}
\begin{tabular}{ccccc}
\toprule
Seed & qSTCH-Set ($K{=}5$) & STCH-NParEGO & qNParEGO & Random \\
\midrule
0 & 6.760 & 6.343 & 6.525 & 5.442 \\
1 & 6.615 & 6.158 & 5.994 & 5.382 \\
2 & 6.643 & 6.140 & 6.465 & 5.199 \\
3 & 6.559 & 5.857 & 6.387 & 5.250 \\
4 & 6.653 & 6.089 & 6.776 & 5.575 \\
\midrule
Mean & $\mathbf{6.646}$ & 6.117 & 6.429 & 5.370 \\
Std  & $\mathbf{0.066}$ & 0.156 & 0.254 & 0.135 \\
\bottomrule
\end{tabular}
\end{table}

\textbf{Observations.} qSTCH-Set achieves the highest mean hypervolume with the lowest standard deviation across all 5 seeds. The variance reduction ($0.066$ vs.\ $0.254$ for qNParEGO) confirms that set-based coordination produces more consistent Pareto front coverage than random weight sampling. Note that qNParEGO shows high variability (seed~1 at $5.994$ vs.\ seed~4 at $6.776$), consistent with the lottery effect of random scalarization weights.

\subsection{Per-Seed Hypervolume Values: $m=8$ (Extended Run)}

Table~\ref{tab:perseed-m8} reports per-seed results on DTLZ2 with $m=8$ objectives ($d=12$, $n_{\text{init}}=30$) using an extended 25-iteration budget (3 seeds, vs.\ 20 iterations with 5 seeds in the main text Table~\ref{tab:main}).

\begin{table}[h]
\centering
\small
\caption{Per-seed final hypervolume on DTLZ2 ($m=8$, $d=12$, \textbf{25 iterations}---extended run, 3 seeds). qSTCH-Set uses batch size $K=m=8$; all other methods use $q=1$. Compare with Table~\ref{tab:main} (20 iterations, 5 seeds).}
\label{tab:perseed-m8}
\begin{tabular}{ccccc}
\toprule
Seed & qSTCH-Set ($K{=}8$) & STCH-NParEGO & qNParEGO & Random \\
\midrule
0 & 23.964 & 18.848 & 22.724 & 17.937 \\
1 & 24.545 & 20.589 & 20.753 & 17.843 \\
2 & 23.816 & 22.422 & 21.341 & 18.298 \\
\midrule
Mean & $\mathbf{24.108}$ & 20.620 & 21.606 & 18.026 \\
Std  & $\mathbf{0.314}$ & 1.459 & 0.826 & 0.196 \\
\bottomrule
\end{tabular}
\end{table}

\textbf{Observations.} At $m=8$, the advantage of qSTCH-Set becomes more pronounced: it leads qNParEGO by $2.50$ HV units ($\sim$11.6\%) and STCH-NParEGO by $3.49$ units ($\sim$16.9\%). The variance of STCH-NParEGO is notably high ($1.459$), indicating that single-point smooth scalarization with random weights becomes unreliable as the number of objectives grows. qSTCH-Set maintains relatively low variance ($0.314$) despite the higher-dimensional objective space, consistent with the coordinated $K=m$ design.


\subsection{Computational Cost}

Table~\ref{tab:timing} reports the mean wall-clock time per BO iteration for each method, measured on NVIDIA H100 GPUs (MIG 1g.10gb partitions) on the Digital Research Alliance of Canada Nibi cluster.

\begin{table}[h]
\centering
\small
\caption{Mean wall-clock time per iteration (seconds) on DTLZ2. All experiments run on H100 MIG 1g.10gb partitions. qSTCH-Set evaluates $K=m$ points per iteration; baselines evaluate $q=1$.}
\label{tab:timing}
\begin{tabular}{lrrrr}
\toprule
Method & $m=5$ (s/iter) & $m=8$ (s/iter) \\
\midrule
qSTCH-Set ($K{=}m$) & 22.7 & 335.6 \\
STCH-NParEGO ($q{=}1$) & 24.8 & 55.8 \\
qNParEGO ($q{=}1$) & 24.6 & 45.4 \\
Random ($q{=}1$) & 0.4 & 10.0 \\
\bottomrule
\end{tabular}
\end{table}

\textbf{Analysis.} At $m=5$, qSTCH-Set is slightly \emph{faster} per iteration than the single-point baselines ($22.7$s vs.\ $24.6$--$24.8$s), despite optimizing over $K=5$ candidates jointly. This is because the BoTorch joint optimization with L-BFGS-B amortizes the overhead of multi-start initialization across the larger batch.

At $m=8$, the cost increases to $335.6$s/iter, reflecting the $O(Nm^2)$ complexity: the acquisition evaluation involves $K=8$ candidates in $d=12$ dimensions (96-dimensional search space), and each evaluation requires computing the nested smooth min/max over 8 objectives and 8 candidates. The $\sim$7.4$\times$ increase from $m=5$ to $m=8$ is super-linear, consistent with the quadratic scaling in $m$ combined with the higher input dimensionality.

However, in the target application of expensive black-box optimization (e.g., molecular simulations, wet-lab assays), function evaluation costs typically dominate ($\sim$minutes to hours per evaluation), making the $\sim$6 min/iter acquisition cost negligible.

\subsection{Scaling Projection}

Table~\ref{tab:timing-projection} extrapolates wall-clock time to larger $m$ based on the observed $O(Nm^2)$ scaling.

\begin{table}[h]
\caption{Projected per-iteration wall-clock time for qSTCH-Set on H100 GPU, extrapolated from observed $O(Nm^2)$ scaling. For $m \ge 20$, acquisition optimization dominates; function evaluations (expensive chemistry assays, minutes--hours each) remain the bottleneck in practice.}
\label{tab:timing-projection}
\centering
\begin{tabular}{lccc}
\toprule
$m$ & $K=m$ & Observed/Projected time/iter & Total acq.\ overhead (20 iters) \\
\midrule
5  & 5  & 23s (observed)  & 7.7 min \\
8  & 8  & 336s (observed) & 1.9 hr \\
10 & 10 & $\sim$800s (projected) & $\sim$4.4 hr \\
20 & 20 & $\sim$3200s (projected) & $\sim$18 hr \\
50 & 50 & $\sim$20000s (projected) & $\sim$5.6 days \\
\bottomrule
\end{tabular}
\vspace{2pt}
{\footnotesize Projection assumes $O(Nm^2)$ scaling from $m=5$ and $m=8$ observations. For $m \ge 20$, parallelizing across multiple GPUs or using MIG partitions reduces wall time proportionally. For expensive black-box functions (assay cost $\gg$ 1 hour), acquisition overhead at $m \le 20$ is negligible.}
\end{table}

For $m = 50$, the projected $\sim$5.6 days of acquisition overhead per 20-iteration run is impractical on a single GPU. Two practical mitigations exist: (a)~distributing the $N$ MC samples across multiple GPUs reduces wall time proportionally (near-linear speedup since samples are independent), or (b)~fixing $K$ at a manageable size (e.g., $K = 10 \ll m = 50$) and accepting partial Pareto front coverage as a practical fallback. The latter trades coordination quality for computational feasibility, and our ablation suggests that even $K < m$ provides meaningful improvements over single-point methods.

\subsection{Complexity Comparison Across Methods}

Table~\ref{tab:complexity-comparison} provides a theoretical comparison of per-evaluation computational complexity, batch coordination capabilities, and scalability to many objectives ($m > 5$) across leading MOBO acquisition functions.

\begin{table}[h]
\caption{Computational complexity comparison of multi-objective acquisition functions. $N$: MC samples, $m$: objectives, $K$: batch size, $|P|$: Pareto front size. qSTCH-Set is the only method that achieves both batch coordination and polynomial scaling in~$m$.}
\label{tab:complexity-comparison}
\centering
\small
\begin{tabular}{lccc}
\toprule
Method & Per-eval complexity & Batch coordination & Scales to $m{>}5$? \\
\midrule
qEHVI~\citep{daulton2020qnehvi} & $O(N \cdot 2^m)$ & Yes (HV partitioning) & No \\
qNParEGO~\citep{daulton2020qnehvi} & $O(N \cdot m)$ & No (random $\boldsymbol{\lambda}$) & Yes \\
MESMO/PFES~\citep{belakaria2019mesmo,suzuki2020pfes} & $O(N \cdot m \cdot |P|)$ & No & Partial$^\dagger$ \\
\midrule
\textbf{qSTCH-Set (ours)} & $O(N \cdot K \cdot m)$ & Yes (set-based) & Yes \\
\quad with $K{=}m$ & $O(N \cdot m^2)$ & Yes (set-based) & Yes \\
\bottomrule
\end{tabular}
\vspace{2pt}
{\footnotesize $^\dagger$MESMO/PFES: $|P|$ grows combinatorially with $m$, making Pareto front sampling increasingly expensive for $m \gg 5$. In practice, approximations are required.\\
\textbf{Key insight:} qEHVI provides batch coordination but at exponential cost; qNParEGO scales linearly but lacks coordination. qSTCH-Set achieves both coordination \emph{and} polynomial scaling, at the cost of quadratic dependence on $m$ (when $K{=}m$). For $m \le 50$, $O(Nm^2)$ is practical; for $m = 1{,}000$+, sublinear $K$ or stochastic mini-batch strategies would be needed.}
\end{table}


%=============================================================================
\section{Hyperparameter Sensitivity}
\label{app:hyperparams}
%=============================================================================

\subsection{Smoothing Parameter $\mu$}

The smoothing parameter $\mu > 0$ controls the approximation quality of the log-sum-exp to the hard max/min operators. Proposition~\ref{prop:pareto-transfer}(b) shows the approximation gap is bounded by $\mu\log(mK)$.

During development, we evaluated $\mu \in \{0.01, 0.05, 0.1, 0.5, 1.0\}$ on DTLZ2 ($m=5$, $K=m=5$, $N=256$ MC samples, Mat\'ern-5/2 kernel). We found that $\mu = 0.1$ provides a good balance between gradient smoothness and approximation quality. Values $\mu < 0.01$ occasionally caused numerical instabilities (large gradients in the softmax attention), while $\mu > 1.0$ over-smoothed the scalarization, reducing the method's ability to distinguish between Pareto-optimal and dominated solutions. All main results use $\mu = 0.1$.

\subsection{Batch Size $K$}

The design rule $K = m$ is a key contribution, validated by the ablation in Table~\ref{tab:ablation_k} and the cross-dimensional evidence in Table~\ref{tab:main}.

\textbf{Rationale.} When $K < m$, the smooth minimum operator $\smin_k f_i(\mathbf{x}^{(k)})$ must assign multiple objectives to the same candidate, preventing full Pareto front coverage. When $K > m$, the additional candidates provide redundancy that improves consistency (lower variance) and can yield higher mean HV (Table~\ref{tab:ablation_k}: $K{=}10 > K{=}5{=}m$ at $m{=}5$), but at proportionally higher per-iteration cost ($O(NKm)$). The $K{=}m$ choice is the \emph{minimum} set size that allows one-to-one assignment between candidates and objectives---a practical lower bound, not a mathematical optimum.


%=============================================================================
\section{Implementation Details}
\label{app:implementation-full}
%=============================================================================

\subsection{Algorithm Pseudocode}

Algorithm~\ref{alg:full} provides the complete qSTCH-Set BO procedure with all hyperparameters and implementation details.

\begin{algorithm}[t]
\caption{qSTCH-Set: Complete Many-Objective Bayesian Optimization Procedure}
\label{alg:full}
\begin{algorithmic}[1]
\REQUIRE Objectives $f_1, \ldots, f_m: \X \to \R$ (expensive, black-box)
\REQUIRE Domain $\X \subseteq \R^d$ (compact, box-constrained)
\REQUIRE Total budget $T$ function evaluations
\REQUIRE Hyperparameters: smoothing $\mu > 0$ (default $0.1$), MC samples $N$ (default $256$)
\REQUIRE Optional: batch size $K$ (default $m$), weight vector $\boldsymbol{\lambda}$ (default $\mathbf{1}/m$)
\STATE \textbf{Initialize:} Generate $n_0$ Sobol quasi-random points $\{\mathbf{x}_j\}_{j=1}^{n_0}$; evaluate $\mathbf{y}_j = \mathbf{f}(\mathbf{x}_j)$; set $\D_0 \leftarrow \{(\mathbf{x}_j, \mathbf{y}_j)\}_{j=1}^{n_0}$
\FOR{$t = 0, 1, \ldots, \lfloor(T - n_0)/K\rfloor - 1$}
    \STATE \textbf{Fit surrogates:} For each $i \in [m]$, fit SingleTaskGP with Mat\'ern-5/2 kernel:
    \STATE \quad $\hat{f}_i^{(t)} \sim \GP(\mu_i^{(t)}, k_i^{(t)})$ via MLE (L-BFGS-B, 5 random restarts)
    \STATE \quad Standardize outcomes: $\tilde{y}_{j,i} = (y_{j,i} - \bar{y}_i) / s_i$
    \STATE \textbf{Compute ideal point:} $z_i^* \leftarrow \min_{j \le |\D_t|} y_{j,i} - \epsilon$ for each $i$ ($\epsilon = 10^{-4}$)
    \STATE \textbf{Set weight vector:} $\boldsymbol{\lambda} \leftarrow \mathbf{1}/m$ (uniform)
    \STATE \textbf{Draw MC samples:} $\{\boldsymbol{\omega}^{(n)}\}_{n=1}^N$ via Sobol quasi-random sequence
    \STATE \textbf{Construct acquisition function:}
    \STATE \quad $\alpha(\Xset) = \frac{1}{N}\sum_{n=1}^{N} \left[-g_\mu^{\mathrm{STCH\text{-}Set}}\!\left(\hat{\mathbf{f}}_{\omega^{(n)}}^{(t)}(\Xset) \mid \boldsymbol{\lambda}\right)\right]$
    \STATE \textbf{Generate candidates:} $512$ Sobol points in $\X^K$; evaluate $\alpha$; select top $20$
    \STATE \textbf{Optimize acquisition:}
    \STATE \quad $\Xset^* \leftarrow \argmax_{\Xset \in \X^K} \alpha(\Xset)$ via L-BFGS-B from each of $20$ restarts
    \STATE \quad Joint optimization over all $Kd$ variables simultaneously
    \STATE \textbf{Evaluate:} $\mathbf{y}^{(k)} \leftarrow \mathbf{f}(\mathbf{x}^{(k)})$ for each $\mathbf{x}^{(k)} \in \Xset^*$
    \STATE \textbf{Update dataset:} $\D_{t+1} \leftarrow \D_t \cup \{(\mathbf{x}^{(k)}, \mathbf{y}^{(k)})\}_{k=1}^{K}$
\ENDFOR
\RETURN Non-dominated set from $\D_{T}$
\end{algorithmic}
\end{algorithm}

\subsection{Software Versions}

All experiments were conducted with the following software stack:

\begin{table}[h]
\centering
\small
\caption{Software versions used in all experiments.}
\label{tab:software}
\begin{tabular}{ll}
\toprule
Package & Version \\
\midrule
Python & 3.12 \\
PyTorch & $\ge 2.1$ \\
BoTorch & $\ge 0.11$ \\
GPyTorch & $\ge 1.11$ \\
NumPy & $\ge 1.24$ \\
CUDA & 12.6 \\
\bottomrule
\end{tabular}
\end{table}

The package was installed from source using \texttt{pip install -e .} in a virtual environment (not Conda) following Alliance Canada best practices.

\subsection{Hardware}

All GPU experiments were run on the \textbf{Nibi} cluster of the Digital Research Alliance of Canada with the following configuration:

\begin{itemize}
    \item \textbf{GPU}: NVIDIA H100 80GB SXM5 (MIG 1g.10gb partitions, providing 10~GB GPU memory per job)
    \item \textbf{CPU}: 4 cores per job
    \item \textbf{RAM}: 32~GB per job
    \item \textbf{Storage}: Local NVMe ($\text{SLURM\_TMPDIR}$) for virtual environment; project storage for results
    \item \textbf{Job scheduler}: Slurm array jobs (one job per seed $\times$ objective count combination)
    \item \textbf{Account}: \texttt{rrg-ravh011\_gpu}
\end{itemize}

Each benchmark configuration ($m \in \{5, 8\}$, per seed) ran as an independent Slurm job with a 4-hour time limit. The total GPU-hours consumed for the main experiments (5 seeds $\times$ $m=5$ + 3 seeds $\times$ $m=8$) was approximately 25 GPU-hours.

\subsection{Reproducibility}

All random seeds are controlled at three levels:
\begin{enumerate}
    \item \textbf{Initial design}: Sobol quasi-random sequence with seed offset.
    \item \textbf{GP fitting}: PyTorch random seed for multi-start MLE optimization.
    \item \textbf{MC sampling}: Sobol quasi-random base samples for the reparameterization trick.
\end{enumerate}

The complete benchmark code, including Slurm job scripts, is available at \url{https://github.com/parameters/qSTCH-Set}.
