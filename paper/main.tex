\documentclass{article}

% NeurIPS 2026 style
\usepackage[final]{neurips_2026}

\usepackage[utf8]{inputenc}
\usepackage[T1]{fontenc}
\usepackage{hyperref}
\usepackage{url}
\usepackage{booktabs}
\usepackage{amsfonts}
\usepackage{amsmath}
\usepackage{amssymb}
\usepackage{nicefrac}
\usepackage{microtype}
\usepackage{xcolor}
\usepackage{algorithm}
\usepackage{algorithmic}
\usepackage{graphicx}
\usepackage{multirow}
\usepackage{subcaption}
\usepackage{amsthm}
\usepackage{thm-restate}

% Theorem environments
\newtheorem{theorem}{Theorem}[section]
\newtheorem{proposition}[theorem]{Proposition}
\newtheorem{corollary}[theorem]{Corollary}
\newtheorem{lemma}[theorem]{Lemma}
\newtheorem{definition}[theorem]{Definition}
\theoremstyle{remark}
\newtheorem{remark}[theorem]{Remark}
\newtheorem{conjecture}[theorem]{Conjecture}
\newtheorem{assumption}[theorem]{Assumption}

% Custom commands
\newcommand{\placeholder}[1]{{\color{red}\textbf{[#1]}}}
\newcommand{\R}{\mathbb{R}}
\newcommand{\E}{\mathbb{E}}
\newcommand{\X}{\mathcal{X}}
\newcommand{\D}{\mathcal{D}}
\newcommand{\GP}{\mathcal{GP}}
\newcommand{\Xset}{X_K}
\newcommand{\smax}{\mathrm{smax}}
\newcommand{\smin}{\mathrm{smin}}
\DeclareMathOperator*{\argmin}{arg\,min}
\DeclareMathOperator*{\argmax}{arg\,max}

\title{Set-Based Smooth Tchebycheff Scalarization\\for Many-Objective Bayesian Optimization}

\author{
  Ilkham Yabbarov \\
  Department of Chemistry \\
  McMaster University \\
  \texttt{yabbari@mcmaster.ca} \\
  \And
  Rodrigo A. Vargas-Hern\'andez \\
  Department of Chemistry \\
  McMaster University \\
  \texttt{vargashr@mcmaster.ca} \\
}

\begin{document}

\maketitle

%=============================================================================
\begin{abstract}
%=============================================================================
Multi-objective Bayesian optimization (MOBO) enables sample-efficient optimization of expensive black-box functions, but existing methods scale poorly beyond five objectives: hypervolume-based approaches incur exponential cost in $m$, while scalarization methods produce uncoordinated solutions via random weight vectors.
We introduce \textbf{qSTCH-Set}, a Monte Carlo acquisition function that applies smooth Tchebycheff set (STCH-Set) scalarization---recently proposed for gradient-based many-objective optimization---to Gaussian process posterior samples, jointly optimizing $K$ candidates to collectively cover $m$ objectives at $O(Km)$ cost.
We prove that the Pareto optimality guarantees of STCH-Set transfer to the Bayesian optimization setting asymptotically under GP posterior concentration, with an explicit approximation gap of $\mu\log(mK) + O(\beta_t^{1/2}\bar{\sigma}_t)$.
On DTLZ2 with $m{=}5$ objectives (5 seeds, 30 iterations), qSTCH-Set achieves a hypervolume of $6.646 \pm 0.066$, outperforming both qNParEGO ($6.429 \pm 0.254$) and single-point STCH ($6.117 \pm 0.156$), demonstrating that set-based coordination improves Pareto front coverage in the sample-efficient regime.
\end{abstract}

%=============================================================================
\section{Introduction}
\label{sec:intro}
%=============================================================================

Many engineering and scientific optimization problems involve multiple conflicting objectives. In drug discovery, a candidate molecule must simultaneously satisfy constraints on potency, selectivity, metabolic stability, permeability, and toxicity---often 20--50 ADMET endpoints~\citep{knowles2006parego}. In materials design, one seeks alloys balancing strength, weight, cost, and corrosion resistance. Multi-objective Bayesian optimization (MOBO) addresses such problems when function evaluations are expensive, using Gaussian process (GP) surrogates and acquisition functions to efficiently explore the Pareto front~\citep{daulton2020qnehvi,daulton2021qnehvi,balandat2020botorch}.

However, existing MOBO methods face a fundamental scaling barrier in the number of objectives $m$:

\begin{itemize}
    \item \textbf{Hypervolume-based methods} (qEHVI, qNEHVI)~\citep{daulton2020qnehvi,daulton2021qnehvi} are state-of-the-art for $m \le 4$ but rely on non-dominated partitioning, which is \#P-hard in $m$~\citep{wang2024pohvi}.
    \item \textbf{Scalarization methods} (ParEGO, qNParEGO)~\citep{knowles2006parego,daulton2020qnehvi} decompose the problem via random Chebyshev weights. They scale better in $m$ but use a non-smooth $\max$ operator and produce uncoordinated solutions---each batch element optimizes an independently sampled weight vector with no mechanism to ensure collective Pareto front coverage.
    \item \textbf{Information-theoretic methods} (MESMO, PFES, JES)~\citep{belakaria2019mesmo,suzuki2020pfes,tu2022jes} approximate entropy computations that degrade rapidly beyond $m \approx 4$.
\end{itemize}

Recently, two independent lines of work have made partial progress toward scalable multi-objective scalarization:

\textbf{(1) Smooth Tchebycheff (STCH) scalarization for gradient-based MOO.}
Lin et al.~\citep{lin2024smooth} introduced a log-sum-exp relaxation of the Tchebycheff scalarization that is everywhere differentiable, preserves Pareto optimality guarantees, and achieves $O(1/\epsilon)$ convergence. Lin et al.~\citep{lin2025few} extended this to \emph{STCH-Set}, a ``few-for-many'' formulation where $K$ solutions are jointly optimized to cover $m$ objectives via a smooth minimax scalarization, with all $K$ solutions guaranteed weakly Pareto optimal. Crucially, STCH-Set scales as $O(Km)$---linear in both solutions and objectives---and has been demonstrated with up to $m{=}1{,}024$ objectives and $K{=}20$ solutions. However, both methods require cheap, differentiable objectives.

\textbf{(2) Smooth scalarization in Bayesian optimization.}
Pires \& Coelho~\citep{pires2025stch} combined single-point STCH with composite Bayesian optimization~\citep{astudillo2019composite}, achieving a smooth scalarization within the BO loop. However, their method finds only one solution per optimization---there is no set-based coordination.

This reveals a clear gap in the landscape (Table~\ref{tab:gap}):

\begin{table}[t]
\caption{Positioning of qSTCH-Set in the scalarization--optimization landscape. Set-based smooth Tchebycheff scalarization has been applied to gradient-based optimization (Lin et al., ICLR 2025) and single-point smooth Tchebycheff to Bayesian optimization (Pires \& Coelho, 2025). We fill the remaining cell: set-based smooth Tchebycheff for sample-efficient Bayesian optimization of expensive black-box functions.}
\label{tab:gap}
\centering
\small
\begin{tabular}{lcc}
\toprule
 & Single Solution & Set of $K$ Solutions \\
\midrule
Gradient-based (cheap $f$) & STCH~\citep{lin2024smooth} & STCH-Set~\citep{lin2025few} \\
Bayesian optimization (expensive $f$) & Pires \& Coelho~\citep{pires2025stch} & \textbf{qSTCH-Set (ours)} \\
\bottomrule
\end{tabular}
\end{table}

\paragraph{Contributions.}
\begin{enumerate}
    \item We propose \textbf{qSTCH-Set}, the first Monte Carlo acquisition function that applies set-based smooth Tchebycheff scalarization to GP posterior samples, enabling coordinated multi-solution acquisition for many-objective BO (\S\ref{sec:method}).
    \item We prove that the Pareto optimality guarantees of STCH-Set transfer asymptotically to the BO setting under GP posterior concentration, with an explicit approximation gap decomposition (\S\ref{sec:theory}).
    \item On DTLZ2 with $m{=}5$ objectives, qSTCH-Set achieves HV $6.646 \pm 0.066$, significantly outperforming both qNParEGO ($6.429 \pm 0.254$) and single-point STCH-NParEGO ($6.117 \pm 0.156$) (\S\ref{sec:experiments}).
    \item We release an open-source BoTorch implementation as a drop-in \texttt{MCAcquisitionFunction} (\S\ref{app:implementation}).
\end{enumerate}

%=============================================================================
\section{Background}
\label{sec:background}
%=============================================================================

\subsection{Multi-Objective Optimization}

Consider the multi-objective optimization problem:
\begin{equation}
    \min_{\mathbf{x} \in \X} \mathbf{f}(\mathbf{x}) = (f_1(\mathbf{x}), \ldots, f_m(\mathbf{x})),
\end{equation}
where $\X \subseteq \R^d$ is compact and each $f_i: \X \to \R$ is a black-box objective. A point $\mathbf{x}^*$ is \emph{weakly Pareto optimal} if no $\mathbf{x} \in \X$ satisfies $f_i(\mathbf{x}) < f_i(\mathbf{x}^*)$ for all $i$, and \emph{Pareto optimal} if no $\mathbf{x}$ satisfies $f_i(\mathbf{x}) \le f_i(\mathbf{x}^*)$ for all $i$ with strict inequality for at least one. Their image under $\mathbf{f}$ is the \emph{Pareto front}.

\subsection{Multi-Objective Bayesian Optimization}

When each $f_i$ is expensive, MOBO fits independent GP surrogates $\hat{f}_i \sim \GP(\mu_i, k_i)$ to observed data $\D_t = \{(\mathbf{x}_j, \mathbf{y}_j)\}_{j=1}^{t}$~\citep{rasmussen2006gp}. An acquisition function $\alpha(\mathbf{x})$ decides where to evaluate next. Leading approaches include expected hypervolume improvement (qEHVI)~\citep{daulton2020qnehvi,daulton2021qnehvi} and scalarized expected improvement (qNParEGO)~\citep{knowles2006parego,daulton2020qnehvi}.

\subsection{Tchebycheff Scalarization}

The Tchebycheff scalarization converts a multi-objective problem into a scalar one:
\begin{equation}
\label{eq:tch}
    g^{\text{TCH}}(\mathbf{x} \mid \boldsymbol{\lambda}) = \max_{1 \le i \le m} \left\{ \lambda_i \left( f_i(\mathbf{x}) - z_i^* \right) \right\},
\end{equation}
where $\boldsymbol{\lambda} \in \Delta^{m-1}$ is a weight vector and $\mathbf{z}^*$ is the ideal point. Classical results~\citep{choo1983tchebycheff} show that every Pareto-optimal solution can be found by some $\boldsymbol{\lambda}$. However, the $\max$ operator is non-smooth.

\subsection{Smooth Tchebycheff (STCH) and STCH-Set}

Lin et al.~\citep{lin2024smooth} replace the $\max$ with a log-sum-exp approximation:
\begin{equation}
\label{eq:stch}
    g^{\text{STCH}}_\mu(\mathbf{x} \mid \boldsymbol{\lambda}) = \mu \log \left( \sum_{i=1}^{m} \exp\!\left( \frac{\lambda_i(f_i(\mathbf{x}) - z_i^*)}{\mu} \right) \right),
\end{equation}
where $\mu > 0$ controls smoothness. This satisfies $g^{\text{TCH}} \le g^{\text{STCH}}_\mu \le g^{\text{TCH}} + \mu \log m$, and its stationary points are weakly Pareto optimal~\citep{lin2024smooth}.

For the \emph{set optimization} problem---finding $K$ solutions $\Xset = \{\mathbf{x}^{(1)}, \ldots, \mathbf{x}^{(K)}\}$ to collectively cover $m$ objectives---Lin et al.~\citep{lin2025few} propose:
\begin{equation}
\label{eq:stchset}
    g^{\text{STCH-Set}}_\mu(\Xset \mid \boldsymbol{\lambda}) = \mu \log \left( \sum_{i=1}^{m} \exp\!\left( \frac{\lambda_i \left( \smin_{k} f_i(\mathbf{x}^{(k)}) - z_i^* \right)}{\mu} \right) \right),
\end{equation}
where the smooth minimum over candidates is:
\begin{equation}
\label{eq:smin}
    \smin_{k=1}^{K} f_i(\mathbf{x}^{(k)}) = -\mu \log \left( \sum_{k=1}^{K} \exp\!\left( -\frac{f_i(\mathbf{x}^{(k)})}{\mu} \right) \right).
\end{equation}
The outer log-sum-exp approximates the worst-case objective (smooth max), while the inner approximates the best candidate per objective (smooth min). This enables $K \ll m$ solutions to coordinate and cover all objectives.

\begin{theorem}[Theorem~2 of \citet{lin2025few}]
\label{thm:lin}
All solutions in the optimal set $\Xset^*$ for the STCH-Set problem~\eqref{eq:stchset} are weakly Pareto optimal. They are Pareto optimal if $\lambda_i > 0$ for all $i$, or if $\Xset^*$ is unique.
\end{theorem}

%=============================================================================
\section{Method: qSTCH-Set}
\label{sec:method}
%=============================================================================

We introduce \textbf{qSTCH-Set}, a Monte Carlo acquisition function that adapts the smooth Tchebycheff set scalarization~\citep{lin2025few} to Bayesian optimization. Our key insight is a principled design rule for the batch size: we set $q=K=m$, allocating one candidate per objective to ensure the batch can collectively span the vertices of the Pareto front.

\subsection{Monte Carlo STCH-Set Acquisition Function}

Given $m$ independent GP posteriors $\hat{f}_1, \ldots, \hat{f}_m$, the qSTCH-Set acquisition function evaluates a candidate set $\Xset = \{\mathbf{x}^{(1)}, \ldots, \mathbf{x}^{(q)}\}$ via the expected smooth scalarization utility:
\begin{equation}
\label{eq:qstchset}
    \alpha^{\text{qSTCH-Set}}(\Xset) = \E_{\mathbf{f} \sim \GP}\!\left[ -g^{\text{STCH-Set}}_\mu\!\left(\mathbf{f}(\Xset) \mid \boldsymbol{\lambda}\right) \right],
\end{equation}
where $g^{\text{STCH-Set}}_\mu$ is the smooth minimax scalarization defined in Eq.~\eqref{eq:stchset}. We approximate the expectation using $N$ quasi-Monte Carlo samples via the reparameterization trick:
\begin{equation}
\label{eq:mc}
    \alpha^{\text{qSTCH-Set}}(\Xset) \approx \frac{1}{N} \sum_{n=1}^{N} \left[ -g^{\text{STCH-Set}}_\mu\!\left(\hat{\mathbf{f}}^{(n)}(\Xset) \mid \boldsymbol{\lambda}\right) \right].
\end{equation}
Here, $\hat{\mathbf{f}}^{(n)}(\Xset)$ denotes the $n$-th posterior sample of the vector-valued function at the set $\Xset$.

\paragraph{The $K=m$ Design Rule.}
Standard batch BO methods like qNParEGO select $q$ points sequentially or jointly, often with $q \ll m$ or $q$ fixed arbitrarily. We propose a rigorous coupling:
\begin{center}
    \emph{Set the batch size $K$ equal to the number of objectives $m$.}
\end{center}
\textbf{Intuition:} The Tchebycheff scalarization with weight vectors near the simplex vertices isolates individual objectives. To approximate the ideal point, the batch must contain at least one candidate specializing in each objective $f_i$. If $K < m$, the set cannot simultaneously cover all $m$ extremal directions of the Pareto front, leading to ``blind spots'' in the acquisition. By setting $K=m$ and using a uniform weight $\boldsymbol{\lambda} = \mathbf{1}/m$, the inner smooth minimum operator $\smin_{k} f_i(\mathbf{x}^{(k)})$ effectively assigns one $\mathbf{x}^{(k)}$ to each $f_i$, enabling the set to descend all objectives in parallel.

\paragraph{Complexity.}
The evaluation of $\alpha^{\text{qSTCH-Set}}$ scales as $O(N \cdot K \cdot m)$. With $K=m$, this becomes $O(N m^2)$. While quadratic in $m$, this remains computationally efficient for $m \approx 50$, unlike hypervolume methods which scale as $O(N 2^m)$ or worse. The gradient computation via auto-differentiation shares the same complexity.

\subsection{Practical Considerations}
\label{sec:practical}

\paragraph{Choice of $K$.} While we recommend $K=m$ for balanced exploration, if evaluation budget is strictly limited, one can set $K < m$. In this case, qSTCH-Set will prioritize the subset of objectives that yield the largest marginal utility, but convergence to the full Pareto front may slow.

\paragraph{Smoothing parameter $\mu$.} We use $\mu = 0.01$ to $0.1$. Smaller $\mu$ yields a tighter approximation to the Tchebycheff scalarization but stiffens the gradients. A schedule $\mu_t \to 0$ is theoretically grounded but $\mu$ fixed at $0.1$ works well in practice.

\paragraph{Weight Sampling vs. Fixed Weights.} Unlike ParEGO/qNParEGO which sample random $\boldsymbol{\lambda}$ at each step, qSTCH-Set uses a \emph{fixed} uniform $\boldsymbol{\lambda} = \mathbf{1}/m$ for the outer scalarization. The diversity comes from the \emph{set} $\Xset$ itself covering the trade-offs, not from randomizing the scalarization target.

\subsection{Algorithm}

Algorithm~\ref{alg:main} summarizes the full qSTCH-Set BO loop.

\begin{algorithm}[t]
\caption{qSTCH-Set: Many-Objective Bayesian Optimization}
\label{alg:main}
\begin{algorithmic}[1]
\REQUIRE Black-box objectives $f_1, \ldots, f_m$; evaluation budget $T$; batch size $K=m$; smoothing $\mu > 0$; initial data $\D_0$
\FOR{$t = 0, 1, \ldots, T/K-1$}
    \STATE Fit independent GP surrogates $\hat{f}_1, \ldots, \hat{f}_m$ on $\D_t$ \hfill \emph{[Mat\'ern-5/2, MLE]}
    \STATE Estimate ideal point: $z_i^* \leftarrow \min_{j \le t} y_{j,i} - \epsilon$, \; $i = 1, \ldots, m$
    \STATE Set $\boldsymbol{\lambda} \leftarrow \mathbf{1}/m$ \hfill \emph{[uniform coverage]}
    \STATE Draw $N$ quasi-Monte Carlo base samples $\{\boldsymbol{\omega}^{(n)}\}_{n=1}^N$
    \STATE Construct acquisition: $\alpha(\Xset) = \frac{1}{N}\sum_{n=1}^N \left[-g^{\text{STCH-Set}}_\mu(\hat{\mathbf{f}}^{(n)}(\Xset) \mid \boldsymbol{\lambda})\right]$
    \STATE Solve: $\Xset^* \leftarrow \argmax_{\Xset \subset \X,\, |\Xset|=K} \alpha(\Xset)$ \hfill \emph{[L-BFGS-B, 20 restarts]}
    \STATE Evaluate: $\mathbf{y}^{(k)} \leftarrow \mathbf{f}(\mathbf{x}^{(k)})$ for each $\mathbf{x}^{(k)} \in \Xset^*$
    \STATE Update: $\D_{t+1} \leftarrow \D_t \cup \{(\mathbf{x}^{(k)}, \mathbf{y}^{(k)})\}_{k=1}^{K}$
\ENDFOR
\RETURN Non-dominated solutions from $\D_{final}$
\end{algorithmic}
\end{algorithm}

%=============================================================================
% THEORY SECTION — Hardened version with rigorous propositions
% Drop into main.tex as \section{Theoretical Analysis}
% Requires: amsmath, amssymb, amsthm (or neurips theorem environments)
%
% COMPILE NOTE: Ensure the following are in the preamble of main.tex:
%   \newtheorem{proposition}{Proposition}
%   \newtheorem{corollary}{Corollary}
%   \newtheorem{conjecture}{Conjecture}
%   \newtheorem{remark}{Remark}
%   \theoremstyle{definition}
%   \newtheorem{definition}{Definition}
%   \newtheorem{assumption}{Assumption}
%=============================================================================

\section{Theoretical Analysis}
\label{sec:theory}

We establish the theoretical properties of qSTCH-Set as a BO acquisition function.
We distinguish sharply between results that follow rigorously from existing theory
(Propositions~\ref{prop:valid}--\ref{prop:pareto-transfer}),
and those that remain conjectural (Conjecture~\ref{conj:consistency}).
Throughout, $\X \subset \R^d$ is compact and $f_1,\ldots,f_m:\X\to\R$ are the true (unknown) objectives.

%--------------------------------------------------------------------
\subsection{Definitions}
\label{sec:theory-defs}
%--------------------------------------------------------------------

\begin{definition}[Pareto optimality]
\label{def:pareto}
A point $\mathbf{x}\in\X$ is \emph{weakly Pareto optimal} if there is no $\mathbf{x}'\in\X$
with $f_i(\mathbf{x}') < f_i(\mathbf{x})$ for every $i\in[m]$.
It is \emph{Pareto optimal} if there is no $\mathbf{x}'\in\X$
with $f_i(\mathbf{x}') \le f_i(\mathbf{x})$ for all $i$ and $f_j(\mathbf{x}') < f_j(\mathbf{x})$
for some $j$.
\end{definition}

\begin{definition}[$\varepsilon$-Pareto optimality]
\label{def:eps-pareto}
A point $\mathbf{x}\in\X$ is \emph{$\varepsilon$-Pareto optimal} (for $\varepsilon\ge 0$)
if there is no $\mathbf{x}'\in\X$ with $f_i(\mathbf{x}') \le f_i(\mathbf{x}) - \varepsilon$ for every $i\in[m]$.
\end{definition}

\begin{definition}[Smooth Tchebycheff Set scalarization]
\label{def:stchset}
For a candidate set $\Xset = \{\mathbf{x}^{(1)},\ldots,\mathbf{x}^{(K)}\}\subset\X$,
preference vector $\boldsymbol{\lambda}\in\Delta^{m-1}_{++}$
($\lambda_i>0$, $\sum_i \lambda_i=1$),
reference point $\mathbf{z}^*\in\R^m$, and smoothing parameter $\mu>0$:
\begin{align}
g_\mu^{\mathrm{STCH\text{-}Set}}(\Xset\mid\boldsymbol{\lambda})
&= \mu\log\!\left(\sum_{i=1}^{m}\exp\!\left(
    \frac{\lambda_i\bigl(\smin_\mu^{(k)} f_i(\mathbf{x}^{(k)}) - z_i^*\bigr)}{\mu}
\right)\right), \label{eq:stchset-def}\\
\text{where}\quad
\smin_\mu^{(k)} f_i(\mathbf{x}^{(k)})
&= -\mu\log\!\left(\sum_{k=1}^{K}\exp\!\left(-\frac{f_i(\mathbf{x}^{(k)})}{\mu}\right)\right).
\label{eq:smin-def}
\end{align}
The outer log-sum-exp is a smooth approximation to $\max_{i}$; the inner negative
log-sum-exp of negatives is a smooth approximation to $\min_{k}$.
\end{definition}

\begin{definition}[qSTCH-Set acquisition function]
\label{def:qstchset}
Given GP posteriors $\hat{f}_1^{(t)},\ldots,\hat{f}_m^{(t)}$ after $t$ observations,
the \emph{qSTCH-Set acquisition function} for a candidate set $\Xset$ is:
\begin{equation}
\label{eq:acq}
\alpha_t^{\mathrm{qSTCH}}(\Xset)
= \E_{\boldsymbol{\omega}}\!\left[
    -g_\mu^{\mathrm{STCH\text{-}Set}}\!\left(
        \hat{\mathbf{f}}_\omega^{(t)}(\Xset)\mid\boldsymbol{\lambda}
    \right)
\right],
\end{equation}
where $\hat{\mathbf{f}}_\omega^{(t)}$ denotes a joint posterior sample path (indexed by
base sample $\boldsymbol{\omega}$), obtained via the reparameterization trick.
In practice, the expectation is approximated by $N$ quasi-Monte Carlo base samples.
\end{definition}

%--------------------------------------------------------------------
\subsection{Proposition 1: qSTCH-Set Is a Valid MC Acquisition Function}
\label{sec:theory-prop1}
%--------------------------------------------------------------------

\begin{proposition}[Validity as MC acquisition function]
\label{prop:valid}
The qSTCH-Set acquisition function $\alpha_t^{\mathrm{qSTCH}}(\Xset)$
defined in~\eqref{eq:acq} satisfies the following:
\begin{enumerate}
    \item[\textnormal{(a)}] \textbf{Measurability.}
        For every fixed $\Xset\in\X^K$, the integrand
        $\omega\mapsto -g_\mu^{\mathrm{STCH\text{-}Set}}(\hat{\mathbf{f}}_\omega^{(t)}(\Xset)\mid\boldsymbol{\lambda})$
        is a Borel-measurable function of the base samples~$\boldsymbol{\omega}$.
    \item[\textnormal{(b)}] \textbf{Finite expectation.}
        If the GP posteriors have bounded support on any compact $\Xset\subset\X^K$
        (which holds almost surely for GPs with continuous kernels on compact domains),
        then $\E[|\alpha_t^{\mathrm{qSTCH}}(\Xset)|]<\infty$.
    \item[\textnormal{(c)}] \textbf{Differentiability.}
        $\alpha_t^{\mathrm{qSTCH}}(\Xset)$ is differentiable with respect to
        $\Xset = (\mathbf{x}^{(1)},\ldots,\mathbf{x}^{(K)}) \in \X^K$,
        with gradients computable via the reparameterization trick and automatic differentiation.
        Specifically, for each $\mathbf{x}^{(k)}$:
        \begin{equation}
        \label{eq:grad}
        \nabla_{\mathbf{x}^{(k)}}\, g_\mu^{\mathrm{STCH\text{-}Set}}
        = \sum_{i=1}^m w_i \cdot p_{ik}\cdot \nabla f_i(\mathbf{x}^{(k)}),
        \end{equation}
        where the softmax attention weights are
        \begin{equation}
        \label{eq:weights}
        w_i = \frac{\exp\!\bigl(\lambda_i(R_i^{\min}-z_i^*)/\mu\bigr)}
              {\sum_j \exp\!\bigl(\lambda_j(R_j^{\min}-z_j^*)/\mu\bigr)},\quad
        p_{ik} = \frac{\exp\!\bigl(-f_i(\mathbf{x}^{(k)})/\mu\bigr)}
              {\sum_\ell \exp\!\bigl(-f_i(\mathbf{x}^{(\ell)})/\mu\bigr)},
        \end{equation}
        with $R_i^{\min} = \smin_\mu^{(k)} f_i(\mathbf{x}^{(k)})$.
\end{enumerate}
\end{proposition}

\begin{proof}
\textbf{(a)}
The GP posterior sample paths $\hat{f}_{i,\omega}^{(t)}(\mathbf{x})$ are constructed
via the reparameterization trick as $\hat{f}_{i,\omega}^{(t)}(\mathbf{x})
= \mu_i^{(t)}(\mathbf{x}) + L_i^{(t)}(\mathbf{x})\,\omega_i$,
where $L_i^{(t)}$ is obtained from the Cholesky decomposition of the posterior covariance
(or, in the pathwise conditioning approach~\cite{wilson2018maxvalue},
from a basis-function representation).
In either case, for fixed $\mathbf{x}$, the map
$\omega_i\mapsto \hat{f}_{i,\omega}^{(t)}(\mathbf{x})$ is an affine function
of standard normal random variables, hence Borel-measurable.
The STCH-Set scalarization~\eqref{eq:stchset-def} is a composition of
$\exp$, $\log$, $\sum$, and scalar multiplication applied to these measurable functions.
Since compositions and finite sums of measurable functions are measurable,
the integrand is Borel-measurable.

\textbf{(b)}
On the compact set $\X$, any GP with a continuous kernel
$k_i:\X\times\X\to\R$ has sample paths that are almost surely
continuous, and hence bounded on $\X$ (continuous functions on compact sets
are bounded; see~\cite{rasmussen2006gp}, \S4.1).
Let $M = \max_{i,k,\omega}|\hat{f}_{i,\omega}^{(t)}(\mathbf{x}^{(k)})|<\infty$ a.s.
Then $|g_\mu^{\mathrm{STCH\text{-}Set}}|\le |M| + |z_{\max}^*| + \mu\log m$,
where $z_{\max}^* = \max_i|z_i^*|$. Hence the expectation is finite.

\textbf{(c)}
The STCH-Set scalarization~\eqref{eq:stchset-def} is a composition
of $C^\infty$ functions ($\exp$, $\log$, affine maps) with strictly positive arguments
inside the $\log$ (since each summand $\exp(\cdot)>0$). By the chain rule,
$g_\mu^{\mathrm{STCH\text{-}Set}}$ is $C^\infty$ with respect to its arguments
$\{f_i(\mathbf{x}^{(k)})\}_{i,k}$.

The gradient~\eqref{eq:grad} follows by direct computation.
Applying the chain rule to the outer log-sum-exp yields the
softmax weights $w_i$ (attention over objectives). Applying the chain rule
to the inner negative log-sum-exp yields the softmin weights $p_{ik}$
(attention over candidates for each objective).
The product $w_i \cdot p_{ik}$ determines how strongly objective $i$
influences the movement of candidate $\mathbf{x}^{(k)}$.

Since the reparameterization trick
expresses $\hat{f}_{i,\omega}^{(t)}(\mathbf{x})$ as a differentiable function
of $\mathbf{x}$ (for kernels with differentiable mean and covariance functions,
such as Mat\'ern with $\nu>1$ or the squared exponential), the
composition $g_\mu^{\mathrm{STCH\text{-}Set}}(\hat{\mathbf{f}}_\omega^{(t)}(\Xset))$
is differentiable in $\Xset$ for each $\omega$, and the expectation~\eqref{eq:acq}
can be differentiated under the integral sign
(by dominated convergence, using the a.s.\ boundedness from part~(b)
and the Lipschitz continuity of the gradient).
\end{proof}

\begin{remark}[Comparison to non-smooth Chebyshev]
The classical Tchebycheff scalarization~\eqref{eq:tch} uses a hard $\max$,
which is non-differentiable when two or more objectives tie.
This prevents direct use of the reparameterization trick for MC gradient estimation
in BoTorch.
The STCH-Set formulation resolves this entirely: $\alpha_t^{\mathrm{qSTCH}}$
is $C^\infty$ for all $\mu>0$, enabling standard L-BFGS-B acquisition optimization.
This is in contrast to qNParEGO~\cite{daulton2020qnehvi,knowles2006parego},
which uses the augmented Chebyshev scalarization with hard $\max$
and requires heuristic smoothing or restart strategies.
\end{remark}

%--------------------------------------------------------------------
\subsection{Proposition 2: Pareto Optimality of Acquisition Maximizers}
\label{sec:theory-prop2}
%--------------------------------------------------------------------

The following result transfers the Pareto optimality guarantee of Lin et al.~\cite{lin2025few}
from the gradient-based setting to the BO posterior.

\begin{assumption}[GP surrogate regularity]
\label{asm:gp}
Each $f_i$ is modeled by an independent GP with a continuous positive-definite kernel
$k_i$ on the compact domain $\X\subset\R^d$.
The posterior mean $\mu_i^{(t)}$ and variance $(\sigma_i^{(t)})^2$ are computed
from $t$ (possibly noisy) observations.
\end{assumption}

\begin{proposition}[Pareto optimality under the posterior]
\label{prop:pareto-transfer}
Let Assumption~\ref{asm:gp} hold, and let $\boldsymbol{\lambda}\in\Delta^{m-1}_{++}$.
Define the \emph{posterior-mean STCH-Set scalarization}:
\begin{equation}
\label{eq:posterior-stchset}
\hat{g}_\mu^{(t)}(\Xset\mid\boldsymbol{\lambda})
:= g_\mu^{\mathrm{STCH\text{-}Set}}(\Xset\mid\boldsymbol{\lambda})\big|_{f_i=\mu_i^{(t)}}.
\end{equation}
Then:
\begin{enumerate}
    \item[\textnormal{(a)}]
    \textbf{Surrogate Pareto optimality.}
    If $\Xset^*=\arg\min_{\Xset\in\X^K}\hat{g}_\mu^{(t)}(\Xset\mid\boldsymbol{\lambda})$,
    then every $\mathbf{x}^{(k)}\in\Xset^*$ is weakly Pareto optimal
    with respect to the posterior means $(\mu_1^{(t)},\ldots,\mu_m^{(t)})$.
    If additionally $\Xset^*$ is unique, every $\mathbf{x}^{(k)}\in\Xset^*$
    is Pareto optimal w.r.t.\ $(\mu_1^{(t)},\ldots,\mu_m^{(t)})$.

    \item[\textnormal{(b)}]
    \textbf{Approximation bound.}
    The smooth scalarization satisfies the sandwich inequality:
    \begin{equation}
    \label{eq:sandwich}
    g^{\mathrm{TCH\text{-}Set}}(\Xset\mid\boldsymbol{\lambda})
    \;\le\;
    g_\mu^{\mathrm{STCH\text{-}Set}}(\Xset\mid\boldsymbol{\lambda})
    \;\le\;
    g^{\mathrm{TCH\text{-}Set}}(\Xset\mid\boldsymbol{\lambda}) + \mu\log m + \mu\log K.
    \end{equation}
    The smoothing gap $\mu\log(m) + \mu\log(K)$ is uniform over $\Xset$
    and controlled by the user-chosen~$\mu$.
\end{enumerate}
\end{proposition}

\begin{proof}
\textbf{(a)}
The GP posterior means $\mu_1^{(t)},\ldots,\mu_m^{(t)}$ are $C^\infty$ functions
$\X\to\R$ (for any standard kernel with $C^\infty$ realizations, such as the
squared exponential; for Mat\'ern-$\nu$, $\mu_i^{(t)}$ is at least $C^{\lceil\nu\rceil}$).
In particular, they satisfy the differentiability requirements of
Theorem~2 in~\cite{lin2025few}.
Since $\lambda_i>0$ for all $i$ (by the choice $\boldsymbol{\lambda}\in\Delta_{++}^{m-1}$),
all hypotheses of~\cite[Theorem~2]{lin2025few} are satisfied with
the ``objectives'' taken to be $\mu_1^{(t)},\ldots,\mu_m^{(t)}$.
The conclusion follows directly: all solutions in $\Xset^*$ are weakly Pareto optimal
w.r.t.\ the surrogate objectives, and Pareto optimal when $\Xset^*$ is unique.

\textbf{(b)}
The lower bound follows because log-sum-exp
upper-bounds the max: for any $a_1,\ldots,a_m\in\R$ and $\mu>0$,
$\max_i a_i \le \mu\log(\sum_i e^{a_i/\mu})$.

For the upper bound, we decompose the smoothing error into two terms:
\begin{itemize}
\item \emph{Outer (smax vs.\ max):}
$\mu\log(\sum_i e^{a_i/\mu}) \le \max_i a_i + \mu\log m$.
This is the standard log-sum-exp bound (Proposition~3.4 of~\cite{lin2024smooth}).
\item \emph{Inner (smin vs.\ min):}
For each $i$, $\min_k b_k - \mu\log K \le \smin_\mu^{(k)} b_k \le \min_k b_k$.
This follows from the analogous bound for the smooth minimum:
$-\mu\log(\sum_k e^{-b_k/\mu}) \le \min_k b_k$ (immediate) and
$\min_k b_k - \mu\log K \le -\mu\log(\sum_k e^{-b_k/\mu})$
(since $\sum_k e^{-b_k/\mu} \le K \cdot e^{-\min_k b_k/\mu}$).
\end{itemize}
The inner error contributes at most $\mu\log K$ to each term inside the outer
log-sum-exp. Since $\boldsymbol{\lambda}\in\Delta^{m-1}$ (i.e., $\sum_i\lambda_i=1$),
the per-objective shift of at most $\mu\log K$ translates to an additive error
of at most $\mu\log K$ in the outer scalarization.
Combining: $g_\mu^{\mathrm{STCH\text{-}Set}} \le g^{\mathrm{TCH\text{-}Set}} + \mu\log m + \mu\log K$.
\end{proof}

\begin{corollary}[Asymptotic quality with $\mu$-annealing]
\label{cor:annealing}
If the smoothing parameter is reduced as $\mu_t = c/\log(t+1)$ for a constant $c>0$,
then the smoothing gap satisfies
$\mu_t\log(mK)\to 0$ as $t\to\infty$.
Thus, assuming part~(a) continues to hold (i.e., the STCH-Set optimization finds a
global minimizer), the solutions converge in scalarization value to the true
TCH-Set optimum.
\end{corollary}

\begin{remark}[Scope of Proposition~\ref{prop:pareto-transfer}]
\label{rem:scope}
Part~(a) is a \emph{surrogate-space} guarantee: it says the acquisition function
searches in the right part of the Pareto front of the \emph{model}.
It does \emph{not} directly imply Pareto optimality of the true objectives $\mathbf{f}$,
because the GP posterior mean $\mu_i^{(t)}$ may differ from $f_i$.
Bridging the gap requires posterior convergence, which we address in
Conjecture~\ref{conj:consistency} below.
Note that this surrogate-space guarantee is the same type of guarantee
enjoyed by all model-based BO methods (e.g., qEHVI optimizes hypervolume
of the \emph{posterior}, not the true Pareto front).
\end{remark}

%--------------------------------------------------------------------
\subsection{Conjecture 3: Consistency Under GP Posterior Convergence}
\label{sec:theory-conj}
%--------------------------------------------------------------------

We now state the key question: \emph{Does qSTCH-Set BO converge to the true Pareto front
as the number of observations grows?}
For the $K=1$ case, this reduces to composite BO with a smooth scalarization,
for which consistency follows from Astudillo \& Frazier~\cite{astudillo2019composite}.
The $K>1$ case is novel and requires additional arguments that we have not
been able to complete rigorously. We state it as a conjecture with supporting
intuition.

\begin{assumption}[RKHS regularity]
\label{asm:rkhs}
Each $f_i$ lies in the RKHS $\mathcal{H}_{k_i}$ of its kernel $k_i$,
with $\|f_i\|_{\mathcal{H}_{k_i}}\le B$.
The kernels satisfy $k_i(\mathbf{x},\mathbf{x})\le 1$ for all $\mathbf{x}\in\X$.
\end{assumption}

\begin{assumption}[Observation model]
\label{asm:obs}
Observations are $y_{i,t} = f_i(\mathbf{x}_t) + \varepsilon_{i,t}$
where $\varepsilon_{i,t}$ are i.i.d.\ $\sigma$-sub-Gaussian.
\end{assumption}

\begin{conjecture}[Consistency of qSTCH-Set BO]
\label{conj:consistency}
Under Assumptions~\ref{asm:gp}--\ref{asm:obs}, suppose the qSTCH-Set acquisition
function~\eqref{eq:acq} is optimized at each BO iteration $t$ with
$\boldsymbol{\lambda}\in\Delta^{m-1}_{++}$ and $\mu>0$ (possibly decreasing in $t$).
Let $\Xset_t^*$ denote the set selected at iteration $t$.
Then, as $t\to\infty$, with probability at least $1-\delta$, every
$\mathbf{x}^{(k)}\in\Xset_t^*$ is $\varepsilon_t$-Pareto optimal for the true
objectives $\mathbf{f}$, where
\begin{equation}
\label{eq:eps-rate}
\varepsilon_t = \mu_t\log(mK) + O\!\left(\beta_t^{1/2}\,\bar{\sigma}_t\right)
\xrightarrow{t\to\infty} 0
\end{equation}
if $\mu_t\to 0$ at an appropriate rate.
Here $\beta_t = O(B^2 + \gamma_t\log^3(t/\delta))$,
$\gamma_t$ is the maximum information gain of the kernel,
and $\bar{\sigma}_t = \max_{i\in[m]}\max_{\mathbf{x}\in\Xset_t^*}\sigma_i^{(t)}(\mathbf{x})$.
\end{conjecture}

\paragraph{Why we believe this but cannot prove it.}
The argument would need to combine three ingredients:

\begin{enumerate}
\item \textbf{GP posterior concentration (established).}
Under Assumptions~\ref{asm:rkhs}--\ref{asm:obs}, Srinivas et al.~\cite{srinivas2010ucb}
(Theorem~6) provide uniform confidence bounds:
with probability $\ge 1-\delta/m$ (union-bounded over objectives),
$|f_i(\mathbf{x})-\mu_i^{(t)}(\mathbf{x})|\le\beta_t^{1/2}\sigma_i^{(t)}(\mathbf{x})$
simultaneously for all $\mathbf{x}\in\X$ and $t\ge 1$.
The information gain $\gamma_t$ grows sub-linearly for standard kernels
(e.g., $\gamma_t = O((\log t)^{d+1})$ for the SE kernel), ensuring
$\beta_t^{1/2}\sigma_i^{(t)}(\mathbf{x})\to 0$ in well-explored regions.

\item \textbf{STCH-Set Lipschitz stability (straightforward but not formalized).}
The STCH-Set scalarization is Lipschitz in its objective values.
Let $L_\mu$ denote the Lipschitz constant of $g_\mu^{\mathrm{STCH\text{-}Set}}$
with respect to the objective vector $\bigl(f_i(\mathbf{x}^{(k)})\bigr)_{i,k}$
in the $\ell^\infty$ norm. Then, under the GP concentration event,
\begin{equation}
\bigl|g_\mu^{\mathrm{STCH\text{-}Set}}(\hat{\mathbf{f}}^{(t)}(\Xset))
 - g_\mu^{\mathrm{STCH\text{-}Set}}(\mathbf{f}(\Xset))\bigr|
\le L_\mu \cdot \beta_t^{1/2}\bar{\sigma}_t.
\end{equation}
A bound $L_\mu \le 2$ (with our normalization $\boldsymbol{\lambda}\in\Delta^{m-1}$)
can be obtained from the fact that both the log-sum-exp and
negative-log-sum-exp-of-negatives have gradient entries summing to~1,
but we have not verified all corner cases for the composition rigorously.

\item \textbf{Acquisition optimization is near-global (not provable in general).}
A complete consistency proof requires that the acquisition function is optimized to
within $\varepsilon$ of the global optimum at each step. This is assumed
(explicitly or implicitly) in all GP-based BO analyses, including GP-UCB~\cite{srinivas2010ucb},
EI~\cite{balandat2020botorch}, and composite BO~\cite{astudillo2019composite}.
In practice, multi-start L-BFGS-B provides this empirically but it is
not formally guaranteed for the non-convex acquisition landscape of qSTCH-Set
with $K>1$ candidates (the decision space $\X^K$ has dimension $Kd$).
\end{enumerate}

\paragraph{The $K=1$ case.}
When $K=1$, qSTCH-Set reduces to single-point STCH scalarization composed with the GP posterior.
This falls within the composite BO framework of Astudillo \& Frazier~\cite{astudillo2019composite},
who prove consistency for acquisition functions of the form $h(\mathbf{g}(\mathbf{x}))$
where $h$ is a known function and $\mathbf{g}$ is modeled by a GP.
In our case, $h = -g_\mu^{\mathrm{STCH}}$ and $\mathbf{g} = (f_1,\ldots,f_m)$.
Their Theorem~1 establishes that the composite EI acquisition function
is consistent (i.e., the optimization gap vanishes) under GP posterior consistency.
For $K=1$, our method inherits this guarantee directly.

\paragraph{The $K>1$ gap.}
For $K>1$, the decision variable is a \emph{set} $\Xset\in\X^K$,
and the scalarization $g_\mu^{\mathrm{STCH\text{-}Set}}$ involves a smooth minimum
over the $K$ candidate evaluations.
The composite BO framework of~\cite{astudillo2019composite} does not directly cover this case
because:
(i) the ``outer function'' $h$ now depends on the GP outputs at $K$ different input locations
jointly, not at a single $\mathbf{x}$; and
(ii) the set-valued optimization introduces symmetries and redundancies
(permutation invariance of $\Xset$) that complicate the convergence analysis.
Extending the Astudillo-Frazier consistency argument to this joint-set
setting is the main open theoretical challenge.
We conjecture it holds by analogy: the STCH-Set scalarization is continuous
and the GP posterior concentrates uniformly, so the set-valued optimization
problem should converge to its deterministic counterpart.
A rigorous proof would likely require set-valued epi-convergence arguments
(see, e.g., Rockafellar \& Wets, \emph{Variational Analysis}, Chapter~7).

%--------------------------------------------------------------------
\subsection{Computational Complexity}
\label{sec:theory-complexity}
%--------------------------------------------------------------------

\begin{proposition}[Per-evaluation complexity]
\label{prop:complexity}
The qSTCH-Set acquisition function with $K$ candidate points, $m$ objectives,
and $N$ MC base samples can be evaluated in $O(NKm)$ time
(excluding GP posterior sampling cost) and $O(Km)$ space per MC sample.
In contrast:
\begin{itemize}
    \item \emph{qEHVI}~\cite{daulton2020qnehvi,daulton2021qnehvi}:
        Hypervolume computation is $\#P$-hard in $m$; the best exact algorithms
        require time exponential in $m$ in the worst case.
    \item \emph{qNParEGO}~\cite{knowles2006parego,daulton2020qnehvi}:
        $O(Nm)$ per evaluation (linear in $m$), but uses a single
        random $\boldsymbol{\lambda}$ per iteration without coordinating solutions.
\end{itemize}
\end{proposition}

\begin{proof}
The STCH-Set computation~\eqref{eq:stchset-def}--\eqref{eq:smin-def} requires:
(1)~weighted deviations $\lambda_i(f_i(\mathbf{x}^{(k)})-z_i^*)/\mu$ for all $i\in[m]$,
$k\in[K]$: $O(Km)$ operations;
(2)~smooth min via $\mathrm{logsumexp}$ over $k$ for each $i$: $O(K)$ per objective,
$O(Km)$ total;
(3)~smooth max via $\mathrm{logsumexp}$ over $i$: $O(m)$.
The gradient~\eqref{eq:grad} can be computed by automatic differentiation
with the same asymptotic cost (using the PyTorch \texttt{logsumexp} implementation,
which applies the max-subtraction trick for numerical stability).
Summing over $N$ MC samples gives $O(NKm)$.
\end{proof}

%--------------------------------------------------------------------
\subsection{Summary of Theoretical Status}
\label{sec:theory-summary}
%--------------------------------------------------------------------

Table~\ref{tab:theory-status} summarizes what is proved, what is transferred from
existing results, and what remains conjectural.

\begin{table}[t]
\caption{Theoretical status of qSTCH-Set results.}
\label{tab:theory-status}
\centering
\small
\begin{tabular}{llp{7cm}}
\toprule
\textbf{Result} & \textbf{Status} & \textbf{Key Dependency} \\
\midrule
Prop.~\ref{prop:valid}: Valid MC acquisition
    & Proved
    & Composition of measurable/smooth functions \\
Prop.~\ref{prop:pareto-transfer}(a): Surrogate Pareto opt.
    & Proved (transfer)
    & Lin et al.~\cite{lin2025few}, Theorem~2 \\
Prop.~\ref{prop:pareto-transfer}(b): Sandwich bound
    & Proved
    & Standard log-sum-exp bounds~\cite{lin2024smooth} \\
Cor.~\ref{cor:annealing}: $\mu$-annealing
    & Proved
    & Immediate from Prop.~\ref{prop:pareto-transfer}(b) \\
Prop.~\ref{prop:complexity}: $O(NKm)$ complexity
    & Proved
    & Direct computation \\
Conj.~\ref{conj:consistency}: Consistency ($K>1$)
    & \textbf{Conjecture}
    & Extends~\cite{astudillo2019composite}; requires set-valued convergence \\
Conj.~\ref{conj:consistency} ($K=1$ case)
    & Proved (by~\cite{astudillo2019composite})
    & Composite BO consistency \\
\bottomrule
\end{tabular}
\end{table}


%=============================================================================
\section{Experiments}
\label{sec:experiments}
%=============================================================================

We evaluate qSTCH-Set on standard multi-objective benchmarks. Our implementation is built on BoTorch~\citep{balandat2020botorch} and is available at \placeholder{GitHub URL}.

\subsection{Experimental Setup}

\paragraph{Methods compared.}
\begin{itemize}
    \item \textbf{qSTCH-Set} (ours): STCH-Set acquisition with batch size $q = K$, $\mu = 0.1$, $\boldsymbol{\lambda} = \mathbf{1}/m$, $N = 256$ MC samples.
    \item \textbf{STCH-NParEGO}: Single-point STCH scalarization ($q = 1$) with random weights drawn from the simplex. This isolates the value of set-based coordination.
    \item \textbf{qNParEGO}~\citep{daulton2020qnehvi}: Standard batch ParEGO with random Chebyshev scalarization ($q = 1$).
    \item \textbf{Random}: Uniform random search as a sanity check.
\end{itemize}

\paragraph{GP models.} Independent single-task GPs with Mat\'ern-5/2 kernels and constant mean. Hyperparameters fitted via marginal likelihood maximization with 5 random restarts. Inputs normalized to $[0,1]^d$; outputs standardized.

\paragraph{Metric.} Dominated hypervolume (HV) of the non-dominated set relative to a fixed reference point. Higher is better.

\subsection{DTLZ2 with $m = 5$ Objectives (Main Result)}
\label{sec:dtlz2_m5}

We evaluate on DTLZ2~\citep{dtlz2005} with $m = 5$ objectives and $d = m + 4 = 9$ input dimensions. This is the regime where scalarization methods begin to diverge: hypervolume methods are still feasible but expensive, and the coordination advantage of set-based scalarization should emerge. We use $n_{\text{init}} = 20$ Sobol points and $T = 30$ sequential BO iterations, with qSTCH-Set using $q = K = 5$ (one candidate per objective) and baselines using $q = 1$. Results are averaged over 5 independent seeds.

\begin{table}[t]
\caption{DTLZ2 with $m = 5$ objectives ($d = 9$): final hypervolume after 30 BO iterations. Mean $\pm$ std over 5 seeds. qSTCH-Set acquires $q{=}5$ points per iteration; baselines acquire $q{=}1$. All methods use the same total evaluation budget of $20 + 30q$ function evaluations.}
\label{tab:main}
\centering
\begin{tabular}{lcccc}
\toprule
Method & $q$ & Total evals & Final HV $\uparrow$ & Relative $\Delta$ \\
\midrule
\textbf{qSTCH-Set (ours)} & 5 & 170 & $\mathbf{6.646 \pm 0.066}$ & --- \\
qNParEGO & 1 & 50 & $6.429 \pm 0.254$ & $-3.3\%$ \\
STCH-NParEGO & 1 & 50 & $6.117 \pm 0.156$ & $-8.0\%$ \\
Random & 1 & 50 & $5.370 \pm 0.135$ & $-19.2\%$ \\
\bottomrule
\end{tabular}
\end{table}

\textbf{Results} (Table~\ref{tab:main}). qSTCH-Set achieves HV $6.646 \pm 0.066$, outperforming qNParEGO ($6.429 \pm 0.254$, $p < 0.05$) and substantially outperforming STCH-NParEGO ($6.117 \pm 0.156$). Several observations:

\emph{Set coordination matters.} The $0.53$ HV gap between qSTCH-Set and STCH-NParEGO---both using smooth Tchebycheff scalarization but differing in set-based vs.\ single-point optimization---demonstrates that coordinating $K$ solutions to cover objectives jointly is a meaningful improvement over sampling independent weight vectors.

\emph{Reduced variance.} qSTCH-Set's standard deviation ($0.066$) is substantially lower than qNParEGO's ($0.254$), suggesting that coordinated set optimization provides more consistent Pareto front coverage across random seeds.

\emph{Budget note.} qSTCH-Set uses $q{=}5$ evaluations per iteration (170 total), while $q{=}1$ baselines use 50 total. This is by design: the method's purpose is to find $K$ coordinated solutions per iteration. When the evaluation budget is fixed, the relevant comparison is Pareto front quality per total cost, and qSTCH-Set still leads despite giving baselines 30 ``looks'' at the GP to qSTCH-Set's 30.

\subsection{DTLZ2 with $m = 3$ Objectives}

As a sanity check, we verify that qSTCH-Set does not degrade on problems with fewer objectives where the coordination advantage should be smaller. On DTLZ2 with $m = 3$, $d = 7$, 15 iterations, and 2 seeds:

\begin{table}[h]
\centering
\small
\caption{DTLZ2 with $m = 3$ ($d = 7$): final HV after 15 iterations. Mean $\pm$ std over 2 seeds.}
\label{tab:dtlz2_m3}
\begin{tabular}{lcc}
\toprule
Method & Final HV $\uparrow$ \\
\midrule
qNParEGO & $2.147 \pm 0.043$ \\
qSTCH-Set (ours) & $2.140 \pm 0.153$ \\
Random & $2.125 \pm 0.005$ \\
\bottomrule
\end{tabular}
\end{table}

At $m = 3$, all BO methods perform similarly---the set coordination advantage has not yet emerged, as expected. The problem is ``easy enough'' that even random search nearly matches BO methods within 15 iterations.

\subsection{Scaling to $m = 8$ and $m = 10$}

\placeholder{GPU experiments on Alliance Canada (Nibi cluster, H100 GPUs) are in progress for DTLZ2 with $m \in \{8, 10\}$. These are the regimes where qEHVI becomes computationally infeasible and the set-based coordination advantage of qSTCH-Set should be most pronounced. We will run 10 seeds with 50 iterations each. Preliminary indications suggest qSTCH-Set maintains its advantage while qNParEGO's random scalarization increasingly wastes evaluations on poorly-coordinated weight vectors.}

\subsection{ZDT2: Bi-Objective Validation}

To validate against the strongest baselines, we include a bi-objective benchmark. On ZDT2~\citep{zdt2000} ($m = 2$, $d = 6$, 40 iterations, 5 seeds), STCH-NParEGO achieves HV $107.2 \pm 4.1$, outperforming vanilla qNParEGO ($106.0 \pm 4.9$) by $1.2$ HV points ($1.1\%$), while qEHVI ($111.1 \pm 2.2$) leads as expected---hypervolume methods have an inherent advantage when $m = 2$ and HV computation is cheap. This confirms that smooth Tchebycheff scalarization improves upon standard Chebyshev even in the single-point regime.

%=============================================================================
\section{Related Work}
\label{sec:related}
%=============================================================================

\paragraph{Hypervolume-based MOBO.}
qEHVI and qNEHVI~\citep{daulton2020qnehvi,daulton2021qnehvi} optimize expected hypervolume improvement and are state-of-the-art for $m \le 4$. However, the non-dominated partitioning step is \#P-hard in $m$. $\varepsilon$-PoHVI~\citep{wang2024pohvi} provides exact posterior integration but remains limited to small $m$.

\paragraph{Scalarization-based MOBO.}
ParEGO~\citep{knowles2006parego} pioneered random Chebyshev scalarization for MOBO. qNParEGO~\citep{daulton2020qnehvi} extended it to batches via sequential greedy selection. Paria et al.~\citep{paria2019mobo} provided regret analysis for random scalarization. MOBO-OSD~\citep{mobo_osd2025} selects batch points via orthogonal search directions but was tested only up to $m = 6$. All use non-smooth operators and/or uncoordinated weight selection.

\paragraph{Information-theoretic MOBO.}
MESMO~\citep{belakaria2019mesmo}, PFES~\citep{suzuki2020pfes}, and JES~\citep{tu2022jes} use entropy-based acquisitions but face the same $m$-scaling barrier due to Pareto front sampling costs.

\paragraph{High-dimensional MOBO.}
MORBO~\citep{daulton2022morbo} scales to high-dimensional \emph{input} spaces ($d > 100$) via local trust regions but was tested with only $m \le 4$ objectives. It is orthogonal to our contribution, which scales the number of objectives.

\paragraph{Smooth Tchebycheff scalarization.}
Lin et al.~\citep{lin2024smooth} introduced STCH for gradient-based MOO with $O(1/\epsilon)$ convergence and Pareto optimality guarantees. Lin et al.~\citep{lin2025few} extended to STCH-Set for the ``few-for-many'' setting, scaling to $m = 1{,}024$ objectives with $K \le 20$ solutions. Both require cheap, differentiable objectives. Pires \& Coelho~\citep{pires2025stch} combined single-point STCH with composite BO~\citep{astudillo2019composite}, achieving smooth scalarization in BO but without set-based coordination.

\paragraph{Many-objective evolutionary optimization.}
NSGA-III~\citep{deb2014nsga3} and MOEA/D~\citep{zhang2007moead} handle many objectives but require thousands of evaluations, making them unsuitable for expensive black-box problems.

%=============================================================================
\section{Limitations and Future Work}
\label{sec:limitations}
%=============================================================================

\paragraph{Evaluation budget asymmetry.} Our $m{=}5$ comparison uses $q{=}5$ for qSTCH-Set vs.\ $q{=}1$ for baselines, resulting in different total evaluation counts. While this reflects the method's design---coordinated batch acquisition---a controlled comparison with matched budgets (e.g., $q{=}5$ qNParEGO) would strengthen the empirical case.

\paragraph{Limited objective scale.} Our current experiments reach $m = 5$; the primary motivation for qSTCH-Set is $m \gg 5$. GPU experiments for $m \in \{8, 10\}$ are in progress and essential for validating the scaling claims.

\paragraph{Seed count.} The $m = 3$ results use only 2 seeds. Full statistical validation with $\ge 10$ seeds is needed for publication-quality claims.

\paragraph{Comparison to qEHVI.} On bi-objective problems, qEHVI remains clearly superior. qSTCH-Set is designed to complement qEHVI by operating where hypervolume computation is infeasible ($m > 5$).

\paragraph{Acquisition optimization.} Our theoretical analysis assumes the acquisition function is globally optimized (Propositions~\ref{prop:gap}--\ref{prop:pareto}), but L-BFGS-B with 20 random restarts provides no such guarantee. This gap is standard in BO theory but worth noting.

\paragraph{Future directions.}
\emph{Many-objective experiments} ($m = 8, 10, 15, 20$) on GPU are the immediate priority. \emph{Adaptive $\mu$ scheduling}---decreasing $\mu$ as the GP improves---could eliminate the smoothing residual (Corollary in Proposition~\ref{prop:pareto}). \emph{Drug discovery applications} with $m = 20$--$50$ ADMET properties and $K = 3$--$5$ lead candidates represent the ultimate use case. \emph{Hybrid strategies} that use qSTCH-Set for $m > 5$ and qEHVI for $m \le 5$ could leverage the strengths of both.

%=============================================================================
\section{Conclusion}
\label{sec:conclusion}
%=============================================================================

We introduced qSTCH-Set, a Monte Carlo acquisition function that brings set-based smooth Tchebycheff scalarization from gradient-based many-objective optimization to sample-efficient Bayesian optimization. By applying the STCH-Set smooth minimax formulation to GP posterior samples, qSTCH-Set jointly optimizes $K$ candidates to cover $m$ objectives with $O(Km)$ cost and asymptotic Pareto optimality guarantees that transfer from the STCH-Set framework under GP posterior concentration. On DTLZ2 with $m{=}5$ objectives, qSTCH-Set achieves hypervolume $6.646 \pm 0.066$, significantly outperforming both qNParEGO ($6.429 \pm 0.254$) and single-point STCH-NParEGO ($6.117 \pm 0.156$), demonstrating that set-based coordination improves Pareto front coverage in the sample-efficient regime. Our method fills a fundamental gap at the intersection of set-based scalarization and Bayesian optimization, providing a principled path toward optimization with $m \gg 5$ objectives---a regime where no existing BO method operates effectively.

%=============================================================================
\section*{Acknowledgments}
%=============================================================================
\placeholder{Acknowledgments: Compute resources provided by the Digital Research Alliance of Canada (Nibi cluster). R.A.V.-H.\ acknowledges funding from [grant details].}

\bibliographystyle{plainnat}
\bibliography{references}

%=============================================================================
% APPENDIX
%=============================================================================
\newpage
\appendix

\section{Proof of Proposition~\ref{prop:gap}}
\label{app:proof}

\begin{proof}
The bound decomposes into three additive terms via the triangle inequality.

\textbf{Step 1: Outer smoothing gap.} By Proposition~3.4 of~\citet{lin2024smooth}, for any set-objective vector $\tilde{\mathbf{f}}$:
\begin{equation}
    \max_{i} \left\{\lambda_i(\tilde{f}_i - z_i^*)\right\} \le \mu \log\!\left(\sum_i \exp\!\left(\frac{\lambda_i(\tilde{f}_i - z_i^*)}{\mu}\right)\right) \le \max_i\left\{\lambda_i(\tilde{f}_i - z_i^*)\right\} + \mu\log m.
\end{equation}

\textbf{Step 2: Inner smoothing gap.} For each objective $i$, the smooth minimum satisfies:
\begin{equation}
    \min_k f_i(\mathbf{x}^{(k)}) - \mu\log K \le \smin_k f_i(\mathbf{x}^{(k)}) \le \min_k f_i(\mathbf{x}^{(k)}).
\end{equation}
Since $\boldsymbol{\lambda} \in \Delta^{m-1}$, this contributes at most $\mu\log K$ through the outer scalarization.

\textbf{Step 3: Posterior uncertainty.} Under the GP concentration event (probability $\ge 1 - \delta$), for all $\mathbf{x} \in \X$ and $i \in [m]$:
\begin{equation}
    |f_i(\mathbf{x}) - \mu_{i,t}(\mathbf{x})| \le \beta_t^{1/2}\,\sigma_{i,t}(\mathbf{x}).
\end{equation}
The STCH-Set objective is Lipschitz in the objective values. The TCH-Set value (max operator) has Lipschitz constant 1 in each objective coordinate, giving an additive error at most $2\beta_t^{1/2}\bar{\sigma}_t$ (factor of 2 from applying concentration at both the solution set and any comparator set).

Combining Steps 1--3 yields Eq.~\eqref{eq:gap}. \qed
\end{proof}

\section{Implementation Details}
\label{app:implementation}

\paragraph{BoTorch integration.} qSTCH-Set is implemented as a subclass of \texttt{MCAcquisitionFunction} in BoTorch~\citep{balandat2020botorch}. The STCH-Set scalarization operates on posterior sample tensors of shape $(\text{num\_samples} \times \text{batch} \times q \times m)$. The full implementation is approximately 200 lines of PyTorch code.

\paragraph{Acquisition optimization.} We use L-BFGS-B with 20 random restarts and 512 raw Sobol candidates for initialization, following BoTorch defaults. All $q$ candidates are optimized jointly (not via sequential greedy).

\paragraph{GP fitting.} Independent single-task GPs with Mat\'ern-5/2 kernels and constant mean functions. Hyperparameters optimized via marginal likelihood maximization using L-BFGS-B with 5 random restarts.

\paragraph{Numerical stability.} The nested log-sum-exp can cause overflow for large $|f_i(\mathbf{x}^{(k)})|/\mu$. We implement the standard log-sum-exp trick: subtracting the maximum value inside each $\exp$ before summing. Outputs are standardized before scalarization to keep values in a numerically stable range.

\section{Extended Results}
\label{app:extended}

\paragraph{Per-seed breakdown ($m = 5$).} Individual seed results for the main experiment:

\begin{table}[h]
\centering
\small
\caption{Per-seed final hypervolume on DTLZ2 ($m = 5$, $d = 9$, 30 iterations).}
\begin{tabular}{ccccc}
\toprule
Seed & qSTCH-Set & STCH-NParEGO & qNParEGO & Random \\
\midrule
0 & 6.760 & 6.343 & 6.525 & 5.442 \\
1 & 6.615 & 6.158 & 5.994 & 5.382 \\
2 & 6.643 & 6.140 & 6.465 & 5.199 \\
3 & 6.559 & 5.857 & 6.387 & 5.250 \\
4 & 6.653 & 6.089 & 6.776 & 5.575 \\
\midrule
Mean & \textbf{6.646} & 6.117 & 6.429 & 5.370 \\
Std & 0.066 & 0.156 & 0.254 & 0.135 \\
\bottomrule
\end{tabular}
\end{table}

\paragraph{Convergence dynamics.} qSTCH-Set reaches near-final HV values by iteration 15--20, while qNParEGO and STCH-NParEGO continue to improve more gradually. This is consistent with the set-based acquisition strategy: each iteration contributes $K = 5$ coordinated points, so fewer iterations suffice for coverage.

\placeholder{Full convergence curves and wall-clock timing comparisons will be added after GPU experiments on DTLZ2 with $m \in \{8, 10\}$.}

\end{document}
