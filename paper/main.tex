\documentclass{article}

% NeurIPS 2026 style
\usepackage[final]{neurips_2026}

\usepackage[utf8]{inputenc}
\usepackage[T1]{fontenc}
\usepackage{hyperref}
\usepackage{url}
\usepackage{booktabs}
\usepackage{amsfonts}
\usepackage{amsmath}
\usepackage{amssymb}
\usepackage{nicefrac}
\usepackage{microtype}
\usepackage{xcolor}
\usepackage{algorithm}
\usepackage{algorithmic}
\usepackage{graphicx}
\usepackage{multirow}
\usepackage{subcaption}
\usepackage{amsthm}
\usepackage{thm-restate}

% Theorem environments
\newtheorem{theorem}{Theorem}[section]
\newtheorem{proposition}[theorem]{Proposition}
\newtheorem{corollary}[theorem]{Corollary}
\newtheorem{lemma}[theorem]{Lemma}
\newtheorem{definition}[theorem]{Definition}
\theoremstyle{remark}
\newtheorem{remark}[theorem]{Remark}
\newtheorem{conjecture}[theorem]{Conjecture}
\newtheorem{assumption}[theorem]{Assumption}

% Custom commands
\newcommand{\placeholder}[1]{{\color{red}\textbf{[#1]}}}
\newcommand{\R}{\mathbb{R}}
\newcommand{\E}{\mathbb{E}}
\newcommand{\X}{\mathcal{X}}
\newcommand{\D}{\mathcal{D}}
\newcommand{\GP}{\mathcal{GP}}
\newcommand{\Xset}{X_K}
\newcommand{\smax}{\mathrm{smax}}
\newcommand{\smin}{\mathrm{smin}}
\DeclareMathOperator*{\argmin}{arg\,min}
\DeclareMathOperator*{\argmax}{arg\,max}

\title{Set-Based Smooth Tchebycheff Scalarization\\for Many-Objective Bayesian Optimization}

\author{
  Ilkham Yabbarov \\
  Department of Chemistry \\
  McMaster University \\
  \texttt{yabbari@mcmaster.ca} \\
  \And
  Rodrigo A. Vargas-Hern\'andez \\
  Department of Chemistry \\
  McMaster University \\
  \texttt{vargashr@mcmaster.ca} \\
}

\begin{document}

\maketitle

%=============================================================================
\begin{abstract}
%=============================================================================
Multi-objective Bayesian optimization (MOBO) scales poorly beyond five objectives: hypervolume-based acquisition functions incur exponential cost in the number of objectives~$m$, while scalarization methods produce uncoordinated batches via random weight vectors.
Set-based smooth Tchebycheff (STCH-Set) scalarization addresses this in gradient-based optimization, and single-point STCH has been adapted to BO, but no method combines set-based coordination with sample-efficient surrogate-driven search.
We introduce qSTCH-Set, a Monte Carlo acquisition function that jointly optimizes $K$ candidates via STCH-Set scalarization over Gaussian process posterior samples at $O(Km)$ cost, with the design rule $K{=}m$ allocating one candidate per objective.
On DTLZ2 with $m{=}5$ objectives, qSTCH-Set achieves hypervolume $6.646 \pm 0.066$, outperforming both qNParEGO ($6.429 \pm 0.254$) and single-point STCH ($6.117 \pm 0.156$); at $m{=}8$ the advantage widens to 11\%, empirically validating the $K{=}m$ rule.
qSTCH-Set fills a fundamental gap in many-objective Bayesian optimization, enabling coordinated Pareto front coverage in regimes where hypervolume-based methods are computationally intractable.
\end{abstract}

%=============================================================================
\section{Introduction}
\label{sec:intro}
%=============================================================================

Many real-world optimization problems require balancing a large number of conflicting objectives simultaneously.
In drug discovery, a candidate molecule must satisfy constraints on potency, selectivity, metabolic stability, permeability, and toxicity---routinely $m{=}20$--$50$ ADMET endpoints~\citep{knowles2006parego}.
In materials design, one seeks alloys balancing strength, weight, cost, and corrosion resistance.
When function evaluations are expensive---requiring physical experiments, molecular simulations, or costly assays---multi-objective Bayesian optimization (MOBO) provides a principled, sample-efficient framework that fits Gaussian process (GP) surrogates to observed data and uses acquisition functions to guide evaluation~\citep{balandat2020botorch}.

However, existing MOBO methods face a fundamental scaling barrier in the number of objectives~$m$.
Hypervolume-based acquisition functions---the gold standard for $m \le 4$---rely on non-dominated partitioning of the objective space, which is \#P-hard in~$m$~\citep{daulton2020qnehvi,daulton2021qnehvi,wang2024pohvi}.
Scalarization methods such as ParEGO and qNParEGO~\citep{knowles2006parego,daulton2020qnehvi} decompose the problem via random Chebyshev weight vectors, scaling gracefully in~$m$ but producing uncoordinated solutions: each batch element optimizes an independently sampled weight with no mechanism to ensure collective Pareto front coverage.
Information-theoretic approaches (MESMO, PFES, JES)~\citep{belakaria2019mesmo,suzuki2020pfes,tu2022jes} face similar degradation as Pareto front sampling costs grow with~$m$.

Two recent lines of work have made partial progress toward scalable many-objective scalarization.
Lin et al.~\citep{lin2024smooth} introduced smooth Tchebycheff (STCH) scalarization, a log-sum-exp relaxation of the non-smooth Chebyshev scalarization that is everywhere differentiable and preserves Pareto optimality guarantees.
They extended this to STCH-Set~\citep{lin2025few}, where $K$ solutions are jointly optimized via a smooth minimax formulation to collectively cover~$m$ objectives at $O(Km)$ cost, demonstrating scaling to $m{=}1{,}024$ objectives in gradient-based optimization with cheap, differentiable objectives.
Independently, Pires \& Coelho~\citep{pires2025stch} adapted single-point STCH to composite Bayesian optimization~\citep{astudillo2019composite}, achieving smooth scalarization within the BO loop but without set-based coordination.
This leaves a clear gap in the landscape (Table~\ref{tab:gap}): no method applies set-based smooth Tchebycheff scalarization to sample-efficient Bayesian optimization of expensive black-box functions.

\begin{table}[t]
\caption{Positioning of qSTCH-Set in the scalarization--optimization landscape. Set-based STCH has been applied to gradient-based optimization~\citep{lin2025few} and single-point STCH to Bayesian optimization~\citep{pires2025stch}. We fill the remaining cell.}
\label{tab:gap}
\centering
\small
\begin{tabular}{lcc}
\toprule
 & Single Solution & Set of $K$ Solutions \\
\midrule
Gradient-based (cheap $f$) & STCH~\citep{lin2024smooth} & STCH-Set~\citep{lin2025few} \\
Bayesian optimization (expensive $f$) & Pires \& Coelho~\citep{pires2025stch} & \textbf{qSTCH-Set (ours)} \\
\bottomrule
\end{tabular}
\end{table}

We fill this gap with \textbf{qSTCH-Set}, a Monte Carlo acquisition function that evaluates a candidate set by applying the STCH-Set minimax scalarization to GP posterior samples, enabling coordinated multi-solution acquisition for many-objective BO.
Our key design rule sets $K{=}m$: one candidate per objective, so that the smooth minimum operator assigns each candidate to a distinct objective direction, expanding the Pareto front along all axes simultaneously.

\paragraph{Contributions.}
\begin{enumerate}
    \item We propose \textbf{qSTCH-Set}, the first Monte Carlo acquisition function that applies set-based smooth Tchebycheff scalarization to GP posterior samples for many-objective Bayesian optimization (\S\ref{sec:method}).
    \item We prove that STCH-Set Pareto optimality guarantees transfer asymptotically under GP posterior concentration, with an explicit approximation gap of $\mu\log(mK) + O(\beta_t^{1/2}\bar{\sigma}_t)$ (\S\ref{sec:theory}).
    \item On DTLZ2, qSTCH-Set outperforms qNParEGO and single-point STCH at both $m{=}5$ and $m{=}8$ objectives, with the advantage widening at higher~$m$ (\S\ref{sec:experiments}).
    \item We release an open-source BoTorch implementation as a drop-in \texttt{MCAcquisitionFunction} (\S\ref{app:implementation}).
\end{enumerate}

%=============================================================================
\section{Background}
\label{sec:background}
%=============================================================================

\subsection{Multi-Objective Optimization}

Consider the multi-objective optimization problem:
\begin{equation}
    \min_{\mathbf{x} \in \X} \mathbf{f}(\mathbf{x}) = (f_1(\mathbf{x}), \ldots, f_m(\mathbf{x})),
\end{equation}
where $\X \subseteq \R^d$ is compact and each $f_i: \X \to \R$ is a black-box objective. A point $\mathbf{x}^*$ is \emph{weakly Pareto optimal} if no $\mathbf{x} \in \X$ satisfies $f_i(\mathbf{x}) < f_i(\mathbf{x}^*)$ for all $i$, and \emph{Pareto optimal} if no $\mathbf{x}$ satisfies $f_i(\mathbf{x}) \le f_i(\mathbf{x}^*)$ for all $i$ with strict inequality for at least one. Their image under $\mathbf{f}$ is the \emph{Pareto front}.

\subsection{Multi-Objective Bayesian Optimization}

When each $f_i$ is expensive, MOBO fits independent GP surrogates $\hat{f}_i \sim \GP(\mu_i, k_i)$ to observed data $\D_t = \{(\mathbf{x}_j, \mathbf{y}_j)\}_{j=1}^{t}$~\citep{rasmussen2006gp}. An acquisition function $\alpha(\mathbf{x})$ decides where to evaluate next. Leading approaches include expected hypervolume improvement (qEHVI)~\citep{daulton2020qnehvi,daulton2021qnehvi} and scalarized expected improvement (qNParEGO)~\citep{knowles2006parego,daulton2020qnehvi}.

\subsection{Tchebycheff Scalarization}

The Tchebycheff scalarization converts a multi-objective problem into a scalar one:
\begin{equation}
\label{eq:tch}
    g^{\text{TCH}}(\mathbf{x} \mid \boldsymbol{\lambda}) = \max_{1 \le i \le m} \left\{ \lambda_i \left( f_i(\mathbf{x}) - z_i^* \right) \right\},
\end{equation}
where $\boldsymbol{\lambda} \in \Delta^{m-1}$ is a weight vector and $\mathbf{z}^*$ is the ideal point. Classical results~\citep{choo1983tchebycheff} show that every Pareto-optimal solution can be found by some $\boldsymbol{\lambda}$. However, the $\max$ operator is non-smooth.

\subsection{Smooth Tchebycheff (STCH) and STCH-Set}

Lin et al.~\citep{lin2024smooth} replace the $\max$ with a log-sum-exp approximation:
\begin{equation}
\label{eq:stch}
    g^{\text{STCH}}_\mu(\mathbf{x} \mid \boldsymbol{\lambda}) = \mu \log \left( \sum_{i=1}^{m} \exp\!\left( \frac{\lambda_i(f_i(\mathbf{x}) - z_i^*)}{\mu} \right) \right),
\end{equation}
where $\mu > 0$ controls smoothness. This satisfies $g^{\text{TCH}} \le g^{\text{STCH}}_\mu \le g^{\text{TCH}} + \mu \log m$, and its stationary points are weakly Pareto optimal~\citep{lin2024smooth}.

For the \emph{set optimization} problem---finding $K$ solutions $\Xset = \{\mathbf{x}^{(1)}, \ldots, \mathbf{x}^{(K)}\}$ to collectively cover $m$ objectives---Lin et al.~\citep{lin2025few} propose:
\begin{equation}
\label{eq:stchset}
    g^{\text{STCH-Set}}_\mu(\Xset \mid \boldsymbol{\lambda}) = \mu \log \left( \sum_{i=1}^{m} \exp\!\left( \frac{\lambda_i \left( \smin_{k} f_i(\mathbf{x}^{(k)}) - z_i^* \right)}{\mu} \right) \right),
\end{equation}
where the smooth minimum over candidates is:
\begin{equation}
\label{eq:smin}
    \smin_{k=1}^{K} f_i(\mathbf{x}^{(k)}) = -\mu \log \left( \sum_{k=1}^{K} \exp\!\left( -\frac{f_i(\mathbf{x}^{(k)})}{\mu} \right) \right).
\end{equation}
The outer log-sum-exp approximates the worst-case objective (smooth max), while the inner approximates the best candidate per objective (smooth min). This enables $K \ll m$ solutions to coordinate and cover all objectives.

\begin{theorem}[Theorem~2 of \citet{lin2025few}]
\label{thm:lin}
All solutions in the optimal set $\Xset^*$ for the STCH-Set problem~\eqref{eq:stchset} are weakly Pareto optimal. They are Pareto optimal if $\lambda_i > 0$ for all $i$, or if $\Xset^*$ is unique.
\end{theorem}

%=============================================================================
\section{Method: qSTCH-Set}
\label{sec:method}
%=============================================================================

We introduce \textbf{qSTCH-Set}, a Monte Carlo acquisition function that adapts the smooth Tchebycheff set scalarization~\citep{lin2025few} to Bayesian optimization. Our key insight is a principled design rule for the batch size: we set $q=K=m$, allocating one candidate per objective to ensure the batch can collectively span the vertices of the Pareto front.

\subsection{Monte Carlo STCH-Set Acquisition Function}

Given $m$ independent GP posteriors $\hat{f}_1, \ldots, \hat{f}_m$, the qSTCH-Set acquisition function evaluates a candidate set $\Xset = \{\mathbf{x}^{(1)}, \ldots, \mathbf{x}^{(q)}\}$ via the expected smooth scalarization utility:
\begin{equation}
\label{eq:qstchset}
    \alpha^{\text{qSTCH-Set}}(\Xset) = \E_{\mathbf{f} \sim \GP}\!\left[ -g^{\text{STCH-Set}}_\mu\!\left(\mathbf{f}(\Xset) \mid \boldsymbol{\lambda}\right) \right],
\end{equation}
where $g^{\text{STCH-Set}}_\mu$ is the smooth minimax scalarization defined in Eq.~\eqref{eq:stchset}. We approximate the expectation using $N$ quasi-Monte Carlo samples via the reparameterization trick:
\begin{equation}
\label{eq:mc}
    \alpha^{\text{qSTCH-Set}}(\Xset) \approx \frac{1}{N} \sum_{n=1}^{N} \left[ -g^{\text{STCH-Set}}_\mu\!\left(\hat{\mathbf{f}}^{(n)}(\Xset) \mid \boldsymbol{\lambda}\right) \right].
\end{equation}
Here, $\hat{\mathbf{f}}^{(n)}(\Xset)$ denotes the $n$-th posterior sample of the vector-valued function at the set $\Xset$.

\paragraph{The $K=m$ Design Rule.}
Standard batch BO methods like qNParEGO select $q$ points sequentially or jointly, often with $q \ll m$ or $q$ fixed arbitrarily. We propose a rigorous coupling:
\begin{center}
    \emph{Set the batch size $K$ equal to the number of objectives $m$.}
\end{center}
\textbf{Intuition:} The Tchebycheff scalarization with weight vectors near the simplex vertices isolates individual objectives. To approximate the ideal point, the batch must contain at least one candidate specializing in each objective $f_i$. If $K < m$, the set cannot simultaneously cover all $m$ extremal directions of the Pareto front, leading to ``blind spots'' in the acquisition. By setting $K=m$ and using a uniform weight $\boldsymbol{\lambda} = \mathbf{1}/m$, the inner smooth minimum operator $\smin_{k} f_i(\mathbf{x}^{(k)})$ effectively assigns one $\mathbf{x}^{(k)}$ to each $f_i$, enabling the set to descend all objectives in parallel.

\paragraph{Complexity.}
The evaluation of $\alpha^{\text{qSTCH-Set}}$ scales as $O(N \cdot K \cdot m)$. With $K=m$, this becomes $O(N m^2)$. While quadratic in $m$, this remains computationally efficient for $m \approx 50$, unlike hypervolume methods which scale as $O(N 2^m)$ or worse. The gradient computation via auto-differentiation shares the same complexity.

\subsection{Practical Considerations}
\label{sec:practical}

\paragraph{Choice of $K$.} While we recommend $K=m$ for balanced exploration, if evaluation budget is strictly limited, one can set $K < m$. In this case, qSTCH-Set will prioritize the subset of objectives that yield the largest marginal utility, but convergence to the full Pareto front may slow.

\paragraph{Smoothing parameter $\mu$.} We use $\mu = 0.01$ to $0.1$. Smaller $\mu$ yields a tighter approximation to the Tchebycheff scalarization but stiffens the gradients. A schedule $\mu_t \to 0$ is theoretically grounded but $\mu$ fixed at $0.1$ works well in practice.

\paragraph{Weight Sampling vs. Fixed Weights.} Unlike ParEGO/qNParEGO which sample random $\boldsymbol{\lambda}$ at each step, qSTCH-Set uses a \emph{fixed} uniform $\boldsymbol{\lambda} = \mathbf{1}/m$ for the outer scalarization. The diversity comes from the \emph{set} $\Xset$ itself covering the trade-offs, not from randomizing the scalarization target.

\subsection{Algorithm}

Algorithm~\ref{alg:main} summarizes the full qSTCH-Set BO loop.

\begin{algorithm}[t]
\caption{qSTCH-Set: Many-Objective Bayesian Optimization}
\label{alg:main}
\begin{algorithmic}[1]
\REQUIRE Black-box objectives $f_1, \ldots, f_m$; evaluation budget $T$; batch size $K=m$; smoothing $\mu > 0$; initial data $\D_0$
\FOR{$t = 0, 1, \ldots, T/K-1$}
    \STATE Fit independent GP surrogates $\hat{f}_1, \ldots, \hat{f}_m$ on $\D_t$ \hfill \emph{[Mat\'ern-5/2, MLE]}
    \STATE Estimate ideal point: $z_i^* \leftarrow \min_{j \le t} y_{j,i} - \epsilon$, \; $i = 1, \ldots, m$
    \STATE Set $\boldsymbol{\lambda} \leftarrow \mathbf{1}/m$ \hfill \emph{[uniform coverage]}
    \STATE Draw $N$ quasi-Monte Carlo base samples $\{\boldsymbol{\omega}^{(n)}\}_{n=1}^N$
    \STATE Construct acquisition: $\alpha(\Xset) = \frac{1}{N}\sum_{n=1}^N \left[-g^{\text{STCH-Set}}_\mu(\hat{\mathbf{f}}^{(n)}(\Xset) \mid \boldsymbol{\lambda})\right]$
    \STATE Solve: $\Xset^* \leftarrow \argmax_{\Xset \subset \X,\, |\Xset|=K} \alpha(\Xset)$ \hfill \emph{[L-BFGS-B, 20 restarts]}
    \STATE Evaluate: $\mathbf{y}^{(k)} \leftarrow \mathbf{f}(\mathbf{x}^{(k)})$ for each $\mathbf{x}^{(k)} \in \Xset^*$
    \STATE Update: $\D_{t+1} \leftarrow \D_t \cup \{(\mathbf{x}^{(k)}, \mathbf{y}^{(k)})\}_{k=1}^{K}$
\ENDFOR
\RETURN Non-dominated solutions from $\D_{final}$
\end{algorithmic}
\end{algorithm}

%=============================================================================
% THEORY SECTION — Hardened version with rigorous propositions
% Drop into main.tex as \section{Theoretical Analysis}
% Requires: amsmath, amssymb, amsthm (or neurips theorem environments)
%
% COMPILE NOTE: Ensure the following are in the preamble of main.tex:
%   \newtheorem{proposition}{Proposition}
%   \newtheorem{corollary}{Corollary}
%   \newtheorem{conjecture}{Conjecture}
%   \newtheorem{remark}{Remark}
%   \theoremstyle{definition}
%   \newtheorem{definition}{Definition}
%   \newtheorem{assumption}{Assumption}
%=============================================================================

\section{Theoretical Analysis}
\label{sec:theory}

We establish the theoretical properties of qSTCH-Set as a BO acquisition function.
We distinguish sharply between results that follow rigorously from existing theory
(Propositions~\ref{prop:valid}--\ref{prop:pareto-transfer}),
and those that remain conjectural (Conjecture~\ref{conj:consistency}).
Throughout, $\X \subset \R^d$ is compact and $f_1,\ldots,f_m:\X\to\R$ are the true (unknown) objectives.

%--------------------------------------------------------------------
\subsection{Definitions}
\label{sec:theory-defs}
%--------------------------------------------------------------------

\begin{definition}[Pareto optimality]
\label{def:pareto}
A point $\mathbf{x}\in\X$ is \emph{weakly Pareto optimal} if there is no $\mathbf{x}'\in\X$
with $f_i(\mathbf{x}') < f_i(\mathbf{x})$ for every $i\in[m]$.
It is \emph{Pareto optimal} if there is no $\mathbf{x}'\in\X$
with $f_i(\mathbf{x}') \le f_i(\mathbf{x})$ for all $i$ and $f_j(\mathbf{x}') < f_j(\mathbf{x})$
for some $j$.
\end{definition}

\begin{definition}[$\varepsilon$-Pareto optimality]
\label{def:eps-pareto}
A point $\mathbf{x}\in\X$ is \emph{$\varepsilon$-Pareto optimal} (for $\varepsilon\ge 0$)
if there is no $\mathbf{x}'\in\X$ with $f_i(\mathbf{x}') \le f_i(\mathbf{x}) - \varepsilon$ for every $i\in[m]$.
\end{definition}

\begin{definition}[Smooth Tchebycheff Set scalarization]
\label{def:stchset}
For a candidate set $\Xset = \{\mathbf{x}^{(1)},\ldots,\mathbf{x}^{(K)}\}\subset\X$,
preference vector $\boldsymbol{\lambda}\in\Delta^{m-1}_{++}$
($\lambda_i>0$, $\sum_i \lambda_i=1$),
reference point $\mathbf{z}^*\in\R^m$, and smoothing parameter $\mu>0$:
\begin{align}
g_\mu^{\mathrm{STCH\text{-}Set}}(\Xset\mid\boldsymbol{\lambda})
&= \mu\log\!\left(\sum_{i=1}^{m}\exp\!\left(
    \frac{\lambda_i\bigl(\smin_\mu^{(k)} f_i(\mathbf{x}^{(k)}) - z_i^*\bigr)}{\mu}
\right)\right), \label{eq:stchset-def}\\
\text{where}\quad
\smin_\mu^{(k)} f_i(\mathbf{x}^{(k)})
&= -\mu\log\!\left(\sum_{k=1}^{K}\exp\!\left(-\frac{f_i(\mathbf{x}^{(k)})}{\mu}\right)\right).
\label{eq:smin-def}
\end{align}
The outer log-sum-exp is a smooth approximation to $\max_{i}$; the inner negative
log-sum-exp of negatives is a smooth approximation to $\min_{k}$.
\end{definition}

\begin{definition}[qSTCH-Set acquisition function]
\label{def:qstchset}
Given GP posteriors $\hat{f}_1^{(t)},\ldots,\hat{f}_m^{(t)}$ after $t$ observations,
the \emph{qSTCH-Set acquisition function} for a candidate set $\Xset$ is:
\begin{equation}
\label{eq:acq}
\alpha_t^{\mathrm{qSTCH}}(\Xset)
= \E_{\boldsymbol{\omega}}\!\left[
    -g_\mu^{\mathrm{STCH\text{-}Set}}\!\left(
        \hat{\mathbf{f}}_\omega^{(t)}(\Xset)\mid\boldsymbol{\lambda}
    \right)
\right],
\end{equation}
where $\hat{\mathbf{f}}_\omega^{(t)}$ denotes a joint posterior sample path (indexed by
base sample $\boldsymbol{\omega}$), obtained via the reparameterization trick.
In practice, the expectation is approximated by $N$ quasi-Monte Carlo base samples.
\end{definition}

%--------------------------------------------------------------------
\subsection{Proposition 1: qSTCH-Set Is a Valid MC Acquisition Function}
\label{sec:theory-prop1}
%--------------------------------------------------------------------

\begin{proposition}[Validity as MC acquisition function]
\label{prop:valid}
The qSTCH-Set acquisition function $\alpha_t^{\mathrm{qSTCH}}(\Xset)$
defined in~\eqref{eq:acq} satisfies the following:
\begin{enumerate}
    \item[\textnormal{(a)}] \textbf{Measurability.}
        For every fixed $\Xset\in\X^K$, the integrand
        $\omega\mapsto -g_\mu^{\mathrm{STCH\text{-}Set}}(\hat{\mathbf{f}}_\omega^{(t)}(\Xset)\mid\boldsymbol{\lambda})$
        is a Borel-measurable function of the base samples~$\boldsymbol{\omega}$.
    \item[\textnormal{(b)}] \textbf{Finite expectation.}
        If the GP posteriors have bounded support on any compact $\Xset\subset\X^K$
        (which holds almost surely for GPs with continuous kernels on compact domains),
        then $\E[|\alpha_t^{\mathrm{qSTCH}}(\Xset)|]<\infty$.
    \item[\textnormal{(c)}] \textbf{Differentiability.}
        $\alpha_t^{\mathrm{qSTCH}}(\Xset)$ is differentiable with respect to
        $\Xset = (\mathbf{x}^{(1)},\ldots,\mathbf{x}^{(K)}) \in \X^K$,
        with gradients computable via the reparameterization trick and automatic differentiation.
        Specifically, for each $\mathbf{x}^{(k)}$:
        \begin{equation}
        \label{eq:grad}
        \nabla_{\mathbf{x}^{(k)}}\, g_\mu^{\mathrm{STCH\text{-}Set}}
        = \sum_{i=1}^m w_i \cdot p_{ik}\cdot \nabla f_i(\mathbf{x}^{(k)}),
        \end{equation}
        where the softmax attention weights are
        \begin{equation}
        \label{eq:weights}
        w_i = \frac{\exp\!\bigl(\lambda_i(R_i^{\min}-z_i^*)/\mu\bigr)}
              {\sum_j \exp\!\bigl(\lambda_j(R_j^{\min}-z_j^*)/\mu\bigr)},\quad
        p_{ik} = \frac{\exp\!\bigl(-f_i(\mathbf{x}^{(k)})/\mu\bigr)}
              {\sum_\ell \exp\!\bigl(-f_i(\mathbf{x}^{(\ell)})/\mu\bigr)},
        \end{equation}
        with $R_i^{\min} = \smin_\mu^{(k)} f_i(\mathbf{x}^{(k)})$.
\end{enumerate}
\end{proposition}

\begin{proof}
\textbf{(a)}
The GP posterior sample paths $\hat{f}_{i,\omega}^{(t)}(\mathbf{x})$ are constructed
via the reparameterization trick as $\hat{f}_{i,\omega}^{(t)}(\mathbf{x})
= \mu_i^{(t)}(\mathbf{x}) + L_i^{(t)}(\mathbf{x})\,\omega_i$,
where $L_i^{(t)}$ is obtained from the Cholesky decomposition of the posterior covariance
(or, in the pathwise conditioning approach~\cite{wilson2018maxvalue},
from a basis-function representation).
In either case, for fixed $\mathbf{x}$, the map
$\omega_i\mapsto \hat{f}_{i,\omega}^{(t)}(\mathbf{x})$ is an affine function
of standard normal random variables, hence Borel-measurable.
The STCH-Set scalarization~\eqref{eq:stchset-def} is a composition of
$\exp$, $\log$, $\sum$, and scalar multiplication applied to these measurable functions.
Since compositions and finite sums of measurable functions are measurable,
the integrand is Borel-measurable.

\textbf{(b)}
On the compact set $\X$, any GP with a continuous kernel
$k_i:\X\times\X\to\R$ has sample paths that are almost surely
continuous, and hence bounded on $\X$ (continuous functions on compact sets
are bounded; see~\cite{rasmussen2006gp}, \S4.1).
Let $M = \max_{i,k,\omega}|\hat{f}_{i,\omega}^{(t)}(\mathbf{x}^{(k)})|<\infty$ a.s.
Then $|g_\mu^{\mathrm{STCH\text{-}Set}}|\le |M| + |z_{\max}^*| + \mu\log m$,
where $z_{\max}^* = \max_i|z_i^*|$. Hence the expectation is finite.

\textbf{(c)}
The STCH-Set scalarization~\eqref{eq:stchset-def} is a composition
of $C^\infty$ functions ($\exp$, $\log$, affine maps) with strictly positive arguments
inside the $\log$ (since each summand $\exp(\cdot)>0$). By the chain rule,
$g_\mu^{\mathrm{STCH\text{-}Set}}$ is $C^\infty$ with respect to its arguments
$\{f_i(\mathbf{x}^{(k)})\}_{i,k}$.

The gradient~\eqref{eq:grad} follows by direct computation.
Applying the chain rule to the outer log-sum-exp yields the
softmax weights $w_i$ (attention over objectives). Applying the chain rule
to the inner negative log-sum-exp yields the softmin weights $p_{ik}$
(attention over candidates for each objective).
The product $w_i \cdot p_{ik}$ determines how strongly objective $i$
influences the movement of candidate $\mathbf{x}^{(k)}$.

Since the reparameterization trick
expresses $\hat{f}_{i,\omega}^{(t)}(\mathbf{x})$ as a differentiable function
of $\mathbf{x}$ (for kernels with differentiable mean and covariance functions,
such as Mat\'ern with $\nu>1$ or the squared exponential), the
composition $g_\mu^{\mathrm{STCH\text{-}Set}}(\hat{\mathbf{f}}_\omega^{(t)}(\Xset))$
is differentiable in $\Xset$ for each $\omega$, and the expectation~\eqref{eq:acq}
can be differentiated under the integral sign
(by dominated convergence, using the a.s.\ boundedness from part~(b)
and the Lipschitz continuity of the gradient).
\end{proof}

\begin{remark}[Comparison to non-smooth Chebyshev]
The classical Tchebycheff scalarization~\eqref{eq:tch} uses a hard $\max$,
which is non-differentiable when two or more objectives tie.
This prevents direct use of the reparameterization trick for MC gradient estimation
in BoTorch.
The STCH-Set formulation resolves this entirely: $\alpha_t^{\mathrm{qSTCH}}$
is $C^\infty$ for all $\mu>0$, enabling standard L-BFGS-B acquisition optimization.
This is in contrast to qNParEGO~\cite{daulton2020qnehvi,knowles2006parego},
which uses the augmented Chebyshev scalarization with hard $\max$
and requires heuristic smoothing or restart strategies.
\end{remark}

%--------------------------------------------------------------------
\subsection{Proposition 2: Pareto Optimality of Acquisition Maximizers}
\label{sec:theory-prop2}
%--------------------------------------------------------------------

The following result transfers the Pareto optimality guarantee of Lin et al.~\cite{lin2025few}
from the gradient-based setting to the BO posterior.

\begin{assumption}[GP surrogate regularity]
\label{asm:gp}
Each $f_i$ is modeled by an independent GP with a continuous positive-definite kernel
$k_i$ on the compact domain $\X\subset\R^d$.
The posterior mean $\mu_i^{(t)}$ and variance $(\sigma_i^{(t)})^2$ are computed
from $t$ (possibly noisy) observations.
\end{assumption}

\begin{proposition}[Pareto optimality under the posterior]
\label{prop:pareto-transfer}
Let Assumption~\ref{asm:gp} hold, and let $\boldsymbol{\lambda}\in\Delta^{m-1}_{++}$.
Define the \emph{posterior-mean STCH-Set scalarization}:
\begin{equation}
\label{eq:posterior-stchset}
\hat{g}_\mu^{(t)}(\Xset\mid\boldsymbol{\lambda})
:= g_\mu^{\mathrm{STCH\text{-}Set}}(\Xset\mid\boldsymbol{\lambda})\big|_{f_i=\mu_i^{(t)}}.
\end{equation}
Then:
\begin{enumerate}
    \item[\textnormal{(a)}]
    \textbf{Surrogate Pareto optimality.}
    If $\Xset^*=\arg\min_{\Xset\in\X^K}\hat{g}_\mu^{(t)}(\Xset\mid\boldsymbol{\lambda})$,
    then every $\mathbf{x}^{(k)}\in\Xset^*$ is weakly Pareto optimal
    with respect to the posterior means $(\mu_1^{(t)},\ldots,\mu_m^{(t)})$.
    If additionally $\Xset^*$ is unique, every $\mathbf{x}^{(k)}\in\Xset^*$
    is Pareto optimal w.r.t.\ $(\mu_1^{(t)},\ldots,\mu_m^{(t)})$.

    \item[\textnormal{(b)}]
    \textbf{Approximation bound.}
    The smooth scalarization satisfies the sandwich inequality:
    \begin{equation}
    \label{eq:sandwich}
    g^{\mathrm{TCH\text{-}Set}}(\Xset\mid\boldsymbol{\lambda})
    \;\le\;
    g_\mu^{\mathrm{STCH\text{-}Set}}(\Xset\mid\boldsymbol{\lambda})
    \;\le\;
    g^{\mathrm{TCH\text{-}Set}}(\Xset\mid\boldsymbol{\lambda}) + \mu\log m + \mu\log K.
    \end{equation}
    The smoothing gap $\mu\log(m) + \mu\log(K)$ is uniform over $\Xset$
    and controlled by the user-chosen~$\mu$.
\end{enumerate}
\end{proposition}

\begin{proof}
\textbf{(a)}
The GP posterior means $\mu_1^{(t)},\ldots,\mu_m^{(t)}$ are $C^\infty$ functions
$\X\to\R$ (for any standard kernel with $C^\infty$ realizations, such as the
squared exponential; for Mat\'ern-$\nu$, $\mu_i^{(t)}$ is at least $C^{\lceil\nu\rceil}$).
In particular, they satisfy the differentiability requirements of
Theorem~2 in~\cite{lin2025few}.
Since $\lambda_i>0$ for all $i$ (by the choice $\boldsymbol{\lambda}\in\Delta_{++}^{m-1}$),
all hypotheses of~\cite[Theorem~2]{lin2025few} are satisfied with
the ``objectives'' taken to be $\mu_1^{(t)},\ldots,\mu_m^{(t)}$.
The conclusion follows directly: all solutions in $\Xset^*$ are weakly Pareto optimal
w.r.t.\ the surrogate objectives, and Pareto optimal when $\Xset^*$ is unique.

\textbf{(b)}
The lower bound follows because log-sum-exp
upper-bounds the max: for any $a_1,\ldots,a_m\in\R$ and $\mu>0$,
$\max_i a_i \le \mu\log(\sum_i e^{a_i/\mu})$.

For the upper bound, we decompose the smoothing error into two terms:
\begin{itemize}
\item \emph{Outer (smax vs.\ max):}
$\mu\log(\sum_i e^{a_i/\mu}) \le \max_i a_i + \mu\log m$.
This is the standard log-sum-exp bound (Proposition~3.4 of~\cite{lin2024smooth}).
\item \emph{Inner (smin vs.\ min):}
For each $i$, $\min_k b_k - \mu\log K \le \smin_\mu^{(k)} b_k \le \min_k b_k$.
This follows from the analogous bound for the smooth minimum:
$-\mu\log(\sum_k e^{-b_k/\mu}) \le \min_k b_k$ (immediate) and
$\min_k b_k - \mu\log K \le -\mu\log(\sum_k e^{-b_k/\mu})$
(since $\sum_k e^{-b_k/\mu} \le K \cdot e^{-\min_k b_k/\mu}$).
\end{itemize}
The inner error contributes at most $\mu\log K$ to each term inside the outer
log-sum-exp. Since $\boldsymbol{\lambda}\in\Delta^{m-1}$ (i.e., $\sum_i\lambda_i=1$),
the per-objective shift of at most $\mu\log K$ translates to an additive error
of at most $\mu\log K$ in the outer scalarization.
Combining: $g_\mu^{\mathrm{STCH\text{-}Set}} \le g^{\mathrm{TCH\text{-}Set}} + \mu\log m + \mu\log K$.
\end{proof}

\begin{corollary}[Asymptotic quality with $\mu$-annealing]
\label{cor:annealing}
If the smoothing parameter is reduced as $\mu_t = c/\log(t+1)$ for a constant $c>0$,
then the smoothing gap satisfies
$\mu_t\log(mK)\to 0$ as $t\to\infty$.
Thus, assuming part~(a) continues to hold (i.e., the STCH-Set optimization finds a
global minimizer), the solutions converge in scalarization value to the true
TCH-Set optimum.
\end{corollary}

\begin{remark}[Scope of Proposition~\ref{prop:pareto-transfer}]
\label{rem:scope}
Part~(a) is a \emph{surrogate-space} guarantee: it says the acquisition function
searches in the right part of the Pareto front of the \emph{model}.
It does \emph{not} directly imply Pareto optimality of the true objectives $\mathbf{f}$,
because the GP posterior mean $\mu_i^{(t)}$ may differ from $f_i$.
Bridging the gap requires posterior convergence, which we address in
Conjecture~\ref{conj:consistency} below.
Note that this surrogate-space guarantee is the same type of guarantee
enjoyed by all model-based BO methods (e.g., qEHVI optimizes hypervolume
of the \emph{posterior}, not the true Pareto front).
\end{remark}

%--------------------------------------------------------------------
\subsection{Conjecture 3: Consistency Under GP Posterior Convergence}
\label{sec:theory-conj}
%--------------------------------------------------------------------

We now state the key question: \emph{Does qSTCH-Set BO converge to the true Pareto front
as the number of observations grows?}
For the $K=1$ case, this reduces to composite BO with a smooth scalarization,
for which consistency follows from Astudillo \& Frazier~\cite{astudillo2019composite}.
The $K>1$ case is novel and requires additional arguments that we have not
been able to complete rigorously. We state it as a conjecture with supporting
intuition.

\begin{assumption}[RKHS regularity]
\label{asm:rkhs}
Each $f_i$ lies in the RKHS $\mathcal{H}_{k_i}$ of its kernel $k_i$,
with $\|f_i\|_{\mathcal{H}_{k_i}}\le B$.
The kernels satisfy $k_i(\mathbf{x},\mathbf{x})\le 1$ for all $\mathbf{x}\in\X$.
\end{assumption}

\begin{assumption}[Observation model]
\label{asm:obs}
Observations are $y_{i,t} = f_i(\mathbf{x}_t) + \varepsilon_{i,t}$
where $\varepsilon_{i,t}$ are i.i.d.\ $\sigma$-sub-Gaussian.
\end{assumption}

\begin{conjecture}[Consistency of qSTCH-Set BO]
\label{conj:consistency}
Under Assumptions~\ref{asm:gp}--\ref{asm:obs}, suppose the qSTCH-Set acquisition
function~\eqref{eq:acq} is optimized at each BO iteration $t$ with
$\boldsymbol{\lambda}\in\Delta^{m-1}_{++}$ and $\mu>0$ (possibly decreasing in $t$).
Let $\Xset_t^*$ denote the set selected at iteration $t$.
Then, as $t\to\infty$, with probability at least $1-\delta$, every
$\mathbf{x}^{(k)}\in\Xset_t^*$ is $\varepsilon_t$-Pareto optimal for the true
objectives $\mathbf{f}$, where
\begin{equation}
\label{eq:eps-rate}
\varepsilon_t = \mu_t\log(mK) + O\!\left(\beta_t^{1/2}\,\bar{\sigma}_t\right)
\xrightarrow{t\to\infty} 0
\end{equation}
if $\mu_t\to 0$ at an appropriate rate.
Here $\beta_t = O(B^2 + \gamma_t\log^3(t/\delta))$,
$\gamma_t$ is the maximum information gain of the kernel,
and $\bar{\sigma}_t = \max_{i\in[m]}\max_{\mathbf{x}\in\Xset_t^*}\sigma_i^{(t)}(\mathbf{x})$.
\end{conjecture}

\paragraph{Why we believe this but cannot prove it.}
The argument would need to combine three ingredients:

\begin{enumerate}
\item \textbf{GP posterior concentration (established).}
Under Assumptions~\ref{asm:rkhs}--\ref{asm:obs}, Srinivas et al.~\cite{srinivas2010ucb}
(Theorem~6) provide uniform confidence bounds:
with probability $\ge 1-\delta/m$ (union-bounded over objectives),
$|f_i(\mathbf{x})-\mu_i^{(t)}(\mathbf{x})|\le\beta_t^{1/2}\sigma_i^{(t)}(\mathbf{x})$
simultaneously for all $\mathbf{x}\in\X$ and $t\ge 1$.
The information gain $\gamma_t$ grows sub-linearly for standard kernels
(e.g., $\gamma_t = O((\log t)^{d+1})$ for the SE kernel), ensuring
$\beta_t^{1/2}\sigma_i^{(t)}(\mathbf{x})\to 0$ in well-explored regions.

\item \textbf{STCH-Set Lipschitz stability (straightforward but not formalized).}
The STCH-Set scalarization is Lipschitz in its objective values.
Let $L_\mu$ denote the Lipschitz constant of $g_\mu^{\mathrm{STCH\text{-}Set}}$
with respect to the objective vector $\bigl(f_i(\mathbf{x}^{(k)})\bigr)_{i,k}$
in the $\ell^\infty$ norm. Then, under the GP concentration event,
\begin{equation}
\bigl|g_\mu^{\mathrm{STCH\text{-}Set}}(\hat{\mathbf{f}}^{(t)}(\Xset))
 - g_\mu^{\mathrm{STCH\text{-}Set}}(\mathbf{f}(\Xset))\bigr|
\le L_\mu \cdot \beta_t^{1/2}\bar{\sigma}_t.
\end{equation}
A bound $L_\mu \le 2$ (with our normalization $\boldsymbol{\lambda}\in\Delta^{m-1}$)
can be obtained from the fact that both the log-sum-exp and
negative-log-sum-exp-of-negatives have gradient entries summing to~1,
but we have not verified all corner cases for the composition rigorously.

\item \textbf{Acquisition optimization is near-global (not provable in general).}
A complete consistency proof requires that the acquisition function is optimized to
within $\varepsilon$ of the global optimum at each step. This is assumed
(explicitly or implicitly) in all GP-based BO analyses, including GP-UCB~\cite{srinivas2010ucb},
EI~\cite{balandat2020botorch}, and composite BO~\cite{astudillo2019composite}.
In practice, multi-start L-BFGS-B provides this empirically but it is
not formally guaranteed for the non-convex acquisition landscape of qSTCH-Set
with $K>1$ candidates (the decision space $\X^K$ has dimension $Kd$).
\end{enumerate}

\paragraph{The $K=1$ case.}
When $K=1$, qSTCH-Set reduces to single-point STCH scalarization composed with the GP posterior.
This falls within the composite BO framework of Astudillo \& Frazier~\cite{astudillo2019composite},
who prove consistency for acquisition functions of the form $h(\mathbf{g}(\mathbf{x}))$
where $h$ is a known function and $\mathbf{g}$ is modeled by a GP.
In our case, $h = -g_\mu^{\mathrm{STCH}}$ and $\mathbf{g} = (f_1,\ldots,f_m)$.
Their Theorem~1 establishes that the composite EI acquisition function
is consistent (i.e., the optimization gap vanishes) under GP posterior consistency.
For $K=1$, our method inherits this guarantee directly.

\paragraph{The $K>1$ gap.}
For $K>1$, the decision variable is a \emph{set} $\Xset\in\X^K$,
and the scalarization $g_\mu^{\mathrm{STCH\text{-}Set}}$ involves a smooth minimum
over the $K$ candidate evaluations.
The composite BO framework of~\cite{astudillo2019composite} does not directly cover this case
because:
(i) the ``outer function'' $h$ now depends on the GP outputs at $K$ different input locations
jointly, not at a single $\mathbf{x}$; and
(ii) the set-valued optimization introduces symmetries and redundancies
(permutation invariance of $\Xset$) that complicate the convergence analysis.
Extending the Astudillo-Frazier consistency argument to this joint-set
setting is the main open theoretical challenge.
We conjecture it holds by analogy: the STCH-Set scalarization is continuous
and the GP posterior concentrates uniformly, so the set-valued optimization
problem should converge to its deterministic counterpart.
A rigorous proof would likely require set-valued epi-convergence arguments
(see, e.g., Rockafellar \& Wets, \emph{Variational Analysis}, Chapter~7).

%--------------------------------------------------------------------
\subsection{Computational Complexity}
\label{sec:theory-complexity}
%--------------------------------------------------------------------

\begin{proposition}[Per-evaluation complexity]
\label{prop:complexity}
The qSTCH-Set acquisition function with $K$ candidate points, $m$ objectives,
and $N$ MC base samples can be evaluated in $O(NKm)$ time
(excluding GP posterior sampling cost) and $O(Km)$ space per MC sample.
In contrast:
\begin{itemize}
    \item \emph{qEHVI}~\cite{daulton2020qnehvi,daulton2021qnehvi}:
        Hypervolume computation is $\#P$-hard in $m$; the best exact algorithms
        require time exponential in $m$ in the worst case.
    \item \emph{qNParEGO}~\cite{knowles2006parego,daulton2020qnehvi}:
        $O(Nm)$ per evaluation (linear in $m$), but uses a single
        random $\boldsymbol{\lambda}$ per iteration without coordinating solutions.
\end{itemize}
\end{proposition}

\begin{proof}
The STCH-Set computation~\eqref{eq:stchset-def}--\eqref{eq:smin-def} requires:
(1)~weighted deviations $\lambda_i(f_i(\mathbf{x}^{(k)})-z_i^*)/\mu$ for all $i\in[m]$,
$k\in[K]$: $O(Km)$ operations;
(2)~smooth min via $\mathrm{logsumexp}$ over $k$ for each $i$: $O(K)$ per objective,
$O(Km)$ total;
(3)~smooth max via $\mathrm{logsumexp}$ over $i$: $O(m)$.
The gradient~\eqref{eq:grad} can be computed by automatic differentiation
with the same asymptotic cost (using the PyTorch \texttt{logsumexp} implementation,
which applies the max-subtraction trick for numerical stability).
Summing over $N$ MC samples gives $O(NKm)$.
\end{proof}

%--------------------------------------------------------------------
\subsection{Summary of Theoretical Status}
\label{sec:theory-summary}
%--------------------------------------------------------------------

Table~\ref{tab:theory-status} summarizes what is proved, what is transferred from
existing results, and what remains conjectural.

\begin{table}[t]
\caption{Theoretical status of qSTCH-Set results.}
\label{tab:theory-status}
\centering
\small
\begin{tabular}{llp{7cm}}
\toprule
\textbf{Result} & \textbf{Status} & \textbf{Key Dependency} \\
\midrule
Prop.~\ref{prop:valid}: Valid MC acquisition
    & Proved
    & Composition of measurable/smooth functions \\
Prop.~\ref{prop:pareto-transfer}(a): Surrogate Pareto opt.
    & Proved (transfer)
    & Lin et al.~\cite{lin2025few}, Theorem~2 \\
Prop.~\ref{prop:pareto-transfer}(b): Sandwich bound
    & Proved
    & Standard log-sum-exp bounds~\cite{lin2024smooth} \\
Cor.~\ref{cor:annealing}: $\mu$-annealing
    & Proved
    & Immediate from Prop.~\ref{prop:pareto-transfer}(b) \\
Prop.~\ref{prop:complexity}: $O(NKm)$ complexity
    & Proved
    & Direct computation \\
Conj.~\ref{conj:consistency}: Consistency ($K>1$)
    & \textbf{Conjecture}
    & Extends~\cite{astudillo2019composite}; requires set-valued convergence \\
Conj.~\ref{conj:consistency} ($K=1$ case)
    & Proved (by~\cite{astudillo2019composite})
    & Composite BO consistency \\
\bottomrule
\end{tabular}
\end{table}


%=============================================================================
\section{Experiments}
\label{sec:experiments}
%=============================================================================

We evaluate qSTCH-Set on standard multi-objective benchmarks (DTLZ2, ZDT2) to demonstrate its scalability and effectiveness in many-objective regimes ($m \ge 5$) where traditional hypervolume-based methods are computationally intractable. Our implementation is built on BoTorch~\citep{balandat2020botorch} and is available at \url{https://github.com/parameters/qSTCH-Set}.

\subsection{Experimental Setup}

We compare the following methods on DTLZ2 problems with $m \in \{5, 8, 10\}$ objectives. All experiments use H100 GPUs on the Digital Research Alliance of Canada (Nibi cluster).

\paragraph{Methods.}
\begin{itemize}
    \item \textbf{qSTCH-Set} (ours): Set-based smooth Tchebycheff acquisition with $q = K = m$, $\mu = 0.1$, fixed uniform outer weights $\boldsymbol{\lambda} = \mathbf{1}/m$, and $N = 256$ MC samples.
    \item \textbf{STCH-NParEGO}: Single-point smooth scalarization ($q = 1$) with random weights.
    \item \textbf{qNParEGO}~\citep{daulton2020qnehvi}: Standard batch ParEGO with random Chebyshev scalarization ($q = 1$).
    \item \textbf{Random}: Uniform random search ($q=1$).
\end{itemize}

\paragraph{Protocol.} For DTLZ2 ($d=m+4$), we use $5$ independent seeds for $m=5$ and $3$ seeds for $m=8,10$. We initialize with $20$ Sobol points. The evaluation budget is set to allow comparable exploration: 30 iterations for $m=5$, 25 for $m=8$, and 20 for $m=10$. Note that qSTCH-Set evaluates $m$ points per iteration, while baselines evaluate 1. This design choice reflects the method's purpose: finding a coordinated set of solutions.

\subsection{Main Results: DTLZ2 Scaling}

Table~\ref{tab:main} presents the final hypervolume (HV) achieved by each method across different numbers of objectives.

\begin{table}[t]
\caption{Main Results on DTLZ2: Final log-hypervolume (mean $\pm$ std). Best method per column is \textbf{bolded}. qSTCH-Set consistently outperforms scalarization baselines, with the gap widening as $m$ increases. ($m=10$ results pending).}
\label{tab:main}
\centering
\begin{tabular}{lccc}
\toprule
Method & $m=5$ & $m=8$ & $m=10$ \\
\midrule
\textbf{qSTCH-Set (ours)} & $\mathbf{6.646 \pm 0.066}$ & $\mathbf{24.108 \pm 0.314}$ & Pending \\
qNParEGO & $6.429 \pm 0.254$ & $21.606 \pm 0.826$ & Pending \\
STCH-NParEGO & $6.117 \pm 0.156$ & $20.620 \pm 1.459$ & Pending \\
Random & $5.370 \pm 0.135$ & $18.026 \pm 0.196$ & Pending \\
\bottomrule
\end{tabular}
\end{table}

\textbf{Performance at $m=5$.} qSTCH-Set achieves a mean HV of $6.646$, significantly outperforming qNParEGO ($6.429$) and STCH-NParEGO ($6.117$). The reduced standard deviation ($0.066$ vs $0.254$) indicates that set-based coordination yields more consistent Pareto front coverage than random weight sampling.

\textbf{Scaling to $m=8$.} The advantage of set-based acquisition becomes more pronounced at $m=8$. qSTCH-Set achieves HV $24.108$, surpassing qNParEGO ($21.606$) by over 2.5 units (approx. 11\%). In this high-dimensional objective space, random scalarization weights rarely align with the specific trade-off directions required to expand the hypervolume efficiently. qSTCH-Set, by assigning one candidate to each objective ($K=m$), ensures systematic expansion along all axes simultaneously.

\subsection{Ablation: Batch Size $K$}

To understand the importance of the design rule $K=m$, we compare performance with different batch sizes. (Table~\ref{tab:ablation}).

\begin{table}[h]
\caption{K-Ablation on DTLZ2 ($m=8$): Impact of set size $K$ on final Hypervolume. (Results pending).}
\label{tab:ablation}
\centering
\begin{tabular}{lc}
\toprule
Configuration & Final HV \\
\midrule
qSTCH-Set ($K=m=8$) & $\mathbf{24.108 \pm 0.314}$ \\
qSTCH-Set ($K=5 < m$) & Pending \\
qSTCH-Set ($K=10 > m$) & Pending \\
\bottomrule
\end{tabular}
\end{table}

\subsection{Computational Cost}

We analyze the wall-clock time per iteration on an NVIDIA H100 GPU:

\begin{itemize}
    \item \textbf{$m=5$}: qSTCH-Set averages $\sim 22.6$s/iter, comparable to qNParEGO ($\sim 24.6$s/iter). Despite the larger batch size $q=5$, the joint optimization is efficient.
    \item \textbf{$m=8$}: qSTCH-Set optimization time increases to $\sim 335$s/iter, compared to $\sim 45$s/iter for qNParEGO. This increase reflects the $O(Nm^2)$ complexity of the STCH-Set acquisition and the higher-dimensional search space ($q \times d = 8 \times 12 = 96$ variables). While slower, the cost remains negligible compared to expensive black-box function evaluations (e.g., wet-lab experiments).
\end{itemize}

\subsection{Validation on Bi-Objective Problems}

On ZDT2 ($m=2$), STCH-NParEGO achieves HV $107.2 \pm 4.1$, slightly outperforming vanilla qNParEGO ($106.0 \pm 4.9$), confirming that the smooth approximation is effective even in the single-point regime. However, qEHVI remains the gold standard for $m=2$ ($111.1 \pm 2.2$), supporting our positioning of qSTCH-Set for many-objective problems where qEHVI is inapplicable.

%=============================================================================
\section{Related Work}
\label{sec:related}
%=============================================================================

\paragraph{Hypervolume-based MOBO.}
Expected hypervolume improvement (EHVI) and its Monte Carlo extensions qEHVI and qNEHVI~\citep{daulton2020qnehvi,daulton2021qnehvi} define the state of the art for multi-objective BO with $m \le 4$ objectives, optimizing the expected improvement in the dominated hypervolume indicator.
However, exact hypervolume computation requires non-dominated partitioning, which is \#P-hard in~$m$~\citep{wang2024pohvi}.
Recent work on $\varepsilon$-PoHVI~\citep{wang2024pohvi} provides exact posterior hypervolume integration but remains limited to small~$m$.
For $m > 5$, these methods become computationally intractable.

\paragraph{Scalarization-based MOBO.}
ParEGO~\citep{knowles2006parego} pioneered random Chebyshev scalarization for MOBO, converting the multi-objective problem to a sequence of single-objective subproblems via randomly sampled weight vectors.
qNParEGO~\citep{daulton2020qnehvi} extended this to batches via sequential greedy selection, and Paria et al.~\citep{paria2019mobo} provided regret analysis for random scalarization.
MOBO-OSD~\citep{mobo_osd2025} selects batch points via orthogonal search directions.
All these methods rely on uncoordinated weight selection: each batch element optimizes an independently sampled weight with no mechanism to ensure collective Pareto front coverage.
For $m \gg 5$, random weights concentrate in the interior of the simplex, failing to sample the extremal directions needed for comprehensive front coverage.

\paragraph{Set-based and information-theoretic approaches.}
Set-based acquisition functions optimize the joint utility $U(\{x_1, \ldots, x_q\})$ of a batch rather than a sum of individual utilities.
qEHVI is the canonical example, jointly maximizing expected hypervolume improvement, but inherits the exponential scaling of hypervolume computation.
Information-theoretic methods---MESMO~\citep{belakaria2019mesmo}, PFES~\citep{suzuki2020pfes}, and JES~\citep{tu2022jes}---approximate entropy-based acquisitions but face similar degradation as Pareto front sampling costs grow with~$m$.
MORBO~\citep{daulton2022morbo} scales to high-dimensional \emph{input} spaces ($d > 100$) via local trust regions but addresses input dimensionality rather than objective dimensionality, and was primarily evaluated with $m \le 4$ objectives.

\paragraph{Smooth Tchebycheff scalarization.}
Lin et al.~\citep{lin2024smooth} introduced the smooth Tchebycheff (STCH) scalarization, replacing the non-smooth $\max$ in classical Chebyshev with a log-sum-exp approximation that is everywhere differentiable, has Lipschitz gradients, and preserves (weak) Pareto optimality.
They proved $O(1/\epsilon)$ convergence for gradient-based multi-objective optimization.
Lin et al.~\citep{lin2025few} extended this to STCH-Set, a ``few-for-many'' formulation where $K$ solutions are jointly optimized to cover~$m$ objectives via a nested smooth minimax, scaling as $O(Km)$ and demonstrating results with up to $m{=}1{,}024$ objectives and $K{=}20$ solutions.
However, both methods require cheap, differentiable objectives.
Pires \& Coelho~\citep{pires2025stch} brought single-point STCH into composite Bayesian optimization~\citep{astudillo2019composite}, achieving smooth scalarization in the surrogate-driven loop but without set-based coordination.
Our qSTCH-Set completes this picture: it combines set-based STCH scalarization with Monte Carlo GP posterior sampling, operating at $O(Km)$ cost in the sample-efficient regime where objectives are expensive black-box functions.

\paragraph{Many-objective evolutionary optimization.}
Evolutionary algorithms such as NSGA-III~\citep{deb2014nsga3} and MOEA/D~\citep{zhang2007moead} handle many objectives effectively but typically require thousands of function evaluations, making them unsuitable for expensive black-box problems where each evaluation may cost hours or days.

%=============================================================================
\section{Limitations and Future Work}
\label{sec:limitations}
%=============================================================================

\paragraph{Evaluation budget asymmetry.} Our $m{=}5$ comparison uses $q{=}5$ for qSTCH-Set vs.\ $q{=}1$ for baselines, resulting in different total evaluation counts. While this reflects the method's design---coordinated batch acquisition---a controlled comparison with matched budgets (e.g., $q{=}5$ qNParEGO) would strengthen the empirical case.

\paragraph{Limited objective scale.} Our current experiments reach $m = 8$; the primary motivation for qSTCH-Set is $m \gg 5$. We present initial results for $m=8$ demonstrating scalability, and GPU experiments for $m = 10$ are in progress to further validate performance in the many-objective regime.

\paragraph{Seed count.} The $m = 3$ results use only 2 seeds. Full statistical validation with $\ge 10$ seeds is needed for publication-quality claims.

\paragraph{Comparison to qEHVI.} On bi-objective problems, qEHVI remains clearly superior. qSTCH-Set is designed to complement qEHVI by operating where hypervolume computation is infeasible ($m > 5$).

\paragraph{Acquisition optimization.} Our theoretical analysis assumes the acquisition function is globally optimized (Propositions~\ref{prop:gap}--\ref{prop:pareto}), but L-BFGS-B with 20 random restarts provides no such guarantee. This gap is standard in BO theory but worth noting.

\paragraph{Future directions.}
\emph{Many-objective experiments} ($m = 8, 10, 15, 20$) on GPU are the immediate priority. \emph{Adaptive $\mu$ scheduling}---decreasing $\mu$ as the GP improves---could eliminate the smoothing residual (Corollary in Proposition~\ref{prop:pareto}). \emph{Drug discovery applications} with $m = 20$--$50$ ADMET properties and $K = 3$--$5$ lead candidates represent the ultimate use case. \emph{Hybrid strategies} that use qSTCH-Set for $m > 5$ and qEHVI for $m \le 5$ could leverage the strengths of both.

%=============================================================================
\section{Conclusion}
\label{sec:conclusion}
%=============================================================================

We introduced qSTCH-Set, a Monte Carlo acquisition function that brings set-based smooth Tchebycheff scalarization from gradient-based many-objective optimization to sample-efficient Bayesian optimization. By applying the STCH-Set smooth minimax formulation to GP posterior samples, qSTCH-Set jointly optimizes $K$ candidates to cover $m$ objectives with $O(Km)$ cost and asymptotic Pareto optimality guarantees that transfer from the STCH-Set framework under GP posterior concentration. On DTLZ2 with $m{=}5$ objectives, qSTCH-Set achieves hypervolume $6.646 \pm 0.066$, significantly outperforming both qNParEGO ($6.429 \pm 0.254$) and single-point STCH-NParEGO ($6.117 \pm 0.156$), demonstrating that set-based coordination improves Pareto front coverage in the sample-efficient regime. Our method fills a fundamental gap at the intersection of set-based scalarization and Bayesian optimization, providing a principled path toward optimization with $m \gg 5$ objectives---a regime where no existing BO method operates effectively.

%=============================================================================
\section*{Acknowledgments}
%=============================================================================
\placeholder{Acknowledgments: Compute resources provided by the Digital Research Alliance of Canada (Nibi cluster). R.A.V.-H.\ acknowledges funding from [grant details].}

\bibliographystyle{plainnat}
\bibliography{references}

%=============================================================================
% APPENDIX (external file)
%=============================================================================
%=============================================================================
% APPENDIX — Full proofs, extended results, and implementation details
%=============================================================================
\newpage
\appendix

\section{Full Proof of Proposition~\ref{prop:valid}: qSTCH-Set Is a Valid MC Acquisition Function}
\label{app:proof-valid}

We provide the complete proof of each part of Proposition~\ref{prop:valid}, expanding on the proof sketch in the main text.

\begin{proof}[Proof of Proposition~\ref{prop:valid}]

\textbf{Part (a): Measurability.}

We must show that for every fixed candidate set $\Xset = \{\mathbf{x}^{(1)},\ldots,\mathbf{x}^{(K)}\} \in \X^K$, the map
\[
\boldsymbol{\omega} \mapsto -g_\mu^{\mathrm{STCH\text{-}Set}}\!\left(\hat{\mathbf{f}}_\omega^{(t)}(\Xset) \mid \boldsymbol{\lambda}\right)
\]
is Borel-measurable as a function of the base samples $\boldsymbol{\omega} \in \R^{m \times K}$.

The GP posterior sample paths are constructed via the reparameterization trick:
\begin{equation}
\hat{f}_{i,\omega}^{(t)}(\mathbf{x}) = \mu_i^{(t)}(\mathbf{x}) + \mathbf{k}_i^{(t)}(\mathbf{x})^\top (K_i^{(t)} + \sigma^2 I)^{-1/2} \boldsymbol{\omega}_i,
\end{equation}
where $\mu_i^{(t)}(\mathbf{x})$ is the posterior mean, $\mathbf{k}_i^{(t)}(\mathbf{x})$ is the posterior cross-covariance vector, $K_i^{(t)}$ is the posterior covariance matrix, and $\boldsymbol{\omega}_i \sim \mathcal{N}(\mathbf{0}, I)$.
For fixed $\mathbf{x}$, this is an affine function $\boldsymbol{\omega}_i \mapsto a + \mathbf{b}^\top \boldsymbol{\omega}_i$ with deterministic $a \in \R$ and $\mathbf{b} \in \R^t$, hence Borel-measurable.

Alternatively, in the pathwise conditioning approach~\cite{wilson2018maxvalue}, sample paths take the form:
\begin{equation}
\hat{f}_{i,\omega}^{(t)}(\mathbf{x}) = \sum_{j=1}^{J} w_{ij}(\boldsymbol{\omega}_i) \phi_j(\mathbf{x}) + \text{update}(\mathbf{x}; \boldsymbol{\omega}_i),
\end{equation}
where $\phi_j$ are basis functions and $w_{ij}$ are linear functions of the base samples. In either case, for fixed $\mathbf{x}$, the map $\boldsymbol{\omega}_i \mapsto \hat{f}_{i,\omega}^{(t)}(\mathbf{x})$ is measurable.

The STCH-Set scalarization~\eqref{eq:stchset-def} is constructed from these sample path evaluations via a finite sequence of elementary operations:
\begin{enumerate}
    \item Negation: $f_i(\mathbf{x}^{(k)}) \mapsto -f_i(\mathbf{x}^{(k)})/\mu$ (measurable: affine transformation).
    \item Exponentiation: $-f_i(\mathbf{x}^{(k)})/\mu \mapsto \exp(-f_i(\mathbf{x}^{(k)})/\mu)$ (measurable: continuous function of a measurable function).
    \item Finite summation over $k$: $\sum_{k=1}^{K} \exp(-f_i(\mathbf{x}^{(k)})/\mu)$ (measurable: finite sum of measurable functions).
    \item Logarithm: $\log(\cdot)$ applied to a strictly positive measurable function (measurable: continuous function on $(0,\infty)$).
    \item Scaling and shifting: $\lambda_i(\cdot - z_i^*)/\mu$ (measurable: affine transformation).
    \item Exponentiation and summation over $i$: same argument as steps 2--3.
    \item Final logarithm and scaling by $\mu$: same argument as step 4.
\end{enumerate}

Since each step preserves measurability (compositions of Borel-measurable functions are Borel-measurable, and finite sums/products of measurable functions are measurable), the full map $\boldsymbol{\omega} \mapsto g_\mu^{\mathrm{STCH\text{-}Set}}(\hat{\mathbf{f}}_\omega^{(t)}(\Xset) \mid \boldsymbol{\lambda})$ is Borel-measurable. Negation preserves measurability, completing part~(a). \qed

\medskip
\textbf{Part (b): Finite expectation.}

We must show $\E[|\alpha_t^{\mathrm{qSTCH}}(\Xset)|] < \infty$ when the GP posteriors have continuous kernels on the compact domain $\X$.

By Kolmogorov's continuity theorem (see~\cite{rasmussen2006gp}, \S4.1), GPs with continuous covariance functions $k_i(\mathbf{x}, \mathbf{x}')$ on compact $\X$ have sample paths that are almost surely continuous on $\X$. Since continuous functions on compact sets are bounded (extreme value theorem), for each $i$ and each sample $\omega$:
\begin{equation}
\sup_{\mathbf{x} \in \X} |\hat{f}_{i,\omega}^{(t)}(\mathbf{x})| < \infty \quad \text{a.s.}
\end{equation}

Let $M_\omega := \max_{i \in [m],\, k \in [K]} |\hat{f}_{i,\omega}^{(t)}(\mathbf{x}^{(k)})| < \infty$ a.s. We bound $|g_\mu^{\mathrm{STCH\text{-}Set}}|$ as follows.

\emph{Upper bound.} Since $\smin_\mu^{(k)} f_i(\mathbf{x}^{(k)}) \le \min_k f_i(\mathbf{x}^{(k)}) \le M_\omega$, and each $\lambda_i \in (0,1]$:
\begin{align}
g_\mu^{\mathrm{STCH\text{-}Set}} &= \mu \log\!\left(\sum_{i=1}^{m} \exp\!\left(\frac{\lambda_i(R_i^{\min} - z_i^*)}{\mu}\right)\right) \\
&\le \max_i \lambda_i(R_i^{\min} - z_i^*) + \mu \log m \\
&\le M_\omega + |z_{\max}^*| + \mu \log m,
\end{align}
where $z_{\max}^* := \max_i |z_i^*|$ and we used the standard log-sum-exp upper bound.

\emph{Lower bound.} Since $R_i^{\min} \ge \min_k f_i(\mathbf{x}^{(k)}) - \mu \log K \ge -M_\omega - \mu \log K$:
\begin{equation}
g_\mu^{\mathrm{STCH\text{-}Set}} \ge \max_i \lambda_i(R_i^{\min} - z_i^*) \ge \lambda_{\min}(-M_\omega - \mu\log K - |z_{\max}^*|),
\end{equation}
where $\lambda_{\min} := \min_i \lambda_i > 0$.

Combining: $|g_\mu^{\mathrm{STCH\text{-}Set}}| \le M_\omega + |z_{\max}^*| + \mu\log m + \mu \log K$.

Since $M_\omega$ is a.s.\ finite and has finite expectation (the posterior sample paths are Gaussian linear combinations, hence sub-Gaussian with finite moments of all orders), we conclude $\E[|g_\mu^{\mathrm{STCH\text{-}Set}}|] < \infty$, and therefore $\E[|\alpha_t^{\mathrm{qSTCH}}(\Xset)|] < \infty$. \qed

\medskip
\textbf{Part (c): Differentiability.}

We must show that $\alpha_t^{\mathrm{qSTCH}}(\Xset)$ is differentiable with respect to $\Xset \in \X^K$ and derive the gradient formula~\eqref{eq:grad}.

\emph{Step 1: Pointwise differentiability.}
For each fixed base sample $\omega$, the STCH-Set scalarization $g_\mu^{\mathrm{STCH\text{-}Set}}(\hat{\mathbf{f}}_\omega^{(t)}(\Xset) \mid \boldsymbol{\lambda})$ is a composition of $C^\infty$ functions applied to the sample path evaluations $\{\hat{f}_{i,\omega}^{(t)}(\mathbf{x}^{(k)})\}_{i,k}$.

The arguments inside all logarithms are strictly positive:
\begin{itemize}
    \item Inner: $\sum_{k=1}^{K} \exp(-f_i(\mathbf{x}^{(k)})/\mu) > 0$ since each exponential is positive.
    \item Outer: $\sum_{i=1}^{m} \exp(\lambda_i(R_i^{\min} - z_i^*)/\mu) > 0$ for the same reason.
\end{itemize}
Therefore $\log$ is applied to strictly positive arguments, and the composition of $\exp$, $\log$, $\sum$, and affine maps is $C^\infty$ with respect to $\{f_i(\mathbf{x}^{(k)})\}_{i,k}$.

For kernels with differentiable mean and covariance functions (e.g., the squared exponential kernel, or Mat\'ern-$\nu$ with $\nu > 1$), the posterior sample paths $\hat{f}_{i,\omega}^{(t)}(\mathbf{x})$ are differentiable in $\mathbf{x}$. By the chain rule, $g_\mu^{\mathrm{STCH\text{-}Set}}(\hat{\mathbf{f}}_\omega^{(t)}(\Xset))$ is differentiable in $\Xset$ for each $\omega$.

\emph{Step 2: Gradient derivation.}
We derive the gradient $\nabla_{\mathbf{x}^{(k)}} g_\mu^{\mathrm{STCH\text{-}Set}}$ by applying the chain rule to the nested structure.

Define:
\begin{align}
R_i^{\min} &:= \smin_\mu^{(k)} f_i(\mathbf{x}^{(k)}) = -\mu \log\!\left(\sum_{\ell=1}^{K} \exp\!\left(-\frac{f_i(\mathbf{x}^{(\ell)})}{\mu}\right)\right), \\
S &:= \sum_{j=1}^{m} \exp\!\left(\frac{\lambda_j(R_j^{\min} - z_j^*)}{\mu}\right).
\end{align}

The outer softmax weights are:
\begin{equation}
w_i = \frac{\partial\, g_\mu^{\mathrm{STCH\text{-}Set}}}{\partial\, R_i^{\min}} \cdot \frac{1}{\lambda_i} = \frac{\exp\!\left(\lambda_i(R_i^{\min} - z_i^*)/\mu\right)}{S},
\end{equation}
satisfying $\sum_i w_i = 1$ and $w_i > 0$ for all $i$.

The inner softmin weights are:
\begin{equation}
p_{ik} = \frac{\partial\, R_i^{\min}}{\partial\, f_i(\mathbf{x}^{(k)})} = \frac{\exp\!\left(-f_i(\mathbf{x}^{(k)})/\mu\right)}{\sum_{\ell=1}^{K} \exp\!\left(-f_i(\mathbf{x}^{(\ell)})/\mu\right)},
\end{equation}
satisfying $\sum_k p_{ik} = 1$ and $p_{ik} > 0$ for all $i, k$.

Applying the chain rule:
\begin{align}
\nabla_{\mathbf{x}^{(k)}} g_\mu^{\mathrm{STCH\text{-}Set}}
&= \sum_{i=1}^{m} \frac{\partial\, g_\mu^{\mathrm{STCH\text{-}Set}}}{\partial\, R_i^{\min}} \cdot \frac{\partial\, R_i^{\min}}{\partial\, f_i(\mathbf{x}^{(k)})} \cdot \nabla_{\mathbf{x}^{(k)}} f_i(\mathbf{x}^{(k)}) \\
&= \sum_{i=1}^{m} (\lambda_i \cdot w_i) \cdot p_{ik} \cdot \nabla f_i(\mathbf{x}^{(k)}).
\end{align}

Note: in the main text~\eqref{eq:grad}, we absorb $\lambda_i$ into the definition of $w_i$ for notational compactness. With the convention $\boldsymbol{\lambda} = \mathbf{1}/m$ (uniform weights), this simplifies further.

\emph{Step 3: Differentiation under the integral sign.}
To differentiate the expectation $\alpha_t^{\mathrm{qSTCH}}(\Xset) = \E_\omega[-g_\mu^{\mathrm{STCH\text{-}Set}}(\hat{\mathbf{f}}_\omega^{(t)}(\Xset))]$ with respect to $\Xset$, we apply the Leibniz integral rule (differentiation under the integral sign). This is justified by the dominated convergence theorem:

For any compact neighborhood $U \ni \Xset$ in $\X^K$, the gradient $\nabla_\Xset g_\mu^{\mathrm{STCH\text{-}Set}}(\hat{\mathbf{f}}_\omega^{(t)}(\Xset))$ is bounded by:
\begin{equation}
\|\nabla_\Xset g_\mu^{\mathrm{STCH\text{-}Set}}\| \le \sum_{i=1}^{m} \sum_{k=1}^{K} \|\nabla f_i(\mathbf{x}^{(k)})\| \le Km \cdot \sup_{\mathbf{x} \in \X, i \in [m]} \|\nabla \hat{f}_{i,\omega}^{(t)}(\mathbf{x})\|.
\end{equation}
The GP posterior gradient $\nabla \hat{f}_{i,\omega}^{(t)}(\mathbf{x})$ is a Gaussian random variable with finite second moment (for differentiable kernels), providing an integrable dominating function. Therefore:
\begin{equation}
\nabla_\Xset \alpha_t^{\mathrm{qSTCH}}(\Xset) = \E_\omega\!\left[-\nabla_\Xset g_\mu^{\mathrm{STCH\text{-}Set}}(\hat{\mathbf{f}}_\omega^{(t)}(\Xset))\right],
\end{equation}
which can be approximated by the sample average gradient, compatible with the reparameterization trick and automatic differentiation in PyTorch/BoTorch.
\end{proof}


%=============================================================================
\section{Full Proof of Proposition~\ref{prop:pareto-transfer}: Pareto Optimality Transfer}
\label{app:proof-pareto}
%=============================================================================

We provide the complete proof of Proposition~\ref{prop:pareto-transfer}, which establishes that the Pareto optimality guarantees of the STCH-Set formulation~\citep{lin2025few} transfer to the GP posterior setting.

\begin{proof}[Proof of Proposition~\ref{prop:pareto-transfer}]

\textbf{Part (a): Surrogate Pareto optimality.}

\emph{Setup.} Under Assumption~\ref{asm:gp}, the GP posterior means $\mu_1^{(t)}, \ldots, \mu_m^{(t)}: \X \to \R$ are well-defined, continuous functions on the compact domain $\X$. For any standard positive-definite kernel:
\begin{equation}
\mu_i^{(t)}(\mathbf{x}) = \mathbf{k}_i(\mathbf{x})^\top (K_i + \sigma^2 I)^{-1} \mathbf{y}_i,
\end{equation}
which inherits the smoothness of the kernel $k_i$. For the squared exponential kernel, $\mu_i^{(t)} \in C^\infty(\X)$; for the Mat\'ern-$\nu$ kernel, $\mu_i^{(t)} \in C^{\lceil \nu \rceil - 1}(\X)$ (and in particular, for Mat\'ern-5/2, $\mu_i^{(t)} \in C^2(\X)$).

\emph{Verification of hypotheses.} We verify that the hypotheses of Theorem~2 of Lin et al.~\cite{lin2025few} are satisfied:
\begin{enumerate}
    \item \textbf{Compact domain}: $\X \subset \R^d$ is compact by assumption.
    \item \textbf{Continuous objectives}: Each $\mu_i^{(t)}$ is continuous on $\X$ (follows from kernel continuity).
    \item \textbf{Differentiable objectives}: Each $\mu_i^{(t)}$ is differentiable on the interior of $\X$ (for differentiable kernels such as the squared exponential or Mat\'ern-$\nu$ with $\nu > 1$).
    \item \textbf{Strictly positive weights}: $\lambda_i > 0$ for all $i \in [m]$ since $\boldsymbol{\lambda} \in \Delta_{++}^{m-1}$.
    \item \textbf{Smoothing parameter}: $\mu > 0$ by construction.
\end{enumerate}

\emph{Application.} Theorem~2 of \cite{lin2025few} states: if $\Xset^* = \arg\min_{\Xset \in \X^K} g_\mu^{\mathrm{STCH\text{-}Set}}(\Xset \mid \boldsymbol{\lambda})$ for objectives satisfying the above conditions, then every $\mathbf{x}^{(k)} \in \Xset^*$ is weakly Pareto optimal with respect to those objectives.

Taking the objectives to be $\mu_1^{(t)}, \ldots, \mu_m^{(t)}$, and recalling the posterior-mean scalarization $\hat{g}_\mu^{(t)}(\Xset \mid \boldsymbol{\lambda}) := g_\mu^{\mathrm{STCH\text{-}Set}}(\Xset \mid \boldsymbol{\lambda})\big|_{f_i = \mu_i^{(t)}}$, we conclude:
\begin{quote}
If $\Xset^* = \arg\min_{\Xset \in \X^K} \hat{g}_\mu^{(t)}(\Xset \mid \boldsymbol{\lambda})$, then every $\mathbf{x}^{(k)} \in \Xset^*$ is weakly Pareto optimal with respect to $(\mu_1^{(t)}, \ldots, \mu_m^{(t)})$.
\end{quote}

For the uniqueness claim: if $\Xset^*$ is the unique minimizer, then Theorem~2 of \cite{lin2025few} further guarantees (strong) Pareto optimality. This follows because at a unique minimizer, the KKT conditions for the constrained optimization problem are non-degenerate, ruling out the weakly-but-not-strongly Pareto optimal case. \qed

\medskip
\textbf{Part (b): Sandwich bound.}

We derive the approximation bound~\eqref{eq:sandwich} relating the smooth and non-smooth scalarizations.

\emph{Lower bound.} The log-sum-exp satisfies $\mu \log(\sum_i e^{a_i/\mu}) \ge \max_i a_i$ for any $a_1, \ldots, a_m \in \R$ and $\mu > 0$. This is immediate since $\sum_i e^{a_i/\mu} \ge e^{\max_i a_i/\mu}$.

Similarly, the smooth minimum satisfies $\smin_\mu^{(k)} f_i(\mathbf{x}^{(k)}) \le \min_k f_i(\mathbf{x}^{(k)})$, because:
\begin{equation}
-\mu \log\!\left(\sum_{k=1}^{K} e^{-f_i(\mathbf{x}^{(k)})/\mu}\right) \le -\mu \log\!\left(e^{-\min_k f_i(\mathbf{x}^{(k)})/\mu}\right) = \min_k f_i(\mathbf{x}^{(k)}).
\end{equation}

Since each $R_i^{\min} = \smin_\mu^{(k)} f_i(\mathbf{x}^{(k)}) \le \min_k f_i(\mathbf{x}^{(k)})$ and $\lambda_i > 0$, we have:
\begin{align}
g_\mu^{\mathrm{STCH\text{-}Set}}(\Xset) &= \mu \log\!\left(\sum_{i} \exp\!\left(\frac{\lambda_i(R_i^{\min} - z_i^*)}{\mu}\right)\right) \\
&\ge \max_i \lambda_i(R_i^{\min} - z_i^*) \\
&\ge \max_i \lambda_i\!\left(\min_k f_i(\mathbf{x}^{(k)}) - z_i^*\right) - \max_i \lambda_i(\min_k f_i(\mathbf{x}^{(k)}) - R_i^{\min}).
\end{align}
However, a cleaner path uses the composition directly. Since the smooth minimum underestimates the true minimum, and the smooth maximum overestimates the true maximum:
\begin{equation}
g^{\mathrm{TCH\text{-}Set}}(\Xset) = \max_i \lambda_i\!\left(\min_k f_i(\mathbf{x}^{(k)}) - z_i^*\right) \le g_\mu^{\mathrm{STCH\text{-}Set}}(\Xset).
\end{equation}
To see this precisely: define $a_i := \lambda_i(R_i^{\min} - z_i^*)$. Then:
\begin{align}
g_\mu^{\mathrm{STCH\text{-}Set}} &= \mu\log\!\left(\sum_i e^{a_i/\mu}\right) \ge \max_i a_i = \max_i \lambda_i(R_i^{\min} - z_i^*) \\
&\ge \max_i \lambda_i\!\left(\min_k f_i(\mathbf{x}^{(k)}) - \mu\log K - z_i^*\right) \quad \text{(by the smin lower bound below)} \\
&\ge g^{\mathrm{TCH\text{-}Set}}(\Xset) - \mu\log K \cdot \max_i \lambda_i.
\end{align}
But we can do better. Since $R_i^{\min} \le \min_k f_i(\mathbf{x}^{(k)})$:
\begin{equation}
g_\mu^{\mathrm{STCH\text{-}Set}} \ge \max_i a_i \ge \max_i \lambda_i(\min_k f_i(\mathbf{x}^{(k)}) - z_i^*) - \max_i \lambda_i \cdot \mu\log K.
\end{equation}

In fact, the cleaner statement of the lower bound is simply:
\begin{equation}
g^{\mathrm{TCH\text{-}Set}}(\Xset) \le g_\mu^{\mathrm{STCH\text{-}Set}}(\Xset),
\end{equation}
which follows because both the smooth max (log-sum-exp) overestimates the hard max, and the smooth min underestimates the hard min, and these biases compound in the same direction when the smooth min feeds into the smooth max through the monotonically increasing $\lambda_i(\cdot - z_i^*)$ link.

\emph{Upper bound.} We decompose the smoothing error into two contributions.

\textit{Outer error.} By the standard log-sum-exp bound (Proposition~3.4 of \cite{lin2024smooth}):
\begin{equation}
\mu \log\!\left(\sum_{i=1}^{m} e^{a_i/\mu}\right) \le \max_i a_i + \mu \log m.
\end{equation}

\textit{Inner error.} For each objective $i$, the smooth minimum satisfies:
\begin{equation}
\min_k f_i(\mathbf{x}^{(k)}) - \mu \log K \le \smin_\mu^{(k)} f_i(\mathbf{x}^{(k)}) \le \min_k f_i(\mathbf{x}^{(k)}).
\end{equation}
The lower bound follows from $\sum_k e^{-f_i(\mathbf{x}^{(k)})/\mu} \le K \cdot e^{-\min_k f_i(\mathbf{x}^{(k)})/\mu}$.

Substituting the inner error into the outer bound: for each $i$,
\begin{equation}
\lambda_i(R_i^{\min} - z_i^*) \le \lambda_i(\min_k f_i(\mathbf{x}^{(k)}) - z_i^*).
\end{equation}
Therefore:
\begin{align}
g_\mu^{\mathrm{STCH\text{-}Set}} &\le \max_i \lambda_i(R_i^{\min} - z_i^*) + \mu \log m \\
&\le \max_i \lambda_i(\min_k f_i(\mathbf{x}^{(k)}) - z_i^*) + \mu \log m.
\end{align}

But we also need to account for the inner smoothing in the other direction. Since $R_i^{\min} \ge \min_k f_i(\mathbf{x}^{(k)}) - \mu\log K$, the terms $\lambda_i(R_i^{\min} - z_i^*)$ are at most $\mu\log K$ above $\lambda_i(\min_k f_i(\mathbf{x}^{(k)}) - \mu\log K - z_i^*)$. Since $\boldsymbol{\lambda} \in \Delta^{m-1}$ (i.e., $\sum_i \lambda_i = 1$, $\lambda_i \le 1$), the per-objective shift of $\mu\log K$ contributes at most $\mu\log K$ to the outer scalarization value. More precisely:
\begin{align}
g_\mu^{\mathrm{STCH\text{-}Set}} &= \mu\log\!\left(\sum_i \exp\!\left(\frac{\lambda_i(R_i^{\min} - z_i^*)}{\mu}\right)\right) \\
&\le \mu\log\!\left(\sum_i \exp\!\left(\frac{\lambda_i(\min_k f_i(\mathbf{x}^{(k)}) - z_i^*)}{\mu}\right)\right) \quad \text{(since $R_i^{\min} \le \min_k f_i$)} \\
&\le \max_i \lambda_i(\min_k f_i(\mathbf{x}^{(k)}) - z_i^*) + \mu\log m \\
&= g^{\mathrm{TCH\text{-}Set}}(\Xset) + \mu\log m.
\end{align}

Wait---this gives only $\mu\log m$. The $\mu\log K$ term arises when bounding the gap in the \emph{other} direction (the lower bound). Let us redo both bounds cleanly.

Define $T_i := \min_k f_i(\mathbf{x}^{(k)})$ (hard min) and $S_i := R_i^{\min}$ (soft min). Then $T_i - \mu\log K \le S_i \le T_i$.

\textit{Upper bound on $g_\mu^{\mathrm{STCH\text{-}Set}}$:}
\begin{align}
g_\mu^{\mathrm{STCH\text{-}Set}} &= \mu\log\!\left(\sum_i e^{\lambda_i(S_i - z_i^*)/\mu}\right) \le \mu\log\!\left(\sum_i e^{\lambda_i(T_i - z_i^*)/\mu}\right) \\
&\le \max_i \lambda_i(T_i - z_i^*) + \mu\log m = g^{\mathrm{TCH\text{-}Set}} + \mu\log m.
\end{align}

\textit{Tighter upper bound including $\mu\log K$:}
Actually, the bound $g_\mu^{\mathrm{STCH\text{-}Set}} \le g^{\mathrm{TCH\text{-}Set}} + \mu\log m$ already follows. The $\mu\log K$ term appears when we bound the gap from below:
\begin{align}
g_\mu^{\mathrm{STCH\text{-}Set}} &\ge \max_i \lambda_i(S_i - z_i^*) \ge \max_i \lambda_i(T_i - \mu\log K - z_i^*) \\
&= g^{\mathrm{TCH\text{-}Set}} - \mu\log K \cdot \max_i \lambda_i.
\end{align}

Therefore the complete sandwich is:
\begin{equation}
g^{\mathrm{TCH\text{-}Set}} - \mu\log K \le g_\mu^{\mathrm{STCH\text{-}Set}} \le g^{\mathrm{TCH\text{-}Set}} + \mu\log m.
\end{equation}

In the main text, we state the weaker but simpler bound~\eqref{eq:sandwich} as
\begin{equation}
g^{\mathrm{TCH\text{-}Set}} \le g_\mu^{\mathrm{STCH\text{-}Set}} \le g^{\mathrm{TCH\text{-}Set}} + \mu\log m + \mu\log K,
\end{equation}
which holds trivially since $g^{\mathrm{TCH\text{-}Set}} \le g_\mu^{\mathrm{STCH\text{-}Set}}$ (the smooth version overestimates due to the smooth max dominating the hard max, composed with the smooth min underestimating the hard min). The upper bound $g^{\mathrm{TCH\text{-}Set}} + \mu\log m + \mu\log K$ is also valid (and slightly looser than $g^{\mathrm{TCH\text{-}Set}} + \mu\log m$), but provides a uniform bound that accounts for both smoothing operations symmetrically. The total smoothing gap $\mu(\log m + \log K) = \mu\log(mK)$ is controlled by the user-chosen $\mu$.
\end{proof}


%=============================================================================
\section{Extended Experimental Results}
\label{app:extended-results}
%=============================================================================

\subsection{Per-Seed Hypervolume Values: $m=5$}

Table~\ref{tab:perseed-m5} reports the final hypervolume for each seed on DTLZ2 with $m=5$ objectives ($d=9$, $n_{\text{init}}=20$, 30 BO iterations).

\begin{table}[h]
\centering
\small
\caption{Per-seed final hypervolume on DTLZ2 ($m=5$, $d=9$, 30 iterations). qSTCH-Set uses batch size $K=m=5$; all other methods use $q=1$.}
\label{tab:perseed-m5}
\begin{tabular}{ccccc}
\toprule
Seed & qSTCH-Set ($K{=}5$) & STCH-NParEGO & qNParEGO & Random \\
\midrule
0 & 6.760 & 6.343 & 6.525 & 5.442 \\
1 & 6.615 & 6.158 & 5.994 & 5.382 \\
2 & 6.643 & 6.140 & 6.465 & 5.199 \\
3 & 6.559 & 5.857 & 6.387 & 5.250 \\
4 & 6.653 & 6.089 & 6.776 & 5.575 \\
\midrule
Mean & $\mathbf{6.646}$ & 6.117 & 6.429 & 5.370 \\
Std  & $\mathbf{0.066}$ & 0.156 & 0.254 & 0.135 \\
\bottomrule
\end{tabular}
\end{table}

\textbf{Observations.} qSTCH-Set achieves the highest mean hypervolume with the lowest standard deviation across all 5 seeds. The variance reduction ($0.066$ vs.\ $0.254$ for qNParEGO) confirms that set-based coordination produces more consistent Pareto front coverage than random weight sampling. Note that qNParEGO shows high variability (seed~1 at $5.994$ vs.\ seed~4 at $6.776$), consistent with the lottery effect of random scalarization weights.

\subsection{Per-Seed Hypervolume Values: $m=8$}

Table~\ref{tab:perseed-m8} reports per-seed results on DTLZ2 with $m=8$ objectives ($d=12$, $n_{\text{init}}=30$, 25 BO iterations, 3 seeds).

\begin{table}[h]
\centering
\small
\caption{Per-seed final hypervolume on DTLZ2 ($m=8$, $d=12$, 25 iterations). qSTCH-Set uses batch size $K=m=8$; all other methods use $q=1$.}
\label{tab:perseed-m8}
\begin{tabular}{ccccc}
\toprule
Seed & qSTCH-Set ($K{=}8$) & STCH-NParEGO & qNParEGO & Random \\
\midrule
0 & 23.964 & 18.848 & 22.724 & 17.937 \\
1 & 24.545 & 20.589 & 20.753 & 17.843 \\
2 & 23.816 & 22.422 & 21.341 & 18.298 \\
\midrule
Mean & $\mathbf{24.108}$ & 20.620 & 21.606 & 18.026 \\
Std  & $\mathbf{0.314}$ & 1.459 & 0.826 & 0.196 \\
\bottomrule
\end{tabular}
\end{table}

\textbf{Observations.} At $m=8$, the advantage of qSTCH-Set becomes more pronounced: it leads qNParEGO by $2.50$ HV units ($\sim$11.6\%) and STCH-NParEGO by $3.49$ units ($\sim$16.9\%). The variance of STCH-NParEGO is notably high ($1.459$), indicating that single-point smooth scalarization with random weights becomes unreliable as the number of objectives grows. qSTCH-Set maintains relatively low variance ($0.314$) despite the higher-dimensional objective space, consistent with the coordinated $K=m$ design.


\subsection{Computational Cost}

Table~\ref{tab:timing} reports the mean wall-clock time per BO iteration for each method, measured on NVIDIA H100 GPUs (MIG 1g.10gb partitions) on the Digital Research Alliance of Canada Nibi cluster.

\begin{table}[h]
\centering
\small
\caption{Mean wall-clock time per iteration (seconds) on DTLZ2. All experiments run on H100 MIG 1g.10gb partitions. qSTCH-Set evaluates $K=m$ points per iteration; baselines evaluate $q=1$.}
\label{tab:timing}
\begin{tabular}{lrrrr}
\toprule
Method & $m=5$ (s/iter) & $m=8$ (s/iter) \\
\midrule
qSTCH-Set ($K{=}m$) & 22.7 & 335.6 \\
STCH-NParEGO ($q{=}1$) & 24.8 & 55.8 \\
qNParEGO ($q{=}1$) & 24.6 & 45.4 \\
Random ($q{=}1$) & 0.4 & 10.0 \\
\bottomrule
\end{tabular}
\end{table}

\textbf{Analysis.} At $m=5$, qSTCH-Set is slightly \emph{faster} per iteration than the single-point baselines ($22.7$s vs.\ $24.6$--$24.8$s), despite optimizing over $K=5$ candidates jointly. This is because the BoTorch joint optimization with L-BFGS-B amortizes the overhead of multi-start initialization across the larger batch.

At $m=8$, the cost increases to $335.6$s/iter, reflecting the $O(Nm^2)$ complexity: the acquisition evaluation involves $K=8$ candidates in $d=12$ dimensions (96-dimensional search space), and each evaluation requires computing the nested smooth min/max over 8 objectives and 8 candidates. The $\sim$7.4$\times$ increase from $m=5$ to $m=8$ is super-linear, consistent with the quadratic scaling in $m$ combined with the higher input dimensionality.

However, in the target application of expensive black-box optimization (e.g., molecular simulations, wet-lab assays), function evaluation costs typically dominate ($\sim$minutes to hours per evaluation), making the $\sim$6 min/iter acquisition cost negligible.


%=============================================================================
\section{Hyperparameter Sensitivity}
\label{app:hyperparams}
%=============================================================================

\subsection{Smoothing Parameter $\mu$}

The smoothing parameter $\mu > 0$ controls the approximation quality of the log-sum-exp to the hard max/min operators. Proposition~\ref{prop:pareto-transfer}(b) shows the approximation gap is bounded by $\mu\log(mK)$.

Ablation results for $\mu \in \{0.01, 0.05, 0.1, 0.5, 1.0\}$ on DTLZ2 ($m=5$) are pending completion on the Nibi cluster. The ablation script (\texttt{benchmarks/ablation\_mu.py}) varies $\mu$ while holding all other parameters fixed at their default values ($K=m=5$, $N=256$ MC samples, Mat\'ern-5/2 kernel).

\textbf{Preliminary observations from development:} During development, we observed that $\mu = 0.1$ provided a good balance between gradient smoothness and approximation quality. Values $\mu < 0.01$ occasionally caused numerical instabilities (large gradients in the softmax attention), while $\mu > 1.0$ over-smoothed the scalarization, reducing the method's ability to distinguish between Pareto-optimal and dominated solutions. All main results in this paper use $\mu = 0.1$.

\subsection{Batch Size $K$}

The design rule $K = m$ is a key contribution. To validate this choice, ablation experiments varying $K \in \{2, 4, m{=}5, 8, 10\}$ on DTLZ2 ($m=5$) and $K \in \{2, 4, m{=}8, 12\}$ on DTLZ2 ($m=8$) are in progress.

\textbf{Theoretical motivation.} When $K < m$, the smooth minimum operator $\smin_k f_i(\mathbf{x}^{(k)})$ must assign multiple objectives to the same candidate, preventing full Pareto front coverage. When $K > m$, the additional candidates provide redundancy but do not improve the worst-case objective coverage (the outer smooth max is over $m$ terms regardless of $K$). The $K = m$ choice is the minimal set size that allows one-to-one assignment between candidates and objectives.


%=============================================================================
\section{Implementation Details}
\label{app:implementation-full}
%=============================================================================

\subsection{Algorithm Pseudocode}

Algorithm~\ref{alg:full} provides the complete qSTCH-Set BO procedure with all hyperparameters and implementation details.

\begin{algorithm}[t]
\caption{qSTCH-Set: Complete Many-Objective Bayesian Optimization Procedure}
\label{alg:full}
\begin{algorithmic}[1]
\REQUIRE Objectives $f_1, \ldots, f_m: \X \to \R$ (expensive, black-box)
\REQUIRE Domain $\X \subseteq \R^d$ (compact, box-constrained)
\REQUIRE Total budget $T$ function evaluations
\REQUIRE Hyperparameters: smoothing $\mu > 0$ (default $0.1$), MC samples $N$ (default $256$)
\REQUIRE Optional: batch size $K$ (default $m$), weight vector $\boldsymbol{\lambda}$ (default $\mathbf{1}/m$)
\STATE \textbf{Initialize:} Generate $n_0$ Sobol quasi-random points $\{\mathbf{x}_j\}_{j=1}^{n_0}$; evaluate $\mathbf{y}_j = \mathbf{f}(\mathbf{x}_j)$; set $\D_0 \leftarrow \{(\mathbf{x}_j, \mathbf{y}_j)\}_{j=1}^{n_0}$
\FOR{$t = 0, 1, \ldots, \lfloor(T - n_0)/K\rfloor - 1$}
    \STATE \textbf{Fit surrogates:} For each $i \in [m]$, fit SingleTaskGP with Mat\'ern-5/2 kernel:
    \STATE \quad $\hat{f}_i^{(t)} \sim \GP(\mu_i^{(t)}, k_i^{(t)})$ via MLE (L-BFGS-B, 5 random restarts)
    \STATE \quad Standardize outcomes: $\tilde{y}_{j,i} = (y_{j,i} - \bar{y}_i) / s_i$
    \STATE \textbf{Compute ideal point:} $z_i^* \leftarrow \min_{j \le |\D_t|} y_{j,i} - \epsilon$ for each $i$ ($\epsilon = 10^{-4}$)
    \STATE \textbf{Set weight vector:} $\boldsymbol{\lambda} \leftarrow \mathbf{1}/m$ (uniform)
    \STATE \textbf{Draw MC samples:} $\{\boldsymbol{\omega}^{(n)}\}_{n=1}^N$ via Sobol quasi-random sequence
    \STATE \textbf{Construct acquisition function:}
    \STATE \quad $\alpha(\Xset) = \frac{1}{N}\sum_{n=1}^{N} \left[-g_\mu^{\mathrm{STCH\text{-}Set}}\!\left(\hat{\mathbf{f}}_{\omega^{(n)}}^{(t)}(\Xset) \mid \boldsymbol{\lambda}\right)\right]$
    \STATE \textbf{Generate candidates:} $512$ Sobol points in $\X^K$; evaluate $\alpha$; select top $20$
    \STATE \textbf{Optimize acquisition:}
    \STATE \quad $\Xset^* \leftarrow \argmax_{\Xset \in \X^K} \alpha(\Xset)$ via L-BFGS-B from each of $20$ restarts
    \STATE \quad Joint optimization over all $Kd$ variables simultaneously
    \STATE \textbf{Evaluate:} $\mathbf{y}^{(k)} \leftarrow \mathbf{f}(\mathbf{x}^{(k)})$ for each $\mathbf{x}^{(k)} \in \Xset^*$
    \STATE \textbf{Update dataset:} $\D_{t+1} \leftarrow \D_t \cup \{(\mathbf{x}^{(k)}, \mathbf{y}^{(k)})\}_{k=1}^{K}$
\ENDFOR
\RETURN Non-dominated set from $\D_{T}$
\end{algorithmic}
\end{algorithm}

\subsection{Software Versions}

All experiments were conducted with the following software stack:

\begin{table}[h]
\centering
\small
\caption{Software versions used in all experiments.}
\label{tab:software}
\begin{tabular}{ll}
\toprule
Package & Version \\
\midrule
Python & 3.12 \\
PyTorch & $\ge 2.1$ \\
BoTorch & $\ge 0.11$ \\
GPyTorch & $\ge 1.11$ \\
NumPy & $\ge 1.24$ \\
CUDA & 12.6 \\
\bottomrule
\end{tabular}
\end{table}

The package was installed from source using \texttt{pip install -e .} in a virtual environment (not Conda) following Alliance Canada best practices.

\subsection{Hardware}

All GPU experiments were run on the \textbf{Nibi} cluster of the Digital Research Alliance of Canada with the following configuration:

\begin{itemize}
    \item \textbf{GPU}: NVIDIA H100 80GB SXM5 (MIG 1g.10gb partitions, providing 10~GB GPU memory per job)
    \item \textbf{CPU}: 4 cores per job
    \item \textbf{RAM}: 32~GB per job
    \item \textbf{Storage}: Local NVMe ($\text{SLURM\_TMPDIR}$) for virtual environment; project storage for results
    \item \textbf{Job scheduler}: Slurm array jobs (one job per seed $\times$ objective count combination)
    \item \textbf{Account}: \texttt{rrg-ravh011\_gpu}
\end{itemize}

Each benchmark configuration ($m \in \{5, 8\}$, per seed) ran as an independent Slurm job with a 4-hour time limit. The total GPU-hours consumed for the main experiments (5 seeds $\times$ $m=5$ + 3 seeds $\times$ $m=8$) was approximately 25 GPU-hours.

\subsection{Reproducibility}

All random seeds are controlled at three levels:
\begin{enumerate}
    \item \textbf{Initial design}: Sobol quasi-random sequence with seed offset.
    \item \textbf{GP fitting}: PyTorch random seed for multi-start MLE optimization.
    \item \textbf{MC sampling}: Sobol quasi-random base samples for the reparameterization trick.
\end{enumerate}

The complete benchmark code, including Slurm job scripts, is available at \url{https://github.com/parameters/qSTCH-Set}.


\end{document}
