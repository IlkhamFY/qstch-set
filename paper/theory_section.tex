%=============================================================================
% THEORY SECTION — Hardened version with rigorous propositions
% Drop into main.tex as \section{Theoretical Analysis}
% Requires: amsmath, amssymb, amsthm (or neurips theorem environments)
%
% COMPILE NOTE: Ensure the following are in the preamble of main.tex:
%   \newtheorem{proposition}{Proposition}
%   \newtheorem{corollary}{Corollary}
%   \newtheorem{conjecture}{Conjecture}
%   \newtheorem{remark}{Remark}
%   \theoremstyle{definition}
%   \newtheorem{definition}{Definition}
%   \newtheorem{assumption}{Assumption}
%=============================================================================

\section{Theoretical Analysis}
\label{sec:theory}

We establish the theoretical properties of qSTCH-Set as a BO acquisition function.
We distinguish sharply between results that follow rigorously from existing theory
(Propositions~\ref{prop:valid}--\ref{prop:pareto-transfer}),
and those that remain conjectural (Conjecture~\ref{conj:consistency}).
Throughout, $\X \subset \R^d$ is compact and $f_1,\ldots,f_m:\X\to\R$ are the true (unknown) objectives.

%--------------------------------------------------------------------
\subsection{Definitions}
\label{sec:theory-defs}
%--------------------------------------------------------------------

\begin{definition}[Pareto optimality]
\label{def:pareto}
A point $\mathbf{x}\in\X$ is \emph{weakly Pareto optimal} if there is no $\mathbf{x}'\in\X$
with $f_i(\mathbf{x}') < f_i(\mathbf{x})$ for every $i\in[m]$.
It is \emph{Pareto optimal} if there is no $\mathbf{x}'\in\X$
with $f_i(\mathbf{x}') \le f_i(\mathbf{x})$ for all $i$ and $f_j(\mathbf{x}') < f_j(\mathbf{x})$
for some $j$.
\end{definition}

\begin{definition}[$\varepsilon$-Pareto optimality]
\label{def:eps-pareto}
A point $\mathbf{x}\in\X$ is \emph{$\varepsilon$-Pareto optimal} (for $\varepsilon\ge 0$)
if there is no $\mathbf{x}'\in\X$ with $f_i(\mathbf{x}') \le f_i(\mathbf{x}) - \varepsilon$ for every $i\in[m]$.
\end{definition}

\begin{definition}[Smooth Tchebycheff Set scalarization]
\label{def:stchset}
For a candidate set $\Xset = \{\mathbf{x}^{(1)},\ldots,\mathbf{x}^{(K)}\}\subset\X$,
preference vector $\boldsymbol{\lambda}\in\Delta^{m-1}_{++}$
($\lambda_i>0$, $\sum_i \lambda_i=1$),
reference point $\mathbf{z}^*\in\R^m$, and smoothing parameter $\mu>0$:
\begin{align}
g_\mu^{\mathrm{STCH\text{-}Set}}(\Xset\mid\boldsymbol{\lambda})
&= \mu\log\!\left(\sum_{i=1}^{m}\exp\!\left(
    \frac{\lambda_i\bigl(\smin_\mu^{(k)} f_i(\mathbf{x}^{(k)}) - z_i^*\bigr)}{\mu}
\right)\right), \label{eq:stchset-def}\\
\text{where}\quad
\smin_\mu^{(k)} f_i(\mathbf{x}^{(k)})
&= -\mu\log\!\left(\sum_{k=1}^{K}\exp\!\left(-\frac{f_i(\mathbf{x}^{(k)})}{\mu}\right)\right).
\label{eq:smin-def}
\end{align}
The outer log-sum-exp is a smooth approximation to $\max_{i}$; the inner negative
log-sum-exp of negatives is a smooth approximation to $\min_{k}$.
\end{definition}

\begin{definition}[qSTCH-Set acquisition function]
\label{def:qstchset}
Given GP posteriors $\hat{f}_1^{(t)},\ldots,\hat{f}_m^{(t)}$ after $t$ observations,
the \emph{qSTCH-Set acquisition function} for a candidate set $\Xset$ is:
\begin{equation}
\label{eq:acq}
\alpha_t^{\mathrm{qSTCH}}(\Xset)
= \E_{\boldsymbol{\omega}}\!\left[
    -g_\mu^{\mathrm{STCH\text{-}Set}}\!\left(
        \hat{\mathbf{f}}_\omega^{(t)}(\Xset)\mid\boldsymbol{\lambda}
    \right)
\right],
\end{equation}
where $\hat{\mathbf{f}}_\omega^{(t)}$ denotes a joint posterior sample path (indexed by
base sample $\boldsymbol{\omega}$), obtained via the reparameterization trick.
In practice, the expectation is approximated by $N$ quasi-Monte Carlo base samples.
\end{definition}

%--------------------------------------------------------------------
\subsection{Proposition 1: qSTCH-Set Is a Valid MC Acquisition Function}
\label{sec:theory-prop1}
%--------------------------------------------------------------------

\begin{proposition}[Validity as MC acquisition function]
\label{prop:valid}
The qSTCH-Set acquisition function $\alpha_t^{\mathrm{qSTCH}}(\Xset)$
defined in~\eqref{eq:acq} satisfies the following:
\begin{enumerate}
    \item[\textnormal{(a)}] \textbf{Measurability.}
        For every fixed $\Xset\in\X^K$, the integrand
        $\omega\mapsto -g_\mu^{\mathrm{STCH\text{-}Set}}(\hat{\mathbf{f}}_\omega^{(t)}(\Xset)\mid\boldsymbol{\lambda})$
        is a Borel-measurable function of the base samples~$\boldsymbol{\omega}$.
    \item[\textnormal{(b)}] \textbf{Finite expectation.}
        If the GP posteriors have bounded support on any compact $\Xset\subset\X^K$
        (which holds almost surely for GPs with continuous kernels on compact domains),
        then $\E[|\alpha_t^{\mathrm{qSTCH}}(\Xset)|]<\infty$.
    \item[\textnormal{(c)}] \textbf{Differentiability.}
        $\alpha_t^{\mathrm{qSTCH}}(\Xset)$ is differentiable with respect to
        $\Xset = (\mathbf{x}^{(1)},\ldots,\mathbf{x}^{(K)}) \in \X^K$,
        with gradients computable via the reparameterization trick and automatic differentiation.
        Specifically, for each $\mathbf{x}^{(k)}$:
        \begin{equation}
        \label{eq:grad}
        \nabla_{\mathbf{x}^{(k)}}\, g_\mu^{\mathrm{STCH\text{-}Set}}
        = \sum_{i=1}^m w_i \cdot p_{ik}\cdot \nabla f_i(\mathbf{x}^{(k)}),
        \end{equation}
        where the softmax attention weights are
        \begin{equation}
        \label{eq:weights}
        w_i = \frac{\exp\!\bigl(\lambda_i(R_i^{\min}-z_i^*)/\mu\bigr)}
              {\sum_j \exp\!\bigl(\lambda_j(R_j^{\min}-z_j^*)/\mu\bigr)},\quad
        p_{ik} = \frac{\exp\!\bigl(-f_i(\mathbf{x}^{(k)})/\mu\bigr)}
              {\sum_\ell \exp\!\bigl(-f_i(\mathbf{x}^{(\ell)})/\mu\bigr)},
        \end{equation}
        with $R_i^{\min} = \smin_\mu^{(k)} f_i(\mathbf{x}^{(k)})$.
\end{enumerate}
\end{proposition}

\begin{proof}
\textbf{(a)}
The GP posterior sample paths $\hat{f}_{i,\omega}^{(t)}(\mathbf{x})$ are constructed
via the reparameterization trick as $\hat{f}_{i,\omega}^{(t)}(\mathbf{x})
= \mu_i^{(t)}(\mathbf{x}) + L_i^{(t)}(\mathbf{x})\,\omega_i$,
where $L_i^{(t)}$ is obtained from the Cholesky decomposition of the posterior covariance
(or, in the pathwise conditioning approach~\cite{wilson2018maxvalue},
from a basis-function representation).
In either case, for fixed $\mathbf{x}$, the map
$\omega_i\mapsto \hat{f}_{i,\omega}^{(t)}(\mathbf{x})$ is an affine function
of standard normal random variables, hence Borel-measurable.
The STCH-Set scalarization~\eqref{eq:stchset-def} is a composition of
$\exp$, $\log$, $\sum$, and scalar multiplication applied to these measurable functions.
Since compositions and finite sums of measurable functions are measurable,
the integrand is Borel-measurable.

\textbf{(b)}
On the compact set $\X$, any GP with a continuous kernel
$k_i:\X\times\X\to\R$ has sample paths that are almost surely
continuous, and hence bounded on $\X$ (continuous functions on compact sets
are bounded; see~\cite{rasmussen2006gp}, \S4.1).
Let $M = \max_{i,k,\omega}|\hat{f}_{i,\omega}^{(t)}(\mathbf{x}^{(k)})|<\infty$ a.s.
Then $|g_\mu^{\mathrm{STCH\text{-}Set}}|\le |M| + |z_{\max}^*| + \mu\log m$,
where $z_{\max}^* = \max_i|z_i^*|$. Hence the expectation is finite.

\textbf{(c)}
The STCH-Set scalarization~\eqref{eq:stchset-def} is a composition
of $C^\infty$ functions ($\exp$, $\log$, affine maps) with strictly positive arguments
inside the $\log$ (since each summand $\exp(\cdot)>0$). By the chain rule,
$g_\mu^{\mathrm{STCH\text{-}Set}}$ is $C^\infty$ with respect to its arguments
$\{f_i(\mathbf{x}^{(k)})\}_{i,k}$.

The gradient~\eqref{eq:grad} follows by direct computation.
Applying the chain rule to the outer log-sum-exp yields the
softmax weights $w_i$ (attention over objectives). Applying the chain rule
to the inner negative log-sum-exp yields the softmin weights $p_{ik}$
(attention over candidates for each objective).
The product $w_i \cdot p_{ik}$ determines how strongly objective $i$
influences the movement of candidate $\mathbf{x}^{(k)}$.

Since the reparameterization trick
expresses $\hat{f}_{i,\omega}^{(t)}(\mathbf{x})$ as a differentiable function
of $\mathbf{x}$ (for kernels with differentiable mean and covariance functions,
such as Mat\'ern with $\nu>1$ or the squared exponential), the
composition $g_\mu^{\mathrm{STCH\text{-}Set}}(\hat{\mathbf{f}}_\omega^{(t)}(\Xset))$
is differentiable in $\Xset$ for each $\omega$, and the expectation~\eqref{eq:acq}
can be differentiated under the integral sign
(by dominated convergence, using the a.s.\ boundedness from part~(b)
and the Lipschitz continuity of the gradient).
\end{proof}

\begin{remark}[Comparison to non-smooth Chebyshev]
The classical Tchebycheff scalarization~\eqref{eq:tch} uses a hard $\max$,
which is non-differentiable when two or more objectives tie.
This prevents direct use of the reparameterization trick for MC gradient estimation
in BoTorch.
The STCH-Set formulation resolves this entirely: $\alpha_t^{\mathrm{qSTCH}}$
is $C^\infty$ for all $\mu>0$, enabling standard L-BFGS-B acquisition optimization.
This is in contrast to qNParEGO~\cite{daulton2020qnehvi,knowles2006parego},
which uses the augmented Chebyshev scalarization with hard $\max$
and requires heuristic smoothing or restart strategies.
\end{remark}

%--------------------------------------------------------------------
\subsection{Proposition 2: Pareto Optimality of Acquisition Maximizers}
\label{sec:theory-prop2}
%--------------------------------------------------------------------

The following result transfers the Pareto optimality guarantee of Lin et al.~\cite{lin2025few}
from the gradient-based setting to the BO posterior.

\begin{assumption}[GP surrogate regularity]
\label{asm:gp}
Each $f_i$ is modeled by an independent GP with a continuous positive-definite kernel
$k_i$ on the compact domain $\X\subset\R^d$.
The posterior mean $\mu_i^{(t)}$ and variance $(\sigma_i^{(t)})^2$ are computed
from $t$ (possibly noisy) observations.
\end{assumption}

\begin{proposition}[Pareto optimality under the posterior]
\label{prop:pareto-transfer}
Let Assumption~\ref{asm:gp} hold, and let $\boldsymbol{\lambda}\in\Delta^{m-1}_{++}$.
Define the \emph{posterior-mean STCH-Set scalarization}:
\begin{equation}
\label{eq:posterior-stchset}
\hat{g}_\mu^{(t)}(\Xset\mid\boldsymbol{\lambda})
:= g_\mu^{\mathrm{STCH\text{-}Set}}(\Xset\mid\boldsymbol{\lambda})\big|_{f_i=\mu_i^{(t)}}.
\end{equation}
Then:
\begin{enumerate}
    \item[\textnormal{(a)}]
    \textbf{Surrogate Pareto optimality.}
    If $\Xset^*=\arg\min_{\Xset\in\X^K}\hat{g}_\mu^{(t)}(\Xset\mid\boldsymbol{\lambda})$,
    then every $\mathbf{x}^{(k)}\in\Xset^*$ is weakly Pareto optimal
    with respect to the posterior means $(\mu_1^{(t)},\ldots,\mu_m^{(t)})$.
    If additionally $\Xset^*$ is unique, every $\mathbf{x}^{(k)}\in\Xset^*$
    is Pareto optimal w.r.t.\ $(\mu_1^{(t)},\ldots,\mu_m^{(t)})$.

    \item[\textnormal{(b)}]
    \textbf{Approximation bound.}
    The smooth scalarization satisfies the sandwich inequality:
    \begin{equation}
    \label{eq:sandwich}
    g^{\mathrm{TCH\text{-}Set}}(\Xset\mid\boldsymbol{\lambda})
    \;\le\;
    g_\mu^{\mathrm{STCH\text{-}Set}}(\Xset\mid\boldsymbol{\lambda})
    \;\le\;
    g^{\mathrm{TCH\text{-}Set}}(\Xset\mid\boldsymbol{\lambda}) + \mu\log m + \mu\log K.
    \end{equation}
    The smoothing gap $\mu\log(m) + \mu\log(K)$ is uniform over $\Xset$
    and controlled by the user-chosen~$\mu$.
\end{enumerate}
\end{proposition}

\begin{proof}
\textbf{(a)}
The GP posterior means $\mu_1^{(t)},\ldots,\mu_m^{(t)}$ are $C^\infty$ functions
$\X\to\R$ (for any standard kernel with $C^\infty$ realizations, such as the
squared exponential; for Mat\'ern-$\nu$, $\mu_i^{(t)}$ is at least $C^{\lceil\nu\rceil}$).
In particular, they satisfy the differentiability requirements of
Theorem~2 in~\cite{lin2025few}.
Since $\lambda_i>0$ for all $i$ (by the choice $\boldsymbol{\lambda}\in\Delta_{++}^{m-1}$),
all hypotheses of~\cite[Theorem~2]{lin2025few} are satisfied with
the ``objectives'' taken to be $\mu_1^{(t)},\ldots,\mu_m^{(t)}$.
The conclusion follows directly: all solutions in $\Xset^*$ are weakly Pareto optimal
w.r.t.\ the surrogate objectives, and Pareto optimal when $\Xset^*$ is unique.

\textbf{(b)}
The lower bound follows because log-sum-exp
upper-bounds the max: for any $a_1,\ldots,a_m\in\R$ and $\mu>0$,
$\max_i a_i \le \mu\log(\sum_i e^{a_i/\mu})$.

For the upper bound, we decompose the smoothing error into two terms:
\begin{itemize}
\item \emph{Outer (smax vs.\ max):}
$\mu\log(\sum_i e^{a_i/\mu}) \le \max_i a_i + \mu\log m$.
This is the standard log-sum-exp bound (Proposition~3.4 of~\cite{lin2024smooth}).
\item \emph{Inner (smin vs.\ min):}
For each $i$, $\min_k b_k - \mu\log K \le \smin_\mu^{(k)} b_k \le \min_k b_k$.
This follows from the analogous bound for the smooth minimum:
$-\mu\log(\sum_k e^{-b_k/\mu}) \le \min_k b_k$ (immediate) and
$\min_k b_k - \mu\log K \le -\mu\log(\sum_k e^{-b_k/\mu})$
(since $\sum_k e^{-b_k/\mu} \le K \cdot e^{-\min_k b_k/\mu}$).
\end{itemize}
The inner error contributes at most $\mu\log K$ to each term inside the outer
log-sum-exp. Since $\boldsymbol{\lambda}\in\Delta^{m-1}$ (i.e., $\sum_i\lambda_i=1$),
the per-objective shift of at most $\mu\log K$ translates to an additive error
of at most $\mu\log K$ in the outer scalarization.
Combining: $g_\mu^{\mathrm{STCH\text{-}Set}} \le g^{\mathrm{TCH\text{-}Set}} + \mu\log m + \mu\log K$.
\end{proof}

\begin{corollary}[Asymptotic quality with $\mu$-annealing]
\label{cor:annealing}
If the smoothing parameter is reduced as $\mu_t = c/\log(t+1)$ for a constant $c>0$,
then the smoothing gap satisfies
$\mu_t\log(mK)\to 0$ as $t\to\infty$.
Thus, assuming part~(a) continues to hold (i.e., the STCH-Set optimization finds a
global minimizer), the solutions converge in scalarization value to the true
TCH-Set optimum.
\end{corollary}

\begin{remark}[Scope of Proposition~\ref{prop:pareto-transfer}]
\label{rem:scope}
Part~(a) is a \emph{surrogate-space} guarantee: it says the acquisition function
searches in the right part of the Pareto front of the \emph{model}.
It does \emph{not} directly imply Pareto optimality of the true objectives $\mathbf{f}$,
because the GP posterior mean $\mu_i^{(t)}$ may differ from $f_i$.
Bridging the gap requires posterior convergence, which we address in
Conjecture~\ref{conj:consistency} below.
Note that this surrogate-space guarantee is the same type of guarantee
enjoyed by all model-based BO methods (e.g., qEHVI optimizes hypervolume
of the \emph{posterior}, not the true Pareto front).
\end{remark}

%--------------------------------------------------------------------
\subsection{Conjecture 3: Consistency Under GP Posterior Convergence}
\label{sec:theory-conj}
%--------------------------------------------------------------------

We now state the key question: \emph{Does qSTCH-Set BO converge to the true Pareto front
as the number of observations grows?}
For the $K=1$ case, this reduces to composite BO with a smooth scalarization,
for which consistency follows from Astudillo \& Frazier~\cite{astudillo2019composite}.
The $K>1$ case is novel and requires additional arguments that we have not
been able to complete rigorously. We state it as a conjecture with supporting
intuition.

\begin{assumption}[RKHS regularity]
\label{asm:rkhs}
Each $f_i$ lies in the RKHS $\mathcal{H}_{k_i}$ of its kernel $k_i$,
with $\|f_i\|_{\mathcal{H}_{k_i}}\le B$.
The kernels satisfy $k_i(\mathbf{x},\mathbf{x})\le 1$ for all $\mathbf{x}\in\X$.
\end{assumption}

\begin{assumption}[Observation model]
\label{asm:obs}
Observations are $y_{i,t} = f_i(\mathbf{x}_t) + \varepsilon_{i,t}$
where $\varepsilon_{i,t}$ are i.i.d.\ $\sigma$-sub-Gaussian.
\end{assumption}

\begin{conjecture}[Consistency of qSTCH-Set BO]
\label{conj:consistency}
Under Assumptions~\ref{asm:gp}--\ref{asm:obs}, suppose the qSTCH-Set acquisition
function~\eqref{eq:acq} is optimized at each BO iteration $t$ with
$\boldsymbol{\lambda}\in\Delta^{m-1}_{++}$ and $\mu>0$ (possibly decreasing in $t$).
Let $\Xset_t^*$ denote the set selected at iteration $t$.
Then, as $t\to\infty$, with probability at least $1-\delta$, every
$\mathbf{x}^{(k)}\in\Xset_t^*$ is $\varepsilon_t$-Pareto optimal for the true
objectives $\mathbf{f}$, where
\begin{equation}
\label{eq:eps-rate}
\varepsilon_t = \mu_t\log(mK) + O\!\left(\beta_t^{1/2}\,\bar{\sigma}_t\right)
\xrightarrow{t\to\infty} 0
\end{equation}
if $\mu_t\to 0$ at an appropriate rate.
Here $\beta_t = O(B^2 + \gamma_t\log^3(t/\delta))$,
$\gamma_t$ is the maximum information gain of the kernel,
and $\bar{\sigma}_t = \max_{i\in[m]}\max_{\mathbf{x}\in\Xset_t^*}\sigma_i^{(t)}(\mathbf{x})$.
\end{conjecture}

\paragraph{Why we believe this but cannot prove it.}
The argument would need to combine three ingredients:

\begin{enumerate}
\item \textbf{GP posterior concentration (established).}
Under Assumptions~\ref{asm:rkhs}--\ref{asm:obs}, Srinivas et al.~\cite{srinivas2010ucb}
(Theorem~6) provide uniform confidence bounds:
with probability $\ge 1-\delta/m$ (union-bounded over objectives),
$|f_i(\mathbf{x})-\mu_i^{(t)}(\mathbf{x})|\le\beta_t^{1/2}\sigma_i^{(t)}(\mathbf{x})$
simultaneously for all $\mathbf{x}\in\X$ and $t\ge 1$.
The information gain $\gamma_t$ grows sub-linearly for standard kernels
(e.g., $\gamma_t = O((\log t)^{d+1})$ for the SE kernel), ensuring
$\beta_t^{1/2}\sigma_i^{(t)}(\mathbf{x})\to 0$ in well-explored regions.

\item \textbf{STCH-Set Lipschitz stability (straightforward but not formalized).}
The STCH-Set scalarization is Lipschitz in its objective values.
Let $L_\mu$ denote the Lipschitz constant of $g_\mu^{\mathrm{STCH\text{-}Set}}$
with respect to the objective vector $\bigl(f_i(\mathbf{x}^{(k)})\bigr)_{i,k}$
in the $\ell^\infty$ norm. Then, under the GP concentration event,
\begin{equation}
\bigl|g_\mu^{\mathrm{STCH\text{-}Set}}(\hat{\mathbf{f}}^{(t)}(\Xset))
 - g_\mu^{\mathrm{STCH\text{-}Set}}(\mathbf{f}(\Xset))\bigr|
\le L_\mu \cdot \beta_t^{1/2}\bar{\sigma}_t.
\end{equation}
A bound $L_\mu \le 2$ (with our normalization $\boldsymbol{\lambda}\in\Delta^{m-1}$)
can be obtained from the fact that both the log-sum-exp and
negative-log-sum-exp-of-negatives have gradient entries summing to~1,
but we have not verified all corner cases for the composition rigorously.

\item \textbf{Acquisition optimization is near-global (not provable in general).}
A complete consistency proof requires that the acquisition function is optimized to
within $\varepsilon$ of the global optimum at each step. This is assumed
(explicitly or implicitly) in all GP-based BO analyses, including GP-UCB~\cite{srinivas2010ucb},
EI~\cite{balandat2020botorch}, and composite BO~\cite{astudillo2019composite}.
In practice, multi-start L-BFGS-B provides this empirically but it is
not formally guaranteed for the non-convex acquisition landscape of qSTCH-Set
with $K>1$ candidates (the decision space $\X^K$ has dimension $Kd$).
\end{enumerate}

\paragraph{The $K=1$ case.}
When $K=1$, qSTCH-Set reduces to single-point STCH scalarization composed with the GP posterior.
This falls within the composite BO framework of Astudillo \& Frazier~\cite{astudillo2019composite},
who prove consistency for acquisition functions of the form $h(\mathbf{g}(\mathbf{x}))$
where $h$ is a known function and $\mathbf{g}$ is modeled by a GP.
In our case, $h = -g_\mu^{\mathrm{STCH}}$ and $\mathbf{g} = (f_1,\ldots,f_m)$.
Their Theorem~1 establishes that the composite EI acquisition function
is consistent (i.e., the optimization gap vanishes) under GP posterior consistency.
For $K=1$, our method inherits this guarantee directly.

\paragraph{The $K>1$ gap.}
For $K>1$, the decision variable is a \emph{set} $\Xset\in\X^K$,
and the scalarization $g_\mu^{\mathrm{STCH\text{-}Set}}$ involves a smooth minimum
over the $K$ candidate evaluations.
The composite BO framework of~\cite{astudillo2019composite} does not directly cover this case
because:
(i) the ``outer function'' $h$ now depends on the GP outputs at $K$ different input locations
jointly, not at a single $\mathbf{x}$; and
(ii) the set-valued optimization introduces symmetries and redundancies
(permutation invariance of $\Xset$) that complicate the convergence analysis.
Extending the Astudillo-Frazier consistency argument to this joint-set
setting is the main open theoretical challenge.
We conjecture it holds by analogy: the STCH-Set scalarization is continuous
and the GP posterior concentrates uniformly, so the set-valued optimization
problem should converge to its deterministic counterpart.
A rigorous proof would likely require set-valued epi-convergence arguments
(see, e.g., Rockafellar \& Wets, \emph{Variational Analysis}, Chapter~7).

%--------------------------------------------------------------------
\subsection{Computational Complexity}
\label{sec:theory-complexity}
%--------------------------------------------------------------------

\begin{proposition}[Per-evaluation complexity]
\label{prop:complexity}
The qSTCH-Set acquisition function with $K$ candidate points, $m$ objectives,
and $N$ MC base samples can be evaluated in $O(NKm)$ time
(excluding GP posterior sampling cost) and $O(Km)$ space per MC sample.
In contrast:
\begin{itemize}
    \item \emph{qEHVI}~\cite{daulton2020qnehvi,daulton2021qnehvi}:
        Hypervolume computation is $\#P$-hard in $m$; the best exact algorithms
        require time exponential in $m$ in the worst case.
    \item \emph{qNParEGO}~\cite{knowles2006parego,daulton2020qnehvi}:
        $O(Nm)$ per evaluation (linear in $m$), but uses a single
        random $\boldsymbol{\lambda}$ per iteration without coordinating solutions.
\end{itemize}
\end{proposition}

\begin{proof}
The STCH-Set computation~\eqref{eq:stchset-def}--\eqref{eq:smin-def} requires:
(1)~weighted deviations $\lambda_i(f_i(\mathbf{x}^{(k)})-z_i^*)/\mu$ for all $i\in[m]$,
$k\in[K]$: $O(Km)$ operations;
(2)~smooth min via $\mathrm{logsumexp}$ over $k$ for each $i$: $O(K)$ per objective,
$O(Km)$ total;
(3)~smooth max via $\mathrm{logsumexp}$ over $i$: $O(m)$.
The gradient~\eqref{eq:grad} can be computed by automatic differentiation
with the same asymptotic cost (using the PyTorch \texttt{logsumexp} implementation,
which applies the max-subtraction trick for numerical stability).
Summing over $N$ MC samples gives $O(NKm)$.
\end{proof}

%--------------------------------------------------------------------
\subsection{Summary of Theoretical Status}
\label{sec:theory-summary}
%--------------------------------------------------------------------

Table~\ref{tab:theory-status} summarizes what is proved, what is transferred from
existing results, and what remains conjectural.

\begin{table}[t]
\caption{Theoretical status of qSTCH-Set results.}
\label{tab:theory-status}
\centering
\small
\begin{tabular}{llp{7cm}}
\toprule
\textbf{Result} & \textbf{Status} & \textbf{Key Dependency} \\
\midrule
Prop.~\ref{prop:valid}: Valid MC acquisition
    & Proved
    & Composition of measurable/smooth functions \\
Prop.~\ref{prop:pareto-transfer}(a): Surrogate Pareto opt.
    & Proved (transfer)
    & Lin et al.~\cite{lin2025few}, Theorem~2 \\
Prop.~\ref{prop:pareto-transfer}(b): Sandwich bound
    & Proved
    & Standard log-sum-exp bounds~\cite{lin2024smooth} \\
Cor.~\ref{cor:annealing}: $\mu$-annealing
    & Proved
    & Immediate from Prop.~\ref{prop:pareto-transfer}(b) \\
Prop.~\ref{prop:complexity}: $O(NKm)$ complexity
    & Proved
    & Direct computation \\
Conj.~\ref{conj:consistency}: Consistency ($K>1$)
    & \textbf{Conjecture}
    & Extends~\cite{astudillo2019composite}; requires set-valued convergence \\
Conj.~\ref{conj:consistency} ($K=1$ case)
    & Proved (by~\cite{astudillo2019composite})
    & Composite BO consistency \\
\bottomrule
\end{tabular}
\end{table}
